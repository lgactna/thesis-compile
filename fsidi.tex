% \documentclass[preprint,12pt]{elsarticle}

%% Use the option review to obtain double line spacing
%% \documentclass[authoryear,preprint,review,12pt]{elsarticle}

%% Use the options 1p,twocolumn; 3p; 3p,twocolumn; 5p; or 5p,twocolumn
%% for a journal layout:
%% \documentclass[final,1p,times]{elsarticle}
%% \documentclass[final,1p,times,twocolumn]{elsarticle}
%% \documentclass[final,3p,times]{elsarticle}
% \documentclass[final,3p,times,twocolumn]{elsarticle}
%% \documentclass[final,5p,times]{elsarticle}
\documentclass[final,5p,times,twocolumn]{elsarticle}

%% Fixing DOIs in the bibtex file, which otherwise don't play well with elsarticle.bst
\usepackage{doi}

% Code listings. We use inconsolata for the font.
\usepackage{listings}
\usepackage{inconsolata}
\usepackage{color}
\usepackage{setspace}
\definecolor{codegreen}{rgb}{0,0.6,0}
\definecolor{codegray}{rgb}{0.5,0.5,0.5}
\definecolor{codepurple}{rgb}{0.58,0,0.82}
\definecolor{backcolour}{rgb}{0.95,0.95,0.92}

% Pandoc generates `passthrough' commands for inline code as described in 
% https://github.com/jgm/pandoc/issues/5696, we can just effectively ignore it
\newcommand{\passthrough}[1]{#1}

\lstset{
    backgroundcolor=\color{backcolour},   
    commentstyle=\color{codegreen},
    keywordstyle=\color{magenta},
    numberstyle=\ttfamily\tiny\color{codegray},
    stringstyle=\color{codepurple},
    basicstyle=\ttfamily\footnotesize\setstretch{1}, 
    breakatwhitespace=false,         
    breaklines=true,                 
    captionpos=b,                    
    keepspaces=true,                 
    numbers=left,                    
    numbersep=5pt,                  
    showspaces=false,                
    showstringspaces=false,
    showtabs=false,                  
    tabsize=2,
    aboveskip=10pt, % Adjust spacing above listings
    belowskip=0pt, % Adjust spacing below listings
}

% Less spacing in between items in \begin{itemize} environments.
\usepackage{enumitem}
\setitemize{noitemsep}

% Better table appearance, particularly for the long ones
% https://tex.stackexchange.com/questions/373362/how-to-avoid-justified-text-in-tabularx
\usepackage{tabularx}
\usepackage{ragged2e}
\usepackage{booktabs}
\newcolumntype{L}[1]{>{\hsize=#1\hsize\RaggedRight} X}

% No footnotes across pages.
\interfootnotelinepenalty=10000

% Set asset path.
\graphicspath{{assets/}}

\journal{Forensic Science International: Digital Investigation}

\begin{document}

\begin{frontmatter}

\title{AKF: A modern synthesis framework for building datasets in digital forensics
    \footnote{The contents of this paper have been derived from Lloyd Gonzales's master's thesis of the same name.}
}

\author[unr]{Lloyd Gonzales}
\author[unr]{Nancy LaTourrette}
\author[unr]{Bill Doherty}
%% Author affiliation
\affiliation[unr]{organization={Department of Computer Science and Engineering, University of Nevada, Reno},%Department and Organization
            addressline={1664 North Virginia Street}, 
            city={Reno},
            postcode={89557}, 
            state={Nevada},
            country={USA}}

%% Abstract
\begin{abstract}
%% Text of abstract
The forensic community depends on datasets containing disk images,
network captures, and other forensic artifacts for education and
research. However, real-world datasets often contain sensitive
information that may be difficult to remove, limiting their use and
distribution. This often leads to the manual development of new datasets
to cover gaps in available datasets. While viable, this approach is
time-consuming and rarely produces datasets that are fully reflective of
real-world conditions. In turn, there is ongoing research into forensic
synthesizers, which automate the process of creating unique, synthetic
datasets that can be publicly distributed without legal and other
logistical concerns.

This work introduces the automated kinetic framework, or AKF, a modular
synthesizer for creating and interacting with virtualized environments
to simulate human activity. AKF makes significant improvements to the
approaches and implementations of prior synthesizers used to generate
forensic artifacts. AKF also improves the process of documenting these
datasets by leveraging the CASE standard to provide human- and
machine-readable reporting. Finally, AKF offers several options for
using these features to build and document datasets, including a custom
scripting language and generative AI workflows. These contributions aim
to enhance the speed at which synthetic datasets can be created and
ensure the long-term usefulness of AKF-generated datasets and the
framework as a whole.
\end{abstract}

%% Graphical abstract
% \begin{graphicalabstract}
%\includegraphics{grabs}
% \end{graphicalabstract}

%% Research highlights: 3-5 points of no more than 85 characters
\begin{highlights}
\item Broad improvements to the automation of creating digital forensic artifacts
\item Use of modern libraries and technologies to simplify artifact generation
\item Automatic comprehensive dataset documentation through the CASE ontology
\item Simplified dataset creation using a custom scripting language and generative AI
\end{highlights}
    
%% Keywords
\begin{keyword}
%% keywords here, in the form: keyword \sep keyword
Forensic image creation \sep
User emulation \sep
Artifact documentation \sep
Dataset documentation \sep
Generative AI \sep
\end{keyword}

\end{frontmatter}

%% Beginning of main text
\section{Introduction}\label{introduction}

The investigation of cybercrime and other computer-related incidents
requires the collection of forensic datasets, which can broadly be
described as a collection of forensic artifacts. These artifacts are
typically contained in a specific medium, such as a disk image or
volatile memory dump. The collection and analysis of these artifacts are
central to the field of digital forensics; they must be conducted in a
manner that ensures the resulting evidence and conclusions are
admissible in a court of law.

From an education and training perspective, these datasets are critical
in introducing new specialists to the field. Datasets in an educational
setting typically cover a range of techniques and tools, allowing
students to practice applying theoretical concepts learned through
lectures and recitations
\cite{adelsteinAutomaticallyCreatingRealistic2005}. Besides the
specific technical skills covered by these scenarios, these datasets aim
to develop the analytical skills needed for students to adapt to
developments in tools and technology
\cite{cooperStandardsDigitalForensics2010}. In other words, students
should be familiar with standard tools and patterns in digital
forensics, providing a foundation on which more niche techniques can be
learned \cite{lawrenceFrameworkDesignWebbased2009}.

Similarly, from a research perspective, these datasets serve two
purposes. The first is in improving specific processes in the analytic
step of a forensic investigation. Examples include the development of
novel techniques for niche platforms and the improvement of existing,
well-established techniques. The second is in upholding the quality of
the investigation process as new techniques and tools to analyze
datasets are developed. One notable example is the Computer Forensics
Tool Testing program maintained by the National Institute of Standards
and Technology, which provides a standard methodology and test corpora
for evaluating specific forensic tool capabilities
\cite{nationalinstituteofstandardsandtechnologyComputerForensicsTool2017}.

However, it is challenging to provide hands-on labs that realistically
and comprehensively support the ideas learned in theoretical courses
\cite{adelsteinAutomaticallyCreatingRealistic2005,guptaDigitalForensicsLab2022,lawrenceFrameworkDesignWebbased2009}.
Similarly, researchers often have specific needs that are not covered by
existing public datasets. In many cases, these challenges can be
attributed to privacy and legal concerns. The sensitive information on
real-world datasets often limits their distribution and usage, even if
they would otherwise be suitable for use. As a result, the community
often uses \emph{synthetic} datasets, which are produced specifically
for educational and research purposes as described by Park and Garfinkel
et al.
\cite{parkTREDEVMPOPCultivating2018,garfinkelBringingScienceDigital2009}.
However, manually creating these datasets is a time-consuming process
that tends to produce datasets with a very narrow scope and limited
reproducibility
\cite{garfinkelBringingScienceDigital2009,grajedaAvailabilityDatasetsDigital2017}.

There is a clear need for a more streamlined, reproducible method of
developing datasets for research and education -- one that is able to
provide the variability and complex content needed for datasets to be
useful in a wide variety of use cases. One such option is the use of
\textbf{forensic synthesizers}, which aim to automate part or all of
dataset creation by programmatically creating forensic artifacts. Over
the past 15 years, there has been significant research into the
development of these synthesizers, each of which produces artifacts
through distinct approaches. Synthesizers are a subset of broader
automation efforts throughout digital forensics, as described by
Michelet et al. \cite{micheletAutomationDigitalForensics2023}.

Our work, the \textbf{automated kinetic framework}, or \textbf{AKF},
directly improves upon the foundations provided by prior synthesizers.
It achieves this by adding to the unique contributions of these past
synthesizers through modern techniques and technologies. In doing so, it
aims to provide the foundation of a larger forensic dataset ecosystem --
not only vastly reducing the time spent developing new datasets for
research and education, but also improving the documentation and
discoverability of these datasets. Additionally, through AKF's focus on
long-term viability through its modular architecture, educators and
researchers will be able to rapidly develop relevant datasets, even as
new developments and technological advancements occur. Our improvements
are designed to address several challenges observed in prior
synthesizers, specifically the inflexibility of older codebases, the
lack of standardized ground truth, and the difficulty of preparing and
using these synthesizers.

This paper begins with an analysis of related work in \autoref{related-work}, covering existing dataset sources and the contributions of prior
synthesizers. In \autoref{artifact-generation}, we introduce AKF's
improvements to creating forensic artifacts by directly modifying disk
images and interacting with virtualized environments. In
\autoref{artifact-generation}, we then describe how AKF documents these
generated datasets with minimal user overhead, providing both machine-
and human-readable options for identifying artifacts of interest.
Finally, in \autoref{dataset-construction}, we explore how AKF exposes
these artifact generation and documentation features in an accessible
manner, enabling users of all experience levels to benefit from AKF's
design. We conclude with an evaluation of AKF in \autoref{sample-demonstration}, as well as a summary of our work and the future
opportunities that exist in \autoref{conclusion-and-future-work}.

\section{Related work}\label{related-work}

We begin with a literature review of two topics. First, we review
existing datasets and the continuing needs of the field, demonstrating
why synthesizers are necessary. We then cover the specific gaps that
these synthesizers have gradually filled.

\subsection{Existing forensic corpora}\label{existing-forensic-corpora}

Forensic datasets have been available for public use (sometimes upon
request) since the early days of the digital forensics field, although
their sourcing and qualities have changed significantly over time. This
topic was explored in far greater detail by Grajeda et al., who
performed a survey of over 700 research articles to identify the various
datasets used throughout the field
\cite{grajedaAvailabilityDatasetsDigital2017}. However, it is still
important to highlight specific datasets relevant to the development of
synthesizers.

Many early datasets were derived from real sources, such as the used
hard drives collected by Garfinkel from 1998 to 2006, the Enron email
corpus obtained during the federal investigation of Enron, the public
Apache mailing archives, and facial recognition collections
\cite{garfinkelForensicCorporaChallenge2007,grajedaAvailabilityDatasetsDigital2017,yannikosDataCorporaDigital2014,ricanekMORPHLongitudinalImage2006}.
There were also some synthetic datasets during this period, such as the
standalone datasets in the early CFTT program developed by NIST and
those produced for DFRWS conferences
\cite{woodsCreatingRealisticCorpora2011}.

The variety of publicly available forensic datasets increased
considerably in the late 2000s, which can be attributed to both the
overall growth of the field (including the broader field of incident
response) and the advancement of computing as a whole. Various notable
datasets described by Grajeda et al.~include malware samples,
file-specific datasets such as collections of Microsoft Office files,
and datasets for mobile phones. It was also during this time that
non-disk datasets, such as volatile memory dumps and network captures,
became more prevalent. Many of these datasets have been aggregated into
platforms such as Digital Corpora
\cite{garfinkelBringingScienceDigital2009,yannikosDataCorporaDigital2014}
and the NIST CFReDS project, which are actively maintained corpora of
forensic datasets. In 2021, Xu et al.~compiled and published a
repository of educational datasets that explicitly focuses on ease of
use, realism, and breadth \cite{xuDesigningSharedDigital2022}.

These existing datasets are indeed expansive and serve as the basis for
much of the training and research material used today. However, the
field continually requires new datasets that reflect current
advancements in technology, such as updated operating systems or new
applications. For example, instant messaging applications have changed
considerably over the history of the field, ranging from MSN Messenger
in the early 2000s to Skype, Discord, Telegram, Signal, Slack, and more.
Artifacts from these applications must be handled differently, even if
the value of the underlying service to an investigator is practically
the same. Similarly, the strategies for analyzing Windows artifacts have
changed significantly from version to version, as new registry keys
become relevant in analyzing applications while others become unused.

This gradual ``aging'' of datasets, in which their relevance degrades
over time, highlights the continuous need for new datasets. Indeed, the
need for current datasets is one of the reasons identified by Grajeda et
al.~for the manual development of new datasets by research authors; it
was often the case that a modern dataset simply did not exist for their
needs. The datasets throughout this section have demonstrated that many
different kinds of datasets have proved to be relevant in digital
forensics research, even if not immediately evident. It follows that any
effort to streamline the development of forensic datasets should be able
to cover as many use cases as possible without requiring significant
changes to the underlying architecture.

Indeed, the synthesizers described in \autoref{analysis-of-prior-synthesizers} have made significant efforts to do so. For example, many
of these synthesizers have focused on implementing features that reflect
the qualities of real-world datasets. These features include the ability
to execute malware samples in a virtualized environment, generate
network captures, and extract application-specific artifacts. Also
notable is the generation of datasets derived from human interactions,
such as public email distribution lists, photographs of human faces, and
transcripts of conversations. While prior synthesizers and AKF have not
explored this in depth, it is briefly addressed as part of the
generative AI work done as part of AKF in \autoref{generative-ai-workflows}.

\subsection{Analysis of prior
synthesizers}\label{analysis-of-prior-synthesizers}

The functionality and availability of synthesizers have varied
considerably over time. However, the goal of these frameworks has
largely remained consistent: they all aim to reduce the effort required
to produce datasets for research and education. The earliest efforts to
streamline forensic lab development were observed in \textbf{FALCON}
\cite{adelsteinAutomaticallyCreatingRealistic2005} and
\textbf{CYDEST} \cite{bruecknerAutomatedComputerForensics2008},
published in 2005 and 2008 respectively, which aimed to simplify the
deployment and evaluation of virtual lab environments to students.

However, \textbf{Forensig2}, described by Moch and Freiling in 2009 and
revisited in 2012, appears to be the first detailed description of a
forensic synthesizer
\cite{mochForensicImageGenerator2009,mochEvaluatingForensicImage2012}.
Users define scenarios through a Python 2 library that provides
abstractions around various virtual machine (VM) operations, such as
formatting disks, creating partitions, and copying files from the host
to the virtual machine. Artifacts are created through a combination of
disk modifications and interactions with a live QEMU-based VM,
ultimately producing a disk image to be analyzed by students.

The vast majority of later synthesizers are similar to Forensig2 in that
they are all implemented in Python, exposing artifact generation
features to users through a Python library. Notable exceptions include
the \textbf{Digital Forensic Evaluation Test (D-FET)} platform
\cite{williamCloudbasedDigitalForensics2011}, the \textbf{Summarized
Forensic XML (SFX)} language
\cite{russellForensicImageDescription2012}, and the work described
by \textbf{Yannikos et al.} \cite{yannikosDataCorporaDigital2014}.
These synthesizers use custom languages for creating datasets, providing
alternative abstractions around their functionality.

A summary of the notable aspects of the remaining Python-based
frameworks -- many of which were identified by Göbel et al.
\cite{gobelForTraceHolisticForensic2022} -- is described below:

\begin{itemize}
\item
  \textbf{ForGeOSI} \cite{maxfraggMaxfraggForGeOSI2023} introduced
  the use of the VirtualBox SDK to automate various operations, forming
  the basis for much of the work done as part of VMPOP.
\item
  \textbf{ForGe} \cite{vistiAutomaticCreationComputer2015} is
  specifically designed to generate NTFS and FAT32 images, placing data
  directly onto disk images by maintaining and serializing custom data
  structures for supported filesystems.
\item
  \textbf{EviPlant} \cite{scanlonEviPlantEfficientDigital2017}
  encompasses a novel method for efficiently distributing generated disk
  images, which is achieved by generating and distributing differential
  ``evidence packages'' to apply to a base image file. An OS-specific
  injection tool generates relevant artifacts based on the evidence
  package. Since the base image must be downloaded only once, each
  reconstructed disk image is significantly smaller than if distributed
  as a standalone image.
\item
  \textbf{VMPOP} \cite{parkTREDEVMPOPCultivating2018} provides an
  architecture for elaborate VirtualBox control (such as attaching USB
  devices and starting video captures) in addition to various
  OS-specific commands. Provided routines for Windows include creating
  restore points, installing programs, mapping network drives, setting
  registry values, and more. Although the architecture as a whole is
  platform-independent, the provided implementations operate with
  VirtualBox and Windows. \textbf{TraceGen}
  \cite{duTraceGenUserActivity2021} is similar in that it utilizes
  the VirtualBox API for various operations, but is more like hystck in
  its use of a Python-based agent to carry out application-specific
  actions.
\item
  \textbf{hystck} \cite{gobelNovelApproachGenerating2020} provides
  routines for automating OS- and application-specific commands through
  both normal Python scripts and YAML configuration files (passed
  through an intermediate interpreter script). The framework produces
  network captures and disk images; similar to EviPlant, it supports
  ``differential'' images that can be distributed and applied to
  ``template'' images.
\item
  \textbf{ForTrace} \cite{gobelForTraceHolisticForensic2022} is the
  most recently developed synthesizer, which directly builds upon hystck
  by providing volatile memory captures (alongside disk and network
  captures) in addition to various other new features (such as the
  ability to execute PowerShell scripts to create Windows artifacts),
  with a focus on a modular architecture. A variant focusing on Android
  artifact generation was developed in 2024
  \cite{demmelDataSynthesisGoing2024}.
\end{itemize}

Clearly, significant work has been done to streamline the process of
developing images, although the availability and functionality of each
framework vary greatly. However, except for ForTrace and VMPOP, none of
these synthesizers are direct extensions of prior works, suggesting that
the framework authors' needs could only be met by developing new
frameworks from scratch. While this is not stated in any prior works
(except for hystck \cite{gobelNovelApproachGenerating2020}), the
limited maturity, availability, and maintenance of prior works likely
contributed to the independent development of most synthesizers. For
example, several synthesizers are not open source and thus cannot easily
be extended, as is the case with Forensig2
\cite{mochForensicImageGenerator2009} and TraceGen
\cite{duTraceGenUserActivity2021}. Another reason is that the focus
of certain synthesizers results in an architecture that is simply
incompatible with the goals of newer works. For example, ForGe's
architecture \cite{vistiAutomaticCreationComputer2015} focuses
primarily on direct filesystem manipulation to generate forensic
artifacts and is unsuitable for a synthesizer requiring virtualization.
Similarly, synthesizers that exclusively leverage agentless artifact
generation as described in \autoref{artifact-generation}, such as
VMPOP \cite{parkTREDEVMPOPCultivating2018}, require significant
architectural changes to support agent-based artifact generation.

However, perhaps the largest motivation for constructing new
synthesizers is the lack of ongoing support for virtually all
synthesizers. For the synthesizers that \emph{are} open source, none are
under active development and maintenance, possibly with the exception of
ForTrace. Additionally, the forensic datasets generated by these
synthesizers have not seen significant adoption in either education or
research; many instructors continue to use the human-generated datasets
available on public platforms.

The inflexibility of prior synthesizers, combined with the overall lack
of support and success of synthesizer-based datasets, contributes to the
lack of shared codebases. However, this is not to say that the
individual contributions of prior synthesizers cannot be merged into a
single project that resolves many of the architectural barriers that
have limited the adoption and extension of existing synthesizers. We now
move to a discussion of the three main areas in which AKF makes
significant improvements: artifact generation, artifact documentation,
and dataset creation.

\section{Artifact generation}\label{artifact-generation}

Each of the synthesizers described in \autoref{analysis-of-prior-synthesizers} takes one of three approaches to artifact generation, as
partly described by Scanlon et al.
\cite{scanlonEviPlantEfficientDigital2017}:

\begin{itemize}
\item
  \textbf{Physical}: No virtualization of software or hardware ever
  occurs; data is written directly to the target medium, such as a disk
  image or virtual hard drive.
\item
  \textbf{Agentless logical}: The synthesizer interacts with a live VM
  to generate artifacts. Interaction is achieved without the need for
  custom software to be installed on the VM; instead, actions are
  achieved using the hypervisor itself or a remote management tool
  native to the virtualized operating system (OS).
\item
  \textbf{Agent-based logical}: The synthesizer interacts with a
  dedicated client, or agent, on a live VM to carry out actions. The VM
  must have the agent installed before any interaction can occur.
\end{itemize}

These three approaches are not mutually exclusive within a single
synthesizer, though many prior synthesizers have supported only one
approach to generate artifacts. \autoref{tbl:prior-techniques} denotes
the approaches used by each of the synthesizers previously discussed.
Where source code is unavailable, a best effort was made to identify the
approach used by a particular synthesizer based on its published paper,
if one exists; otherwise, the entire row contains question marks.


\begin{table*}[tb]
\footnotesize
\centering
\begin{tabularx}{\linewidth}{L{0.2} L{0.2666} L{0.2666} L{0.2666}}
\toprule
  \textbf{Synthesizer} & \textbf{Physical} & \textbf{Agentless} & \textbf{Agent-based} \\
\midrule
  \textbf{FALCON} \cite{adelsteinAutomaticallyCreatingRealistic2005},
  2005 & ? & ? & ? \\
  \textbf{CYDEST} \cite{bruecknerAutomatedComputerForensics2008}, 2008
  & ? & ? & ? \\
  \textbf{Forensig2}
  \cite{mochForensicImageGenerator2009,mochEvaluatingForensicImage2012},
  2009 & Yes (filesystem mounting; filesystem-independent editing) & Yes
  (over SSH only) & No \\
  \textbf{D-FET} \cite{williamCloudbasedDigitalForensics2011}, 2011 &
  Yes (filesystem mounting; filesystem-independent editing) & No & No \\
  \textbf{SFX} \cite{russellForensicImageDescription2012}, 2012 & Yes
  (filesystem mounting) & No & No \\
  \textbf{Yannikos et al.} \cite{yannikosDataCorporaDigital2014}, 2014
  & ? & ? & ? \\
  \textbf{ForGeOSI} \cite{maxfraggMaxfraggForGeOSI2023}, 2014 & No &
  Yes (hypervisor interfaces) & No \\
  \textbf{ForGe} \cite{vistiAutomaticCreationComputer2015}, 2015 & Yes
  (filesystem-aware editing) & No & No \\
  \textbf{ForGen} \cite{jjk422Jjk422ForGen2019}, 2016 & No & No &
  No \\
  \textbf{EviPlant} \cite{scanlonEviPlantEfficientDigital2017}, 2017 &
  Yes (filesystem-independent editing) & No & Yes (unknown mechanism) \\
  \textbf{VMPOP} \cite{parkTREDEVMPOPCultivating2018}, 2018 & No & Yes
  (hypervisor interfaces) & No \\
  \textbf{hystck} \cite{gobelNovelApproachGenerating2020}, 2020 & No &
  No & Yes (Python agent) \\
  \textbf{TraceGen} \cite{duTraceGenUserActivity2021}, 2021 & No & No
  & Yes (unknown mechanism) \\
  \textbf{ForTrace} \cite{gobelForTraceHolisticForensic2022}, 2022 &
  No & No & Yes (Python agent) \\ \\
\bottomrule
\end{tabularx}
\caption{Summary of artifact generation techniques in prior synthesizers}\label{tbl:prior-techniques}
\end{table*}


There are advantages and disadvantages to each approach, in addition to
requiring distinct implementation techniques. Although AKF makes
improvements in all three techniques, we only discuss physical and
agentless generation, as they are where AKF makes the largest
improvements. More specifically, this section addresses the
implementation of the action automation library
(\passthrough{\lstinline!akflib!} \footnote{\passthrough{\lstinline!akflib!}
  is publicly available at \url{https://github.com/lgactna/akflib}}) to
generate artifacts by either directly editing disk images or interacting
with a live virtual machine.

\subsection{Physical generation}\label{physical-generation}

Physical artifact creation encompasses any technique in which the
virtualization of an operating system is not used to generate artifacts.
This allows the synthesizer to bypass the operating system or related
software that could lead to undesirable non-deterministic behavior. This
is sometimes referred to as \emph{simulating} the creation of artifacts
rather than \emph{virtualizing} their creation.

For example, a scenario developer may want to guarantee that a
particular deleted file is partially overwritten by another file,
ensuring that the deleted file is recoverable from the slack space of
the newly placed file. However, it is extremely difficult to force the
reuse of the same physical disk clusters from a userspace application.
Operating systems rarely expose low-level filesystem functionality to
applications; furthermore, operating systems are still subject to
hardware drivers that regularly rearrange physical space, such as those
that engage in wear leveling. In turn, it is sometimes necessary to
avoid virtualization to reliably place data on a disk image.

There are three primary techniques for physical artifact planting that
have been implemented by prior synthesizers. These are:

\begin{itemize}
\item
  \textbf{Filesystem mounting}, as done by Forensig2
  \cite{mochForensicImageGenerator2009} and SFX
  \cite{russellForensicImageDescription2012}, in which the
  filesystem is mounted to the host and edited directly.
\item
  \textbf{Filesystem-independent direct editing}, as done by EviPlant
  \cite{scanlonEviPlantEfficientDigital2017}, in which edits to
  specific physical addresses on the disk image are made without any
  parsing or knowledge of the underlying filesystem.
\item
  \textbf{Filesystem-aware direct editing}, as done by ForGe
  \cite{vistiAutomaticCreationComputer2015} and EviPlant
  \cite{scanlonEviPlantEfficientDigital2017}, in which filesystem
  data structures are parsed to determine the physical address(es) of
  the disk image to write to. (It is unclear how EviPlant achieves this,
  as its source code is not available.)
\end{itemize}

AKF supports all three physical techniques to varying degrees; however,
AKF makes the largest improvements in filesystem-aware direct editing.
ForGe \cite{vistiAutomaticCreationComputer2015} implements this by
implementing NTFS and FAT32 data structures in Python, allowing it to
create a fully virtual representation of these filesystems. This
virtualized filesystem provides ForGe with the information needed to
insert data into known slack and unallocated space while maintaining
filesystem consistency. While extremely powerful (and implements a
valuable feature not found in any other synthesizer to date), it is
inflexible in two specific aspects:

\begin{itemize}
\item
  ForGe does not provide a generic interface for the filesystems it
  supports. Although the NTFS and FAT32 wrappers provide the same
  methods with identical signatures, this is not strictly enforced by a
  parent class. The lack of a generic interface means that the
  functionality supported across all filesystems is unclear, as is the
  functionality that must be implemented for new filesystems to be
  compatible with ForGe.
\item
  ForGe lacks a ``frontend'' to support arbitrary disk types, regardless
  of the underlying filesystem. ForGe does not support multi-partition
  disks or common non-raw disk formats such as VHD, VMDK, or VDI.
\end{itemize}

Read-only libraries addressing these two issues have been in development
since the introduction of ForGe but have not been integrated into other
synthesizers to achieve the same write capabilities as ForGe. Two such
libraries are \passthrough{\lstinline!libtsk!} (also known as The Sleuth
Kit), the C++ library that powers the open-source digital forensics
software Autopsy \cite{SleuthkitSleuthkit2025}, and
\passthrough{\lstinline!libyal!}, a collection of libraries for
analyzing formats not supported by \passthrough{\lstinline!libtsk!}.
These libraries eventually supported the development of
\passthrough{\lstinline!dfvfs!}, or the Digital Forensics Virtual File
System, a Python library that leverages \passthrough{\lstinline!libtsk!}
and multiple libraries from the \passthrough{\lstinline!libyal!} project
to provide a generic interface for analyzing a variety of disk image
formats and filesystems \cite{Log2timelineDfvfs2025}.

AKF uses \passthrough{\lstinline!dfvfs!} to locate the clusters of a
file at a known path in a filesystem, which can then be used to identify
the start of slack space within the file's clusters (as well as
unallocated space in the filesystem). By adding the physical offset of
the cluster within the filesystem to the offset of the filesystem's
partition in the disk image, AKF can write to the exact location of a
file's slack space using the offset-based method described earlier. This
achieves feature parity with ForGe by completely delegating filesystem-
and image-specific details to \passthrough{\lstinline!dfvfs!}, enabling
a filesystem-independent method for locating slack space.

More generally, this technique provides a deterministic method for
inserting data within the slack space of a filesystem, simulating the
deallocation of a file and its partial replacement with a known file.
However, this does not fully simulate the process of deleting a file
through a running operating system and having a new file replace the
deallocated clusters; naturally, this does not generate OS-specific
artifacts associated with deleting and creating files and fails to
generate the filesystem artifacts that could exist with the original
file (such as the ``deleted'' file's name in an NTFS master file table).
Future work in the field could address this gap by combining
\passthrough{\lstinline!dfvfs!} with additional techniques to improve
its consistency with OS operations.

\subsection{Agent-based generation}\label{agent-based-generation}

\textbf{Agent-based artifact creation} involves the use of a dedicated
executable on a VM. This executable serves as an interface between the
host machine and the guest machine. This program runs commands natively
on the VM on behalf of the host machine, accepting commands over a
dedicated network interface. This allows for greater flexibility and
more complex actions to be taken when compared to agentless approaches.
In particular, it allows application-specific functionality to be
implemented using existing automation frameworks such as Selenium
\cite{SeleniumHQSelenium2025}, Playwright
\cite{MicrosoftPlaywrightpython2025}, and PyAutoGUI
\cite{sweigartAsweigartPyautogui2025}, rather than implementing such
functionality from scratch.

Agent-based artifact creation is the approach taken by hystck/ForTrace
\cite{gobelNovelApproachGenerating2020,gobelForTraceHolisticForensic2022},
which refers to its agent as an ``interaction manager.'' Because
ForTrace provides extensive agent functionality and is by far the most
mature synthesizer, AKF's agents borrow heavily from ForTrace's
approach. However, AKF improves upon ForTrace by addressing two specific
implementation issues:

\begin{itemize}
\item
  Commands in ForTrace are sent through a simple string-based protocol.
  In particular, it is challenging to send complex Python objects as
  arguments or return values, as these objects often cannot be easily
  serialized to a string without losing information. An example relevant
  to AKF is passing Playwright browser objects, which contain an
  internal state that is difficult to extract and reconstruct using
  strings alone.
\item
  ForTrace depends on Python's runtime introspection to discover the
  correct modules and functions to call based on the contents of a
  command string, importing these modules during runtime. While this is
  a valid approach, it is more complex than the approach taken by AKF.
\end{itemize}

AKF resolves these issues through the use of RPyC, a library for
symmetric remote procedure calls, for agent communication
\cite{TomerfilibaorgRpyc2025}. Although the RPyC protocol is
symmetric, it is often used in typical client-server architectures to
allow clients to manipulate remote Python objects as if they were local
objects, as well as invoke remote (server) functions using local
(client) parameters. Delegating the serialization and deserialization of
complex objects to RPyC enables us to perform complex operations that
would have been difficult to implement with the simple string-based
protocol of ForTrace.

In ``new-style'' RPyC, this is achieved by running a \emph{service} on
the device where remote operations should be performed. Services expose
a set of functions and attributes that can be accessed remotely by an
RPyC client, doing so by setting up a TCP listener for requests to
access these exposed objects. Clients access these functions and
attributes by name as if they were local objects; arguments passed to
functions are serialized and deserialized in the background, as are the
results of function calls and attribute accesses.

AKF's application-specific functionality is divided into individual RPyC
``subservices'' created on demand. These subservices implement
automation support for a specific application or group of actions and
are analogous to the agent-side code of individual ForTrace modules. The
agent's main loop is itself an RPyC service responsible for creating and
destroying these subservices upon request. All subservices are known to
this ``dispatch'' service at initialization, eliminating the need for
runtime introspection to find application-specific modules. A high-level
diagram of this design can be seen in \autoref{fig:agent-modular}.

\begin{figure}[htbp]
\centering
\includegraphics[width=1\linewidth]{agent-modular.png}
\caption{AKF agent architecture}\label{fig:agent-modular}
\end{figure}

From an implementation and usability perspective, this design provides
three significant improvements over ForTrace. First, the routing of
functions is wholly delegated to RPyC. Instead of manually constructing
a message with the function name and its associated parameters (as
strings) over the network, the process of serializing parameters and
routing them to the correct function call is abstracted away by RPyC.

Second, this allows us to pass and return arbitrarily complex objects
(for which we do not have to manually write the serialization and
deserialization logic). When passing complex objects from the agent to
the server or vice versa, a reference to the object is sent over the
network and wrapped by a \emph{proxy object}, which behaves like the
original object \cite{TheoryOperationRPyC}. Importantly, it is
usually not necessary to distinguish between local and remote/proxy
objects of the same type when writing code, which eliminates the extra
complexity of using proxies.

Finally, the ability to interact with complex remote objects allows us
to significantly reduce the actual code written as part of the API
exposed to the host. For example, there is no need to implement a
wrapper for every method available as part of a Playwright page object;
instead, a reference to the Playwright object \emph{running on the
virtual machine} can be given to the host machine. Instead of writing
individual methods for opening pages, navigating to specific elements,
and so on, we can use the methods that already exist in the Playwright
object -- any local calls on the host's proxy object will lead to remote
outcomes on the host, as desired. This, of course, does not preclude the
ability to write convenience methods for more complex actions requiring
the Playwright object.

Together, these three features significantly simplify the process of not
only implementing and managing support for individual applications, but
also the actual use of the agent. Simplified application-specific
support makes it easier for the community to extend AKF. Furthermore,
this modular approach to application-specific support enables the
addition or removal of functionality from the agent with minimal effort.
Since not all automation frameworks are available on every platform or
operating system, these ``subservices'' enable us to provide support for
as many different platforms as possible with minimal changes to the
underlying codebase.

The list of subservices supported by the AKF Windows agent\footnote{The
  AKF Windows agent, or \passthrough{\lstinline!akf-windows!}, is
  publicly available at \url{https://github.com/lgactna/akf-windows}} is
described in \autoref{tbl:akf-applications} below. Although only three
subservices are implemented, each subservice is an example of a distinct
design pattern that could be easily adapted to implement other
application-specific functionality. (Support for specific applications,
such as Thunderbird or Firefox, has already been explored and
implemented in prior works such as ForTrace.)


\begin{table*}[tb]
\footnotesize
\centering
\begin{tabularx}{\linewidth}{L{0.2} L{0.2} L{0.6}}
\toprule
  \textbf{Subservice} & \textbf{Dependencies} & \textbf{Features} \\
\midrule
  \passthrough{\lstinline!autogui!} & PyAutoGUI
  \cite{sweigartAsweigartPyautogui2025a} & Hypervisor-independent
  mouse and keyboard control, as well as other PyAutoGUI features \\
  \passthrough{\lstinline!artifacts!} & Windows-Prefetch-Parser
  \cite{wittPoorBillionaireWindowsPrefetchParser2025} & Collection of
  Windows artifacts and conversion to corresponding CASE objects \\
  \passthrough{\lstinline!chromium!} & Playwright
  \cite{MicrosoftPlaywrightpython2025} & Automated webpage browsing;
  also allows for performing complex actions such as completing forms and
  clicking links based on HTML selectors \\ \\
\bottomrule
\end{tabularx}
\caption{Implemented subservices for the AKF Windows agent}\label{tbl:akf-applications}
\end{table*}


A generic hypervisor interface is used to support agent discovery and
communication. To avoid polluting network captures with agent-related
packets, virtual machines are expected to use a NAT adapter for Internet
communications and a ``maintenance'' host-only adapter for
agent-specific communications. In turn, hypervisor-specific
implementations must expose the ability to discover the IP address of
the host-only adapter. This allows AKF scripts to communicate with the
root RPyC service and any subservices over the host-only adapter,
concealing them from network captures on the NAT adapter.

\section{Artifact documentation}\label{artifact-documentation}

There exists a gap in the ability of instructors and researchers to
perform bulk searches for specific forensic artifacts in public
datasets. For example, the NIST CFReDS repository
\cite{nationalinstituteofstandardsandtechnologyCFReDSPortal}, one of
the largest listings of public forensic datasets, does not have a
unified standard for describing uploaded images. Although users can
search by keywords and human-applied tags, detailed artifact information
is not available in a standardized format that can be programmatically
queried.

For many datasets, an instructor or researcher must read through a PDF
answer key (if one exists) or analyze the image themselves to determine
if a particular artifact is present. Answer keys are not inherently
machine-readable and are not suited for identifying specific artifact
types in bulk. Additionally, the content of human-made reports may be
limited to what the author believes is significant, even if other
artifacts of interest are present in the image. In turn, it may be
difficult to quickly determine if a dataset is useful in demonstrating a
particular technique to students or validating a specific feature of a
newly developed tool.

Now that we are able to generate artifacts using the techniques
described in \autoref{artifact-generation}, how do we document what was
generated? This section describes not only our approach to providing
machine- and human-readable reporting for AKF-generated datasets, but
also AKF's role in providing a foundation for reproducible forensic
research.

\subsection{CASE bundles}\label{case-bundles}

A rigid, well-defined format for ground truth is invaluable to
researchers engaging in tool validation and development. We identified
CASE, developed by Casey et al.
\cite{caseyAdvancingCoordinatedCyberinvestigations2017}, as the best
format to meet these needs. CASE is a vendor-neutral format designed to
document both technical and non-technical information about a digital
forensics case. It aims to provide as many definitions as possible for
OS-specific and application-specific artifacts while still providing the
flexibility to describe artifacts from uncommon applications. These
definitions are written in the Terse RDF Triple Language, also known as
Turtle, which expresses object attributes and types in a plaintext
format. Instances of these objects are expressed in a CASE ``bundle,''
which is typically serialized into a format such as JSON-LD.

Because the CASE format itself is language-agnostic, it is necessary to
write language-specific libraries that enable the instantiation of CASE
objects. At the time of writing, the CASE project provides official
Python bindings for CASE version 1.4
\cite{CaseworkCASEMappingPython}. Each unique object type is
represented as a Python class, which can be instantiated to produce
individual objects. However, this library has several limitations,
particularly the need to manually maintain these definitions due to the
instantiation and serialization logic contained in each class.

AKF contributes and leverages its own bindings for CASE\footnote{AKF's
  CASE bindings are publicly available at
  \url{https://github.com/lgactna/case-pydantic}}. Its foundation is the
Pydantic library for Python, which allows developers to easily define
classes with typed attributes based on Python type hints
\cite{colvinPydantic2024}. This allows us to vastly simplify the
declaration of individual CASE objects while providing runtime type
validation and automatic casting. More importantly, this simplicity
allows us to automatically generate our Python bindings directly from
the Turtle definitions. This is particularly relevant when considering
the active development of CASE version 2.0, which has significant
differences from version 1.4.

Various functions and classes throughout the AKF core libraries and
agent API accept an optional CASE bundle when invoked or instantiated.
As CASE-compatible functions are called, they can automatically add CASE
objects corresponding to the artifacts generated through their
execution. For example, if the agent subservice API for automating
Chromium browser actions is provided with a CASE bundle, navigating to a
page using the API can also generate a CASE object describing the page
visit and add it to the bundle. This process can occur entirely within
the host, allowing CASE-related logic to remain out of the agent where
needed.

Other functions are dedicated entirely to creating CASE objects based on
runtime analysis. For example, it is possible to create objects for
Prefetch artifacts as soon as an action occurs. However, these objects
are likely to become outdated if their corresponding applications are
launched later in the scenario, thus changing the content of the
prefetch files and making the existing CASE objects inaccurate. In turn,
the \passthrough{\lstinline!artifacts!} subservice mentioned in
\autoref{agent-based-generation} can collect Windows prefetch
files immediately before a disk image is created, allowing it to
construct CASE prefetch objects that reflect the disk image without
requiring separate tooling. CASE-oriented functionality can also be
implemented in existing subservices; for example, the
\passthrough{\lstinline!chromium!} subservice can create CASE objects
for Chrome and Edge browser history after all browser automation actions
have been performed.

The flexibility of these two approaches -- enabled by CASE's deep
integration into AKF -- makes it possible to construct CASE objects in a
manner that requires little additional effort from scenario developers.
In most cases, scenario developers do not need to be concerned with
instantiating their own CASE objects when using high-level APIs so long
as an AKF library developer has written support for automatic CASE
object construction. This significantly reduces the need for scenario
developers to construct ground truth information by manually analyzing
synthesizer-created outputs.

By extension, this means that the detailed documentation of AKF outputs
is innate to many scenarios constructed using AKF. Lowering the effort
required to document an AKF-generated scenario improves the likelihood
that any public AKF scenario can be immediately valuable (or determined
to be valuable) to researchers and educators. This significantly
contributes to AKF's goal of supporting an ecosystem around its images;
the CASE bundles of many scenarios can be queried in bulk to identify
datasets that might be useful for a specific purpose without having to
download the dataset itself. This information can also be used to
identify and analyze broader trends across scenarios, such as the
frequency of a particular artifact appearing in all Windows datasets.

While this machine-readable reporting significantly improves the ability
of the forensic community to locate useful datasets, it is verbose and
unsuitable as a human-readable summary. Human-readable reporting is
particularly relevant in a classroom setting, where the distribution of
simplified answer keys to graders and students focusing on key artifacts
is preferable to the exhaustive reporting provided by a CASE bundle.
This leads us to the following section, which addresses the conversion
of AKF-generated metadata into human-readable reports.

\subsection{PDF reporting}\label{pdf-reporting}

Converting a rigid, well-defined format to a human-readable format is
often easier than performing the reverse operation. Indeed, this is the
approach taken by AKF, which does not create human-readable reports as
an immediate output of dataset generation. Instead, AKF supports a
simple yet flexible system for generating human-readable PDF reports
from existing CASE bundles after generating a dataset.

AKF implements human-readable reporting through a set of ``renderers.''
Each renderer accepts a complete CASE bundle and extracts all CASE
objects of a particular type. The renderer then uses the information
contained in these objects to generate a Markdown document, which can
include formatted text, tables, images, and other visual elements. The
results of each renderer are combined to form a larger Markdown document
(or documents) with multiple sections, one for each renderer. The
combined document can be converted to a PDF using Pandoc
\cite{macfarlanePandoc2025}, a general-purpose tool for converting
between documents of various types. Users can modify the generated
Markdown documents before running Pandoc if desired.

A single CASE bundle can be passed through as many or as few renderers
as needed to generate a suitable report for a dataset, as depicted in
\autoref{fig:case-reporting}. So long as the original CASE bundle is
available, users can reanalyze datasets with arbitrary renderers; this
means that dataset reports can be regenerated with as much detail as a
user needs for a specific use case. Furthermore, if new renderers are
developed for artifacts that are present in older datasets, the
human-readable report can be regenerated to include these artifacts.

\begin{figure}[htbp]
\centering
\includegraphics[width=1\linewidth]{case-reporting.png}
\caption{Diagram of modular rendering system}\label{fig:case-reporting}
\end{figure}

This modular, ``evergreen'' approach to reporting allows these reports
to be interpreted as a focused snapshot of what a dataset contains.
Importantly, this can be done without compromising the dataset itself;
the CASE bundle remains the single, comprehensive source of truth.
Compare this with human-written PDF reports, which may contain human
biases and are rarely maintained in older datasets.

\autoref{fig:pdf-sample} shows parts of an AKF-generated report for the
sample dataset described in \autoref{sample-demonstration}. Note that
\autoref{fig:pdf-sample} uses the Eisvogel template
\cite{waglerWandmalfarbePandoclatextemplate2025} for Pandoc,
significantly improving the appearance and readability of generated
documents.

After generating forensic artifacts and any documentation that should be
included with the dataset, the challenge of distributing this
information remains. More precisely, how do we make our dataset as
accessible, reusable, and discoverable as possible?

\subsection{Reproducibility}\label{reproducibility}

A key challenge identified by Grajeda et al.~was the difficulty in
reproducing results in the field of digital forensics. While this is
primarily attributed to the \emph{availability} of forensic datasets in
general, it can also be attributed to challenges in the
\emph{reproducibility} of creating synthetic datasets. Before addressing
the low-level use of AKF as part of \autoref{dataset-construction}, we
will briefly discuss the infrastructure needed to support community
usage of the outputs of AKF scenarios and synthetic datasets as a whole.

There are four elements that must be distributed with a scenario to make
a dataset (and its results) reproducible:

\begin{itemize}
\item
  Any core outputs or individual artifacts generated from the virtual
  machine.
\item
  Any metadata, ground truth, or other reporting that describes the
  scenario.
\item
  The precise instructions required to build the scenario from the
  provided base image, whether human- or machine-readable instructions.
\item
  The OS-specific ``base image'' used to create the dataset, typically a
  virtual machine or disk with a newly installed operating system on
  which all synthesizer actions are performed.
\end{itemize}

The first two have long been a part of many public datasets. However,
less common are detailed instructions to rebuild the dataset from
scratch. A high-level timeline of actions taken, such as that provided
by Woods et al.~in their educational dataset
\cite{woodsCreatingRealisticCorpora2011}, is too imprecise to
guarantee that others will recreate the dataset in the exact same
manner. Non-determinism can be acceptable and even desirable in
education contexts, but it is less desirable for tool validation and
research. Synthesizers address this through their machine-readable
scripts, which document and execute the instructions needed to
reconstruct a dataset. However, this also depends on the availability of
an OS- and synthesizer-specific base image, which may not always be
included with datasets due to legal or logistical constraints.

AKF is designed to provide all four of these elements in every scenario
it creates; elements 1, 2, and 3 are inherent to AKF's design, while
base images can be provided as Vagrantfiles as described in
\autoref{setup-and-usage}. This ensures the reproducibility of both the
creation of AKF datasets and any results derived from them.

The inclusion of these elements in a standardized distribution format
could be used to build a distribution platform similar to CFReDS with
more powerful discovery and querying functionality. While CASE enables
queries on a dataset's artifacts, it is also valuable to query the
contents of Vagrantfiles and AKF scripts. For example, a user may want
to search for all images that use the agent-based Chromium artifact
generation described in \autoref{agent-based-generation}, which can be
achieved by searching for the inclusion of the relevant AKF libraries in
the scenario's scripts. This does not address the challenge of storing
and distributing scenarios efficiently to support such a platform; this
is discussed briefly in \autoref{conclusion-and-future-work}.

With the reproducibility and value of AKF-generated scenarios
established, we now discuss how to invoke and leverage the underlying
technologies that provide these benefits.

\section{Dataset construction}\label{dataset-construction}

At this point, we have provided the functionality for automating
artifact generation in a near-deterministic manner with comprehensive
logging and reporting. However, there is still the challenge of exposing
this functionality in a user-friendly manner. This section explores the
various improvements AKF makes in simplifying the initial setup process,
writing scripts to create new datasets, and converting abstract ideas
into a coherent script.

\subsection{Setup and usage}\label{setup-and-usage}

Like many of its predecessors, AKF implements its functionality and
exposes its API in Python 3. There are numerous advantages to a
Python-based API; besides the relatively low difficulty of setting up
and using Python, its rich ecosystem allows dataset creation to be
extended through other libraries from the Python ecosystem.

Users must install two foundational technologies for AKF to operate:
Python 3.11 or later and a supported hypervisor (currently, only
VirtualBox). AKF uses \passthrough{\lstinline!pyproject.toml!} to define
Python library dependencies, which can be installed into a virtual
environment using a package manager such as
\passthrough{\lstinline!pip!} or \passthrough{\lstinline!uv!}.

At this point, a virtual machine must be prepared for use with AKF. As
with prior synthesizers, it is possible to manually configure a machine
by downloading a supported operating system and creating a new virtual
machine from scratch. The manual process, similar to that of other
synthesizers, involves installing the operating system, configuring
network interfaces, installing the AKF agent on the device, and then
configuring it to run in the background on startup. The resulting
virtual machine can be cloned and reused for multiple datasets as
needed.

Although relatively straightforward, this process is still
time-consuming, especially when adapted to new operating systems. While
a prepared AKF virtual machine can be distributed in a virtual appliance
format such as OVF, this can run into legal issues if the software on
the underlying operating system is copyrighted. To help resolve this,
AKF uses modern infrastructure-as-code solutions to vastly simplify the
setup of new virtual machines. Vagrant, developed by HashiCorp, is a
tool for rapidly building development environments
\cite{HashicorpVagrant2025}. It allows users to define and build
virtual machines on several virtualization platforms, including
VirtualBox and VMWare. Virtual machines are built by configuring a base
image according to a Vagrantfile, which describes hypervisor-specific
configuration options and instructions to configure the machine. The
Vagrantfile can be distributed to users, allowing them to build the same
virtual machine without needing to download full virtual drives.

The AKF Windows agent includes a Vagrantfile for creating a new Windows
11 virtual machine with the agent installed and configured, which can
easily be adapted for other platforms and hypervisors. The
Vagrantfile(s) used to generate a dataset should be included with the
dataset itself to maximize reproducibility, as described in \textbf{\#
Reproducibility}. A robust ecosystem of Vagrant boxes exists for Linux
and Windows, many of which can be retrieved from the Vagrant public
registry \cite{hashicorpHashiCorpCloudPlatform}. When combined with
the flexibility of Vagrant across multiple virtualization platforms,
this can significantly improve the reproducibility and usability of AKF.
It should also be noted that Vagrant can configure and build larger
environments with multiple machines. For organizations that can express
their corporate environments as Vagrantfiles, AKF can perform artifact
generation at scale, allowing for incident response datasets that
reflect real-world networks and events.

Following setup, developers can build scenarios using the AKF core
libraries (\passthrough{\lstinline!akflib!}) and the API of the
platform-specific agent installed onto the virtual machine (such as
\passthrough{\lstinline!akf\_windows!}). This reflects typical
imperative usage, in which environment setup, artifact generation, and
output generation are handled explicitly through a script executed
through the Python interpreter. A Python script demonstrating web
browsing and disk image creation can be seen in \autoref{lst:6.2a}.

\begin{lstlisting}[label={lst:6.2a}, caption={Example of an imperative AKF scenario}, language=Python]
from akf_windows.api.chromium import ChromiumServiceAPI
from akflib.core.hypervisor.vbox import VBoxExportFormatEnum
from akflib.core.hypervisor.vbox import VBoxHypervisor
from pathlib import Path

# Instantiate a hypervisor object tied to a specific virtual machine
vbox_obj = VBoxHypervisor("akf-windows-1")

# Start the virtual machine
vbox_obj.start_vm(wait_for_guest_additions=True)

# Visit a single website
with ChromiumServiceAPI.auto_connect(vbox_obj.get_maintenance_ip()) as chromium_service:
    chromium_service.kill_edge()
    chromium_service.set_browser("msedge")
    page = chromium_service.browser.new_page()
    page.goto("bbc.co.uk")

# Stop the virtual machine
vbox_obj.stop_vm(force=False)

# Export the virtual machine to a disk image
vbox_obj.create_disk_image(
    Path("C:/Users/user/Desktop/akf-windows_1.raw"),
    VBoxExportFormatEnum.RAW
)
\end{lstlisting}

\subsection{The AKF scripting
language}\label{the-akf-scripting-language}

Although powerful, not everybody needs the flexibility of an imperative
programming language like Python, where the user specifies \emph{how}
artifacts must be created in a step-by-step manner. This brings us to
declarative languages, in which the user only specifies \emph{what}
artifacts must exist in the final dataset, and the synthesizer
determines \emph{how} to generate the artifacts. More precisely,
declarative scripts encapsulate the same functionality that could be
achieved by writing code, but expose this in a simpler format. Similar
concepts can be seen in automation frameworks like Ansible, which allows
users to perform complex actions by writing simple YAML scripts.

In designing the AKF declarative syntax, the declarative syntaxes of
prior synthesizers and unrelated technologies were evaluated. The two
syntaxes that contributed most to the AKF declarative syntax were those
of ForTrace and Ansible; in particular, the modular nature of both
syntaxes was adapted to AKF, as was the overall structure of Ansible's
YAML ``playbooks.''

Declarative scripts are comprised of metadata, global configuration, a
set of libraries to import, and individual tasks to execute as part of
the scenario. Each task refers to a single \emph{module} by name,
accepting a dictionary of arguments that determine how the module
behaves. Each module encapsulates some specific functionality, such as
visiting a group of websites or generating a disk image. AKF's
interpreter \passthrough{\lstinline!akf-translate!}, which implements
support for our language, parses the script and does one of two things:

\begin{itemize}
\item
  \textbf{Execution}: When instructed to perform actions directly from
  the declarative script, the interpreter should import AKF core
  libraries and agent APIs to perform the required actions encapsulated
  by each task.
\item
  \textbf{Translation}: When instructed to translate the declarative
  script, the interpreter should generate the equivalent code that
  \emph{would} perform the required actions if executed through a
  standard Python interpreter with the necessary libraries installed.
\end{itemize}

The ability of AKF to both execute and translate declarative scripts
provides significant flexibility to scenario developers. To the best of
our knowledge, prior synthesizers have only supported direct execution
from declarative scripts, which limits the opportunities to use
declarative scripts as a ``starting point'' for writing more complex
imperative scripts. (In fact, the code in \autoref{lst:6.2a} was derived
from the script shown in \autoref{lst:6.3.2a}.) An example of a minimal
AKF scenario, carrying out the same actions as the imperative AKF script
in the prior section, can be seen in \autoref{lst:6.3.2a}:

\begin{lstlisting}[label={lst:6.3.2a}, caption={Example of a declarative AKF scenario}, ]
name: Minimal scenario
description: Browses to the BBC website and exports a disk image.
author: User
seed: "0"
libraries:
  - akflib.modules
  - akf_windows.modules
actions:
  - name: Instantiate a hypervisor object tied to a specific virtual machine
    module: vbox_start
    args:
      machine_name: "akf-windows-1"
  - name: Start the virtual machine
    module: vbox_start_machine

  # Visit a website using Microsoft Edge. A temporary instance of the Chromium
  # subservice API is created for the lifetime of this module.
  - name: Visit a single website
    module: chromium_visit_urls
    args:
      browser: "msedge"
      urls: 
       - "bbc.co.uk"

  # Stop the virtual machine and export the virtual machine to a disk image.
  - name: Stop the virtual machine
    module: vbox_stop_machine
    args:
      force: false
  - name: Export the virtual machine to a disk image
    module: vbox_create_disk_image
    args:
      output_path: "C:/Users/user/Desktop/akf-windows_1.raw"
      image_format: "raw"
\end{lstlisting}

The execution flow of \passthrough{\lstinline!akf-translate!} itself is
straightforward. Given a path to a YAML script, the interpreter loads
the necessary libraries and configuration keys defined in the file,
instantiating resources accordingly. Then, the interpreter runs each
module under the \passthrough{\lstinline!actions!} key with the provided
arguments and configuration in order, continuing until all actions have
been processed. Modules can read and modify a global state dictionary,
allowing otherwise independent modules to cooperate. This is
particularly useful in allowing for ``outputs,'' such as CASE bundles,
to be passed and gradually constructed across modules. This design
allows for context-aware code generation and action execution.

These modules can be located in any library so long as they can be found
through Python's import system. For example, both
\passthrough{\lstinline!akflib!} and the AKF Windows agent contain their
own declarative module libraries, leveraging functionality specific to
each code repository. All modules in the script are located and
``cached'' at the start of script execution, which allows for validation
and runtime efficiency. This design allows declarative modules to be
written independently of the libraries they depend on, reducing the
``impact'' of supporting declarative features on the core imperative
libraries. In fact, this independence allows for the AKF module system
to be used in general-purpose scripting, similar to Ansible; it is not
tightly bound to the creation of forensic scenarios and artifacts. That
said, the existing declarative modules of AKF expose nearly all existing
functionality provided by \passthrough{\lstinline!akflib!}, including
PDF report generation, virtual machine interaction, and more.

The list of declarative modules available through
\passthrough{\lstinline!akflib!} and the Windows agent is described in
\autoref{tbl:akf-declarative-modules} below. Note that these modules are
referred to by their alias, not their fully-qualified module paths.


\begin{table*}[tb]
\footnotesize
\centering
\begin{tabularx}{\linewidth}{L{0.3} L{0.7}}
\toprule
  \textbf{Name} & \textbf{Description} \\
\midrule
  \passthrough{\lstinline!create\_akf\_bundle!} & Creates a new CASE
  bundle for use throughout the declarative scenario. \\
  \passthrough{\lstinline!write\_akf\_bundle!} & Write the contents of the
  currently active CASE bundle to disk as a JSON-LD file. \\
  \passthrough{\lstinline!render\_akf\_bundle!} & Pass the currently
  active CASE bundle through a set of provided renderers and construct a
  valid PDF using Pandoc. \\
  \passthrough{\lstinline!vbox\_start!} & Create a new VirtualBox instance
  bound to a specific virtual machine by name. \\
  \passthrough{\lstinline!vbox\_start\_machine!} & Power on the currently
  active virtual machine. \\
  \passthrough{\lstinline!vbox\_stop\_machine!} & Power off the currently
  active virtual machine. \\
  \passthrough{\lstinline!vbox\_create\_disk\_image!} & Export a disk
  image of the current virtual machine. \\
  \passthrough{\lstinline!artifact\_service\_start!} & Remotely start and
  connect to the subservice responsible for collecting Windows
  artifacts. \\
  \passthrough{\lstinline!artifact\_service\_stop!} & Disconnect from the
  subservice responsible for collecting Windows artifacts. \\
  \passthrough{\lstinline!prefetch!} & Analyze all prefetch files on disk
  and construct their corresponding CASE objects. \\
  \passthrough{\lstinline!chromium\_service\_start!} & Remotely start and
  connect to the subservice responsible for interacting with Chromium
  browsers. \\
  \passthrough{\lstinline!chromium\_service\_stop!} & Disconnect from the
  subservice responsible for interacting with Chromium browsers. \\
  \passthrough{\lstinline!chromium\_visit\_urls!} & Visit one or more URLs
  using a specified web browser. \\
  \passthrough{\lstinline!chromium\_history!} & Collect browsing history
  from a specified web browser and construct their corresponding CASE
  objects. \\ \\
\bottomrule
\end{tabularx}
\caption{Available AKF declarative modules}\label{tbl:akf-declarative-modules}
\end{table*}


Although these declarative modules (and the imperative library) provide
users with significant flexibility in \emph{using} AKF, there remains
the challenge of building artifacts and scenarios to use with AKF. The
following section addresses this challenge.

\subsection{Generative AI workflows}\label{generative-ai-workflows}

Users of synthesizers must still perform a significant amount of work
when generating individual artifacts. For example, although AKF and
other synthesizers can streamline the process of placing and generating
artifacts, users must still provide some of the artifacts themselves.
For example, if a user wants to include an email chain or other
conversation in the dataset, the user would need to provide the entirety
of the conversation themselves. Such conversations would need to be
consistent with the ``theme'' of a scenario.

This is particularly relevant when adding background noise intended to
emulate benign activity; a real user's device would have many email
conversations irrelevant to a particular scenario and would
significantly contribute to the realism of a synthetic dataset. Existing
datasets, such as the Data Leakage Case produced by NIST, contain many
documents that are not thematically consistent. For example, the
``technical documents'' present in the scenario are actually files from
the general-purpose Govdocs corpora
\cite{garfinkelBringingScienceDigital2009}, with a cover page
denoting their intended role in the scenario. One PowerPoint file is
portrayed as a presentation of the detailed design of the product, but
its actual contents are that of a Yale University presentation on brain
physiology.

Recent advancements in AI models have made it significantly easier to
generate text, images, and other media from high-level descriptions that
are consistent with a broader theme. Indeed, this can be used to produce
individual artifacts; generative AI models such as DeepSeek-R1
\cite{deepseek-aiDeepSeekR1IncentivizingReasoning2025} and SDXL 1.0
\cite{podellSDXLImprovingLatent2023} can be used to produce
standalone artifacts that can then be used as part of a dataset. For
example, a large language model (LLM) can be directed to create
technical documentation, email conversations, or image generation
prompts that are related to a corporate espionage scenario.

However, perhaps a larger challenge is determining the specific actions
that must be performed to create a dataset consistent with some larger
theme. Even with the automation features provided by AKF and the ability
to create individual artifacts through generative AI, it still takes
time to turn an idea into a specific sequence of actions. This is where
AKF's simple, modular declarative scripting language is particularly
powerful. In particular, there are two features of the declarative
syntax that lends itself well to this process:

\begin{itemize}
\item
  The role of each module is well-defined and corresponds one-to-one to
  a specific action that a human might take, such as visiting a set of
  websites.
\item
  The expected inputs for each module are well-defined, as they are
  documented by the required Pydantic argument model for each module.
\end{itemize}

In the same way that declarative scripts are an abstraction around
imperative code, an LLM can be used to facilitate natural language
prompts as an abstraction around creating declarative scripts. To
explore this, an instance of \passthrough{\lstinline!deepseek-r1:32b!}
was provided with the following information. (The actual prompts and
outputs used to derive the content in this section are included with the
scenario examples in the agent GitHub repository.)

\begin{itemize}
\item
  An overview of the purpose of the declarative syntax.
\item
  The overall structure of the syntax, such as the required top-level
  keys and the structure of individual actions under the
  \passthrough{\lstinline!actions!} key.
\item
  A Markdown-formatted table in which each row contains the name,
  description, and arguments of available modules. Except for the
  arguments column, the table was equivalent to
  \autoref{tbl:akf-declarative-modules}.
\item
  A prompt delimited by \passthrough{\lstinline!<scenario\_prompt>!}
  tags that should be used to build the overall scenario.
\end{itemize}

We observed that this information was generally sufficient for
DeepSeek-R1 to ``understand'' the AKF syntax, allowing it to generate
simple declarative scripts. Although the prompt containing this
information was manually written, much of the prompt could be used as a
template and substituted with the contents of automatically generated
documentation. For example, the table of available modules could be
generated through existing code metadata, such as the contents of
docstrings and Pydantic argument models.

In one notable case, DeepSeek-R1 was instructed to create a simple
scenario where a user visited news-related websites with Microsoft Edge
and entertainment-related websites with Google Chrome, ensuring that the
machine underwent multiple power cycles throughout the scenario.
Although DeepSeek-R1 was not provided with examples specific to invoking
the \passthrough{\lstinline!chromium\_visit\_urls!} module, it correctly
constructed an argument dictionary containing several websites that are
thematically consistent for both browsers. Additionally, it correctly
used the \passthrough{\lstinline!vbox\_start\_machine!} and
\passthrough{\lstinline!vbox\_stop\_machine!} actions to perform
multiple power cycles.

However, DeepSeek-R1 struggled to understand the interactions between
related modules without additional guidance. For example, it did not
infer that the use of the \passthrough{\lstinline!vbox\_stop\_machine!}
and \passthrough{\lstinline!vbox\_start\_machine!} actions required that
a hypervisor instance had been previously created using
\passthrough{\lstinline!vbox\_start!}. Even when the module descriptions
were modified to explicitly state these requirements, it continued to
invoke \passthrough{\lstinline!vbox\_start\_machine!} without a
preceding \passthrough{\lstinline!vbox\_start!} action. In contrast,
DeepSeek-R1 appeared to understand that
\passthrough{\lstinline!chromium\_service\_start!} should be called
before calling any other Chromium-related modules, even when this
requirement was not stated. Indeed, the Chromium-related modules
support, but do not require, the use of
\passthrough{\lstinline!chromium\_service\_start!} to explicitly connect
to the Chromium RPyC subservice.

Additionally, DeepSeek-R1 would sometimes invoke modules incorrectly,
such as providing arguments that did not exist or invoking them with the
incorrect name. One notable example was that it incorrectly used
\passthrough{\lstinline!vbox\_start\_machine!} to both start and stop
the virtual machine, adding an erroneous
\passthrough{\lstinline!action: "stop"!} argument when stopping the
virtual machine. In some cases, DeepSeek-R1 deviated significantly from
the declarative syntax and used keys that did not exist, such as
specifying module arguments outside of the
\passthrough{\lstinline!args!} dictionary.

Despite these issues related to correctness, it was clear that
DeepSeek-R1 could convert abstract prompts into a sequence of actions
based on the available modules provided. Much like using declarative
scripts as a starting point for imperative scripts, these outputs can
serve as a foundation for more complex declarative scripts. Furthermore,
these issues could be alleviated by improving the details of the prompt,
such as providing specific examples of module usage and the overall
results they cause. In particular, DeepSeek-R1 is a reasoning model,
which means that it ``thinks out loud'' before responding; the model's
thoughts could be used to identify and eliminate vagueness or
uncertainty in the provided prompt. Writing precise, detailed
documentation with this LLM-oriented pipeline in mind could further
improve the quality of generated outputs.

\section{Sample demonstration}\label{sample-demonstration}

\subsection{Scenario overview}\label{scenario-overview}

To demonstrate several of AKF's features and the value of an
AKF-generated dataset, we will now explore a simple scenario in which a
user visits websites, saves photos, and then downloads and runs
ransomware. Our ransomware simply encrypts the contents of the Downloads
folder before sending the key details and a unique identifier to a
remote web server over HTTP. The ransomware was written in Python and
bundled into an executable with PyInstaller, then uploaded publicly to
Google Drive, with a link posted on Pastebin.

As part of this scenario, we generated a disk image, network traffic
capture, and a volatile memory dump. These contain the information
needed to both understand how the ransomware was downloaded and reverse
the encryption performed by the ransomware. The scripts and other
resources used to construct the scenario are included with the
akf-windows repository; the resulting dataset is available on CFReDS.

To prepare a suitable virtual machine for this scenario, we used a
Vagrantfile to automate the process of creating and configuring a
Windows 11 VirtualBox machine with the AKF agent installed. We
configured two network adapters for the machine: a NAT adapter connected
to the internet and a host-only adapter for communicating with the
agent. We then enabled packet capturing over the NAT adapter, which
would allow us to capture the key sent by the ransomware. Finally, we
created a CASE bundle for use throughout the scenario, which will
include some of the relevant artifacts generated as part of this
dataset.

Once the machine was turned on, we directed the agent to visit a series
of posts on the Reddit platform according to a predefined list of URLs.
This was achieved by using the Chromium subservice, which uses
Playwright to interact with webpages. For each post, we saved a
screenshot of the page to the Downloads folder, which the ransomware
would later encrypt. After visiting several Reddit posts, we had the
agent navigate to and download the ransomware. We then used the
PyAutoGUI subservice to run the ransomware in File Explorer using a
sequence of keystrokes, emulating how a human would normally run the
ransomware. Shortly after, we created a volatile memory dump of the
machine (with the ransomware process still running), then turned the
machine off and created a raw disk image of the machine. The CASE
bundle, along with a PDF report detailing the bundle's contents, was
also exported. Relevant sections of the PDF report can be seen in
\autoref{fig:pdf-sample}.

\begin{figure}[htbp]
\centering
\includegraphics[width=1\linewidth]{pdf-sample-scenario.png}
\caption{Snippet of generated PDF report}\label{fig:pdf-sample}
\end{figure}

\subsection{Analysis}\label{analysis}

Now, we turn to a manual analysis of the generated dataset. As part of
this, we will establish the following:

\begin{itemize}
\item
  That the generated artifacts are consistent with the actions specified
  in the script.
\item
  That the generated artifacts are consistent with the documentation
  generated by AKF (when analyzed using external tools).
\item
  That it is possible to reverse the operations performed by the
  ransomware by utilizing multiple aspects of the dataset, specifically
  the disk image, volatile memory dump, and network capture.
\end{itemize}

First, we wanted to verify that the web browsing artifacts were
consistent with the list of URLs specified in the script. We began by
extracting the SQLite database used to store Microsoft Edge browsing
history and viewed the \passthrough{\lstinline!urls!} table containing
browsing history, as shown in \autoref{fig:browser-db}. Indeed, the
entries of the History file are consistent with the URLs specified, as
well as the links shown in the PDF report in \autoref{fig:pdf-sample}.
By extension, they are also consistent with the JSON-serialized CASE
bundle.

\begin{figure}[htbp]
\centering
\includegraphics[width=1\linewidth]{browser-db.png}
\caption{Contents of Microsoft Edge SQLite
database}\label{fig:browser-db}
\end{figure}

We also used Eric Zimmerman's PECmd tool to analyze the prefetch entries
contained in the disk image, allowing us to verify the results
documented in the generated PDF (and the CASE bundle). As expected, both
indicate that a file called \passthrough{\lstinline!cats.exe!} was run,
as was Microsoft Edge. More generally, we noted that although the run
times and actual executable names were consistent, the execution counts
recorded by AKF were not. Indeed, AKF uses a different library than
PECmd, which may explain the discrepancy between the two.

It is worth noting that several artifacts contained in the Microsoft
Edge database and the Prefetch folder are specific to AKF. These
artifacts, in addition to other AKF ``indicators,'' will be discussed
later.

To demonstrate the value of each part of the dataset in a digital
forensics, incident response, and reverse engineering context, we will
now attempt to decrypt the contents of the Downloads folder using only
the data provided in the dataset. This process involves many skills that
are characteristic of a good educational dataset.

First, we can observe that the ransomware is still present in the disk
image located in the user's Downloads folder. The compiled Python
bytecode can be extracted using a tool such as pyinstxtractor
\cite{extremecoders-reExtremecodersrePyinstxtractor2025}, which can
then be passed to a decompiler such as PyLingual
\cite{wiedemeierPYLINGUALPerfectDecompilation2024}. Upon
decompilation, we can determine the mechanism by which files are
encrypted, as well as the information required to decrypt them. Analysis
reveals that the key information is generated randomly at runtime and
then forwarded to a remote server.

To retrieve the information necessary for decryption, we can analyze
either the packet capture or the memory dump independently. If we choose
to analyze the packet capture, we can search for the domain requested by
the ransomware, as indicated in the decompiled code. Doing so reveals
the exact parameters required to reconstruct the key, as shown in
\autoref{fig:wireshark-analysis}.

\begin{figure}[htbp]
\centering
\includegraphics[width=1\linewidth]{wireshark-analysis.png}
\caption{Relevant web request observed in packet
capture}\label{fig:wireshark-analysis}
\end{figure}

Similarly, we can search the memory dump for the key information. Using
a tool such as the Volatility framework or a hex editor, we can analyze
the dump to discover the same JSON in memory as seen in the packet
capture. \autoref{fig:hex-analysis} depicts the JSON as seen in a hex
editor.

\begin{figure}[htbp]
\centering
\includegraphics[width=1\linewidth]{hex-analysis.png}
\caption{Relevant data seen in sample RAM dump}\label{fig:hex-analysis}
\end{figure}

At this point, a decryption program can be written based on the
decompiled ransomware logic. To demonstrate that the information
contained in the dataset is sufficient to write a decryption script, we
passed the decompiled ransomware to a large language model and asked it
to produce a decryption script. The resulting script, after
simplification, was successfully used to decrypt the contents of the
Downloads folder. This script is included with our dataset.

\subsection{Discussion}\label{discussion}

It is clear that even with a scenario as simple as this, there is
educational value in this AKF-generated scenario. Indeed, there are
several artifacts that could be used to teach specific concepts, such as
reverse engineering the ransomware, using tools like Volatility to
analyze the memory dump, or manually inspecting the contents of a SQLite
database. Taken together, this dataset can be used to evaluate the
skills learned by students throughout an incident response course, for
example.

Several observations can also be made regarding the presence of
artifacts specific to synthesizer use. First, as mentioned previously,
several applications that were explicitly installed as part of the
scenario were present in both Prefetch reports. In particular, multiple
processes associated with the VirtualBox Guest Additions were present.
Additionally, the default user created by Vagrant was still present, as
were various PowerShell scripts executed by our Vagrant during setup.
These detract from the realism of the disk images but do not affect the
analysis performed above. Additionally, many of these artifacts are
likely to be consistent across Vagrant-produced virtual machines and can
likely be documented and ignored for most purposes. Notably, the AKF
agent was absent from the Prefetch reports produced by either AKF or
PECmd.

Similar observations can be made about the volatile memory dump. Most
processes are realistic, with the exception of the AKF agent and the
VirtualBox Guest Additions. However, we observed no synthesizer-related
traffic in our packet capture, likely because of our separate network
interface for agent communication.

More generally, the topic of identifying artifacts generated as a side
effect of using a synthesizer has been explored previously by Moch and
Freiling in their discussion of Forensig2
\cite{mochForensicImageGenerator2009,mochEvaluatingForensicImage2012}.
They came to similar conclusions, noting that evidence of synthesizer
use is easily discoverable; indeed, the use of virtualized hardware and
an agent is likely to produce many artifacts that would not be present
in a real-world dataset.

\section{Conclusion and future work}\label{conclusion-and-future-work}

Public forensic datasets are invaluable to advancing research and
education throughout digital forensics. However, high-quality datasets
are presently few in number and may not fit specific needs, motivating
the development of new datasets. Constructing these datasets by hand is
time-consuming and prone to errors; yet, it remains the primary method
through which new datasets are created. In turn, there is a need for
synthesizers, which allow users to create datasets using high-level
scripting languages that can automate many common actions. Although
prior synthesizers have addressed the creation of specific forensic
artifacts, there are still opportunities to improve their flexibility
and usability while promoting their usage throughout the forensic
community.

Our work, the \emph{automated kinetic framework} (AKF), introduces a
modern approach to forensic synthesis through a modular architecture
focusing on sustained development and community adoption. It provides
significant advancements in artifact generation, logging and validation,
and the overall construction of new scenarios. It improves artifact
generation by integrating numerous technologies not leveraged by prior
synthesizers, greatly simplifying the implementation of the overall
architecture without compromising the breadth of features available
through the framework. By using a centralized logging architecture and
the CASE ontology, AKF can generate detailed, queryable metadata for the
datasets it produces, enabling users to quickly identify artifacts of
interest within a dataset. AKF also exposes a simple declarative syntax
for generating artifacts, allowing users to develop scenarios without
writing code using the underlying Python libraries. This declarative
syntax can be used with generative AI to further streamline scenario
development, both when constructing individual artifacts and when
building a complete scenario. Finally, we provide a sample ransomware
dataset that highlights various features of the framework.

Of course, there are still limitations to what AKF can accomplish.
Numerous opportunities exist to leverage existing and emerging
technologies to extend the framework. For example, recent advancements
in AI have significantly lowered the barrier to automating open-ended
tasks. One notable example is OpenAI's Operator, an ``agent'' capable of
automating tasks on webpages using only natural language prompts. The
ability to automate GUI-based applications based on high-level prompts
is particularly valuable for synthesizers, as this can dramatically
reduce the time needed to implement artifact generation for new
applications or operating systems. Such functionality is complementary
to ``conventional'' synthesizers, which depend on well-defined scripts
as opposed to natural language. For artifacts that do not require
deterministic, verifiable creation -- such as ``background noise'' --
AI-based generation may be preferable to the verbose scripts used to
generate datasets today.

There are also several practical limitations in our current
implementation of AKF. For example, AKF currently only supports
VirtualBox machines running Windows as its target environment. Support
for other hypervisors can be easily achieved through new implementations
of the hypervisor-agnostic interface and minor modifications to the
AKF-provided Vagrantfile. Similarly, consistent with AKF's focus on
using existing automation frameworks through agents to implement
application-specific functionality, much of the effort for supporting
other platforms involves writing and deploying a new OS-specific agent.
Much of the code and overall design can be inherited from the Windows
agent, enabled through the portability of Python, the lack of
OS-specific assumptions made by RPyC, and the cross-platform support of
many automation frameworks.

AKF does not address two notable topics that are prevalent in the field
of dataset creation. The first topic is the distribution of generated
datasets, which is particularly important in environments with limited
bandwidth or storage space. Several synthesizers have explored this
topic; for example, SFX \cite{russellForensicImageDescription2012}
reduces dataset sizes by truncating irrelevant information in a process
called ``partition squeezing,'' while EviPlant
\cite{scanlonEviPlantEfficientDigital2017} uses a form of
differential imaging in which only differences from some base image are
distributed to users over the network. Such approaches could be
implemented as part of AKF in the future.

The second topic is the application of synthesizers to the field of
mobile datasets, which are comparable in importance to desktop datasets.
There has been limited work in developing forensic synthesizers for
mobile platforms. Two notable examples are FADE
\cite{ceballosdelgadoFADEForensicImage2022}, developed by Delgado et
al.~in 2022, and a branch of ForTrace developed by Demmel et al.~in 2024
\cite{demmelDataSynthesisGoing2024}. It may be possible to apply
some of the approaches used as part of AKF in mobile dataset creation,
but more research is needed to determine viable options for streamlining
the construction of both Android and iOS datasets.

There is no doubt that continuing advancements in technology will
improve the process of constructing complex datasets for digital
forensics. Our contributions have been made with the hope that they will
not only form the basis of future developments in dataset synthesis, but
also advance research and education throughout digital forensics.

%% Beginning of appendices
\appendix

%% Using bibtex to generate items, expecting article_bib.bib
\bibliographystyle{elsarticle-num} 
\bibliography{article_bib}

\end{document}