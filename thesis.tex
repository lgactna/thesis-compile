\documentclass[letterpaper,12pt]{report}

% == Packages ==========================================================
\usepackage{amsmath}
\usepackage{amssymb}
\usepackage{amsfonts}
\usepackage{latexsym}
\usepackage{lipsum}
\usepackage{multicol}
\usepackage{accents}
\usepackage{wrapfig}
\usepackage{float}
\usepackage{tabularx}
\usepackage{euscript}
\usepackage{color}
\usepackage{xcolor}
\usepackage{graphicx}
\usepackage{float}
\usepackage{subcaption}
\usepackage{epsfig}
\usepackage{hyperref}
\usepackage{setspace}
\usepackage{fancyhdr}
\usepackage[T1]{fontenc} 
\usepackage{hyperref}
\usepackage[left = 1.5in, right = 1in, top = 1in, bottom = 1.25in, head = 0.5in]{geometry}
\usepackage{textcomp}
\usepackage{cite}
\usepackage{pdfpages}
\usepackage[linesnumbered,lined,boxruled,commentsnumbered]{algorithm2e}
\usepackage[acronym]{glossaries}
\graphicspath{{Figures/}}
\usepackage{attachfile}

% Below are ones I addded
% Native table support
\usepackage{booktabs}

% Pandoc uses tightlist for bullet lists, you can just do nothing
% See https://tex.stackexchange.com/questions/257418/error-tightlist-converting-md-file-into-pdf-using-pandoc
\def\tightlist{}

% == End packages

\renewcommand*\ttdefault{pcr}


\makeglossaries
\loadglsentries{lloyd_gonzales_thesis_glossary.tex}

\bibliographystyle{IEEEtran}

% == Customizations ====================================================
\allowdisplaybreaks

% == Page Style ========================================================
\pagestyle{fancyplain}
\renewcommand{\headrulewidth}{0.0pt}				% gets rid of lines on header
\fancyhead{}								      	% clears all header and footer fields
\fancyfoot{}
\fancyhead[R]{\thepage}						      	% inserts page number in top right

\begin{document}

% == Use fake numbering until abstract =================================
\pagenumbering{Alph}

% == Title Page +=======================================================
\newpage
\thispagestyle{empty}
\singlespacing

\begin{center}

\null

\vspace{1.5in}

University of Nevada, Reno \\

\vspace{1.5in}

\textbf{AKF: A modern synthesis framework for building digital forensic datasets}

\vspace{1.5in}

A thesis submitted in partial fulfillment of the\\
requirements for the degree of Master of Science in\\
Computer Science and Engineering\\

\vspace{1in}

by

\vspace{0.25in}
Lloyd Gonzales
\vspace{0.5in}

Nancy LaTourrette, Advisor \\
May 2025\\

\end{center}

% == Committee Approval =================================================

% == Set page numbering style ============================================
\pagenumbering{roman}
\setcounter{page}{0}

% == Abstract ===========================================================
\newpage
\onehalfspace

\begin{abstract}

\thispagestyle{fancyplain}
\doublespacing
As our world becomes increasingly dependent on technology, the
advancement of digital forensics has become a key focus in the fight
against cybercrime. The field depends greatly on the availability of
disk images, network captures, and other forensic datasets for
education, tool validation, and research. However, real-world datasets
often contain sensitive information that may be difficult to remove,
making them difficult to distribute publicly. As a result, researchers
and educators can encounter gaps in available datasets, often leading to
the manual development of new, suitable datasets. While viable, this
approach is time-consuming and rarely produces datasets that accurately
reflect real-world scenarios suitable for comprehensive training and
education. In turn, there is ongoing research into forensic
synthesizers, which automate the process of creating unique synthetic
datasets that can be publicly distributed without legal and logistical
concerns.

This thesis introduces the \emph{automated kinetic framework}, or AKF, a
modular synthesizer for creating and interacting with virtualized
environments to simulate user activity. AKF significantly improves upon
the architectural designs of prior synthesizers while maintaining
feature parity and usability. Additionally, AKF leverages the CASE
standard to provide human- and machine-readable reporting, exposing
low-level details in a searchable format. Finally, AKF provides options
for leveraging generative AI to develop high-level scenarios as well as
individual artifacts. These contributions are intended to not only
improve the speed at which synthetic datasets can be created, but also
ensure the long-term usefulness of AKF-generated datasets and the
framework as a whole.
\end{abstract}

% == Set page number ====================================================
\setcounter{page}{2}
\doublespacing


% == Dedication ==========================================================

% \begin{center}

\section*{Dedication}

To those in the osu! tournament community, without whom I would have
never embarked on this journey;

To my numerous teachers and professors, especially Keith Lightfoot,
Rodney Rogers, Marc Miller, and Gabbi Bachand, who I have limitless and
appreciation and admiration for;

To those part of the United States Cyber Team and the broader CTF
community, for igniting my interest in digital forensics and supporting
me even when I flailed like a fish out of water;

And, of course, to my friends, family, and bed, who provided motivation
when there was none.

% \end{center}

% == Acknowledgments ====================================================
\newpage

\section*{Acknowledgments}

I want to express my immense gratitude to Nancy Latourrette for her
support, guidance, and mentorship throughout the development of this
thesis. This thesis would be nowhere without her ideas and experience,
and I am truly grateful and honored to have been able to work with her
throughout this experience.

I would also like to thank Bill Doherty for his review of a prior paper,
from which some of this content is derived.


% == Table of contents ===================================================
\newpage
\renewcommand*\contentsname{Table of Contents}
\tableofcontents

\newpage
\listoftables

\newpage
\listoffigures

% == Set new page number style ===========================================
\newpage
\pagenumbering{arabic}
\setcounter{page}{1}
\linespread{2}

% == Begin thesis content
\chapter{Introduction}\label{chapter-one}

\section{History of digital
forensics}\label{history-of-digital-forensics}

Digital forensics is a relatively new field that rose out of the growing
need to address computer crimes. Through the explosion of computing that
has occurred over the last 30 years, computers have become critical in
virtually every modern industry. In turn, they have become the target of
many attacks. Much like conventional crimes, these cyberattacks leave
behind traces of digital evidence that can be analyzed to determine the
methodologies of the attackers, the scope and extent of the damage, and
more. Today, digital forensics is also applicable to a variety of
non-criminal contexts, including research and corporate investigations,
further highlighting the need for quality educational material and
hands-on training for new forensic analysts.

The origins of digital forensics can be traced to the 1980s, during
which computers began to be adopted by the general public. Prior to
this, computers were largely restricted to industry, academia, and
governments with dedicated infrastructure and staff
\cite{pollittHistoryDigitalForensics2010}. However, with the
introduction of PCs more accessible to the typical consumer, such as the
Commodore 64 and the IBM PC, computer usage within the public grew. And
in turn, so did crimes committed with computers; people discovered that
they could hack telephone networks to illegally obtain software and
``free'' telephone services \cite{jonesInsightDigitalForensics2022}.
In 1983, Canada would be the first government to amend its criminal code
to cover cybercrime; the United States, the United Kingdom, and
Australia would follow suit in the following decade.

During this time, investigations were relatively ad-hoc and very simple.
Teams often built their own software to analyze devices, though a few
hobbyists and software vendors began to develop dedicated forensic
tools. However, computers had yet to truly enter the mainstream; the
vast majority of individual computer owners were (relatively wealthy)
computer hobbyists rather than the general public. In 1984, the U.S.
Census Bureau determined that only about 8.2\% of U.S. households owned
a computer \cite{robertkominskiComputerUseUnited1988}, nearly
doubling to 15\% in 1989 \cite{robertkominskiComputerUseUnited1991}.

As technology continued to advance throughout the 1990s, digital
forensics -- and computing as a whole -- began to quickly grow in scope
and importance. The rise of mobile devices and the internet drastically
changed the role of computing in the eyes of the public. With it came
the rise of cybercrime and the importance of computer investigations.
Early investigations were often performed by investigators who happened
to be experienced with computers, rather than those with formal digital
forensics training \cite{hargreavesDigitalForensicsEducation2017}.

By the turn of the century, digital forensics had grown beyond law
enforcement and niche cases to a major focus of research and education.
The first use of the phrase ``computer forensics'' in ACM literature
appeared in 1999 \cite{cooperStandardsDigitalForensics2010}, with
the first Digital Forensic Research Workshop (DFRWS) held in 2001 to
identify priorities in the growing field of digital forensics
\cite{palmerRoadMapDigital2001}. A particularly notable case was
that of the September 11, 2001 attacks, in which computers were found to
contain meaningful evidence related to the organization and planning of
the attack. Intelligence communities and law enforcement agencies
throughout the world began establishing digital forensics teams,
shifting the agenda of digital forensics from individuals and small
teams to governments and professional organizations.

It was during this time that tools such as EnCase and Forensic Toolkit
(better known as FTK) evolved to become dedicated products that remain a
mainstay of the digital forensics field today
\cite{pollittHistoryDigitalForensics2010}. At the same time,
anti-forensics began to grow in popularity; numerous tools and resources
were developed with the express goal to exploit and hinder the digital
forensics process, generally with the stated motivation of guarding
users' privacy and protecting users from punishment for undesirable
computer activity
\cite{geigerEvaluatingCommercialCounterForensic2005,harrisArrivingAntiforensicsConsensus2006}.

By this time, it had become abundantly clear that there needed to be
dedicated training to develop specialists in digital forensics.
Undergraduate and graduate programs dedicated to the study of digital
forensics were gradually developed
\cite{andersonComparativeStudyTeaching2006}, along with numerous
efforts to standardize and improve digital forensics curricula
\cite{cooperStandardsDigitalForensics2010,nanceDigitalForensicsDefining2009,nanceDigitalForensicsDefining2010}.
This became particularly important due to the growing importance of
digital forensics from a legal perspective; analysts must follow a
strict procedure that ensures the admissibility of digital evidence into
court \cite{conklinComputerForensics2022}. Simultaneously, analysts
must have the experience needed to provide an unbiased, accurate opinion
of digital evidence in court as an expert witness.

Digital forensics continues to be an important focus in industry -- and
in turn, education and research. Broadly, the Bureau of Labor Statistics
projects that information security employment will grow 32\% over the
next decade from 2022 to 2032, adding over 50,000 new jobs
\cite{bureauoflaborstatisticsu.s.departmentoflaborInformationSecurityAnalysts2023}.
The diversity and depth of digital forensics continues to grow with the
broad variety of devices and software involved in modern computing.

With the growing importance and complexity of digital forensics, there
is a clear need for high-quality, realistic data for researchers and
instructors alike. Yet, there continues to be significant gaps in both
the quantity and variety of material suitable for digital forensic
training. The primary barrier to availability has been the privacy and
legal concerns associated with releasing real-world data
\cite{garfinkelForensicCorporaChallenge2007}. From a research
perspective, the result is that many researchers develop their own
datasets, often with a very narrow scope and a very low degree of
reproducibility
\cite{garfinkelBringingScienceDigital2009,grajedaAvailabilityDatasetsDigital2017}.
From an education perspective, the result is that most training material
is either manually created by instructors or reused from existing
sources. That is, researchers and instructors alike often create their
own datasets because publicly available corpora is insufficient;
however, this is a time-consuming process that responds slowly to
changes in technology and software.

Various efforts have been made to automate or streamline the process of
creating new forensic datasets from high-level descriptions and
predefined forensic artifacts. These forensic synthesis frameworks (also
referred to as ``synthesizers'' throughout this paper) include a variety
of features that are geared towards research and education. This
includes the rapid creation of datasets for a large classroom, the
generation of metadata useful in tool validation, and the ability to
export a variety of forensic artifacts from the synthesizer. In
particular, these frameworks enable instructors to create images that
allow students to learn about a specific forensic technique while
emulating some of the real-world challenges that forensic analysts face
in industry.

However, there is still much work to be done towards increasing the
accessibility and flexibility of these frameworks. Before exploring
synthesizers in greater detail, it is necessary to first understand the
purpose and characteristics of forensic datasets in a variety of
contexts. Doing so will not only outline \emph{why} the development of
synthesizers is necessary, but also \emph{what} features these
synthesizers must provide. Once we have established this foundation, we
can begin exploring \emph{how} a synthesizer should be architected --
the focus of the reminder of this thesis.

\section{Purpose of forensic
datasets}\label{purpose-of-forensic-datasets}

REPLACE CODEBLOCK HERE

\subsection{In industry}\label{in-industry}

Before considering the use of forensic datasets in research and
education, we begin by exploring how these datasets are acquired and
used in the ``real world'' -- that is, investigations made by
professionals in industry.

In practice, forensic datasets -- and digital forensics as a whole --
are used for a variety of purposes. In particular, Conklin et
al.~identify three primary cases in which digital forensics may be
performed \cite{conklinComputerForensics2022}:

\begin{itemize}
\tightlist
\item
  \emph{To investigate computer systems related to a violation of law}:
  This includes cases such as the distribution and storage of illegal
  content, the use of a computer to launch denial of service attacks
  against an individual or organization, and the proliferation of
  harmful malware within an organization.
\item
  \emph{To investigate computer systems for compliance (or a violation
  of) an organization's policies}: This primarily covers internal
  investigations in which a user may act well within the laws of their
  jurisdiction, but may have violated a company policy. For example,
  many companies restrict access to computing systems based on the time
  of day as a security measure. Although a user may normally have
  authorization to access the organization's network, an unexpected
  weekend access to the network may require investigation to determine
  if the activity was done with malicious intent.
\item
  \emph{Responding to a legal (or internal) request for digital
  evidence}: This process is known as e-discovery, in which an
  organization preserves and produces digital information typically as
  part of the discovery process in civil lawsuits. With court approval,
  organizations can be compelled to turn over relevant information to a
  particular lawsuit, including digital documents and digital artifacts
  such as file metadata. (The \emph{Federal Rules of Civil Procedure}
  were amended in December 2006 to include ``electronically stored
  information'' as part of civil discovery
  \cite{withersj.ElectronicallyStoredInformation2006}.)
\end{itemize}

Additionally, it is important to note that the acquisition of forensic
images are only one specific part of the overall digital forensics
process, in which analysts must consider the priority of obtaining
evidence and local requirements for ensuring the admissibility of
digital evidence into a court of law. Again, Conklin et al.~identify
several steps throughout the lifespan of a digital forensic
investigation that are summarized here
\cite{conklinComputerForensics2022}:

\begin{itemize}
\tightlist
\item
  \emph{Identification}: While outside of the scope of forensics itself,
  it is important to determine the scope of the devices that need to be
  analyzed as a result of some incident. Necessarily, it is impossible
  to investigate an incident until an organization can ascertain that
  one has occurred (which, in turn, requires implementing detection and
  protection as described in the NIST Cybersecurity Framework
  \cite{nistNISTCybersecurityFramework2023}). Similarly, if the
  scope of an incident is poorly-defined, the subsequent forensic
  investigation may fail to find significant related evidence, which has
  a direct outcome on the success of the investigation (and often, the
  response of the organization as a whole).
\item
  \emph{Preservation}: After identifying the relevant machines, analysts
  must secure and preserve the physical device itself. With guidance
  from the organization and an analyst's judgment, this often involves
  prioritizing the devices that must be imaged first; for example, a
  critical server may be more likely to cycle out important logs first,
  or an employee's device may only hold important information in
  volatile memory.
\item
  \emph{Collection:} At this point, an analyst must now duplicate the
  digital evidence, in addition to any relevant physical evidence. This
  must be done in a way that passes legal scrutiny; that is to say, it
  must meet requirements for accuracy, reliability, and relevance
  \cite{conklinComputerForensics2022,garfinkelBringingScienceDigital2009}
  (in the United States). In the case of disk imaging, this is typically
  done with a write blocker and cryptographic hashing algorithms to
  ensure a faithful copy has been created of nonvolatile memory.
\item
  \emph{Analysis}: Here, an analyst uses tools and their own knowledge
  to identify significant pieces of information within collected images,
  reconstructing data fragments and drawing conclusions to form a
  coherent timeline and scenario. In many cases, this is done using
  tools such as Sleuth Kit, Autopsy, EnCase, and other domain-specific
  software \cite{jonesInsightDigitalForensics2022}, which often
  parse and automatically identify data of interest on a reconstructed
  file system.
\item
  \emph{Reporting}: After an analyst (or a team of analysts) has
  completed their analysis of the collected data, they must summarize
  and provide a non-technical overview of the conclusions drawn from the
  investigation.
\end{itemize}

Throughout this process, analysts must adhere to well-established norms
for ensuring the admissibility of any digital evidence in court. One
notable example is the use of a chain of custody, which details who has
had access to the evidence. This applies to ``conventional'' and digital
forensics alike, as this documentation asserts that the evidence has not
been tampered with as it is transported between analysts and locations.
Without proper documentation such as a chain of custody, critical
evidence may not meet the legal requirements for admissibility, changing
the outcome of a trial.

Generally speaking, a forensic investigation is only part of a larger
incident response effort; for example, if a data breach occurs, a
digital forensic investigation may be used to determine the scope and
methodology of the breach itself. However, other teams within the
organization may be responsible for patching the devices responsible for
the breach, determining legal consequences, and engaging in other
recovery-related activities. Additionally, a forensic analyst may have
other responsibilities outside of this ``chain''; for example, in
addition to developing a written report, they may be called as an expert
witness to testify and defend conclusions drawn from the investigation
\cite{andersonComparativeStudyTeaching2006,conklinComputerForensics2022,cooperStandardsDigitalForensics2010}.

In the context of research and education, most hands-on or practical
training typically focuses on the analysis and reporting steps, rather
than the rest of the investigation process. That is to say, although the
theory provided in training covers this entire process in detail, it is
relatively rare for students to secure data from an organization, use a
write blocker and other hardware needed to image devices, and then
present the conclusions as an expert witness
\cite{cooperStandardsDigitalForensics2010}. While experience here is
important, it is often impractical; not every university will have a
large forensics lab or a courtroom regularly available. Additionally,
the techniques of analysis and reporting are arguably the most
important; while the other steps can be learned ``on the job''
relatively quickly, all students must be familiar with modern software
and tooling, as well as recent advancements in forensic and
anti-forensic techniques. (As a more concrete example: deep technical
knowledge of operating systems is not required to acquire disk images,
but it is certainly required for effective analysis and reporting.)

It is for this reason that the vast majority of education and research
focuses primarily on the analysis step. Improvements to the distribution
and creation of training material -- such as the use of online labs and
synthesizers -- are largely driven by the fact that no physical lab is
strictly necessary, focusing instead on the software and skillset needed
to effectively analyze images and draw conclusions
\cite{bruecknerAutomatedComputerForensics2008,lawrenceFrameworkDesignWebbased2009}.

\subsection{In research}\label{in-research}

In many cases, the focus of research in digital forensics is on
improving specific processes in the analytic step of a forensic
investigation. This includes the development of analysis techniques for
niche platforms, direct improvements to existing techniques, or novel
methodologies for performing forensic investigations for a particular
platform. For example, recent publications from the Digital Forensic
Research Conference detail a new hashing technique for detecting
similarities in arbitrary files \cite{changFbHashNewSimilarity2019},
an analysis of the NAND memory of the Nintendo 3DS
\cite{pessolanoForensicAnalysisNintendo2019}, and the development of
a new tool for the automated analysis of Android mobile devices
\cite{linAutomatedForensicAnalysis2018}. Necessarily, each of these
publications require forensic datasets; in most cases, these are
obtained or developed manually by the authors, as opposed to using an
existing public or private dataset.

Besides novel contributions to the field, other research focuses on
upholding the quality of the investigation process as new technologies
and tools to analyze datasets are developed. For example, NIST maintains
the \emph{Computer Forensics Tool Testing} (CFTT) program, which
provides a common methodology and test corpora for evaluating specific
tool capabilities
\cite{nationalinstituteofstandardsandtechnologyComputerForensicsTool2017}.
Specific capabilities tested include the ability of a tool to perform
string searching, disk imaging, and the recovery of deleted files. The
project also includes catalogs for forensic algorithms, software, and
tools, as well as the \emph{Computer Forensic Reference Data Sets}
(CFReDS) project, which is a repository of forensic datasets contributed
by NIST and other organizations that is often used by both instructors
and researchers
\cite{nationalinstituteofstandardsandtechnologyCFReDSPortal}.

\subsection{In education}\label{in-education}

REPLACE CODEBLOCK HERE

Finally, we address the generation of forensic datasets as it applies to
education. Datasets in an educational setting typically focus on
covering a range of techniques and tools, allowing students to practice
applying theoretical concepts learned through lectures and recitations
\cite{adelsteinAutomaticallyCreatingRealistic2005}. Often, this is
done in the context of a specific scenario, such as a user stealing
files from a protected company server, using steganography to hide
information, or recovering a file from volatile memory. Besides the the
specific technical skills covered by these scenarios, the overall goal
of these images is to develop the analytical skills needed to adapt to
new technologies. as tools and the technologies they analyze continue to
evolve \cite{cooperStandardsDigitalForensics2010}. In other words,
students should be familiar with common tools and patterns in digital
forensics, providing a foundation on which more niche techniques can be
learned \cite{lawrenceFrameworkDesignWebbased2009}.

Although the focus of individual forensic images is often to improve the
skills of students in the analysis phase, these images have a direct
impact on the reporting phase as well. Indeed, students must be able to
accurately summarize their conclusions, using their judgement to
describe a scenario and identify topics of interest in a manner that is
consistent with law. Although important, digital forensics is not just
the effective use of tools and techniques; a student should also aim to
be well-rounded in the legal, social, and professional aspects of
digital forensics \cite{andersonComparativeStudyTeaching2006}.

The challenge, however, is providing hands-on labs that comprehensively
support the ideas learned in theoretical courses. Indeed, instructors
face numerous challenges when providing realistic lab material,
including limited access to the necessary software and hardware; the
extensive time needed to develop, distribute and grade labs; and the
high variability between different forensics programs
\cite{adelsteinAutomaticallyCreatingRealistic2005,guptaDigitalForensicsLab2022,lawrenceFrameworkDesignWebbased2009}.
Much of this difficulty in providing high-quality images can be traced
to the issues associated with acquiring images for educational purposes,
as described in the following section.

\section{Real and synthetic
datasets}\label{real-and-synthetic-datasets}

Now that we have identified the purpose of forensic datasets and why
they are needed, we now move to a discussion of how these datasets are
acquired, as well as various issues encountered when using these
datasets. This section focuses on the qualities of real and synthetic
datasets, including some examples. A more detailed survey of existing
datasets is presented in \autoref{existing-forensic-corpora}.

Broadly speaking, there are two types of forensic datasets summarized by
Park \cite{parkTREDEVMPOPCultivating2018}, based on the original
taxonomy described by Garfinkel et al.
\cite{garfinkelBringingScienceDigital2009}. The first is ``real''
data, which is data that was organically created by human beings without
the explicit intent of being used in a forensic investigation. The other
is ``synthetic'' data, which is data generated for specific forensic
purposes, including research and education. Synthetic datasets are
considerably more common than real datasets in education for the reasons
discussed in this section, though their manual creation still poses a
significant problem.

It should also be noted that this section provides a general overview of
forensic datasets as it relates to education and research, providing the
necessary context for forensic synthesizers. However, a more exhaustive
description of dataset construction and usage has been performed by
Horsman and Lyle in \cite{horsmanDatasetConstructionChallenges2021}.

REPLACE CODEBLOCK HERE

\subsection{Real datasets}\label{real-datasets}

Real data is inherently the most ``realistic'' form of forensic data,
containing extensive background noise as a result of typical computer
usage in addition to a broad variety of software, operating system
artifacts, and other files that might be of interest in a real forensic
investigation. These are most reflective of the scenarios that industry
professionals face, and allow students to train themselves in separating
relevant content from irrelevant content while identifying and
synthesizing details from both a system and a human perspective.

In general, real data can be sourced from far more than just the hard
drives of computers. Other potential and previously used data sources
include:

\begin{itemize}
\tightlist
\item
  social media (which often contains a variety of artifacts with
  revealing metadata and open source information)
  \cite{baggiliDataSourcesAdvancing2015};
\item
  packet sniffers and dedicated forensic tools for IoT devices
  \cite{meffertForensicStateAcquisition2017};
\item
  video game consoles
  \cite{grajedaAvailabilityDatasetsDigital2017,pessolanoForensicAnalysisNintendo2019};
\item
  cloud web server logs \cite{rahmanNewWebForensic2020};
\item
  honeypots \cite{mochForensicImageGenerator2009};
\item
  and the Apache and Python mailing archives
  \cite{grajedaAvailabilityDatasetsDigital2017}.
\end{itemize}

Many more public forensic repositories are described by Grajeda et al.,
though just as many datasets used in research remain private
\cite{grajedaAvailabilityDatasetsDigital2017}. Within reason, with
the billions of internet-connected devices today, there should be no
shortage of sources for real datasets. However, real datasets are the
most challenging to work with in education for a variety of reasons,
particularly the legal and privacy concerns surrounding the use and
distribution of these datasets, as well as their broad scope and lack of
prior analysis.

Necessarily, real datasets were not created to educate students about a
particular technique, and may not adequately supplement an instructor's
material without significant effort. For example, Garfinkel identifies
several requirements for forensic datasets to be suitable for a broad
variety of uses in research and industry
\cite{garfinkelForensicCorporaChallenge2007}. In particular,
Garfinkel notes that in addition to the lack of variety in publicly
available forensic corpora, many real datasets suffer from a lack of
complexity, supplemental annotations, or ongoing maintenance - all of
which are often needed in an educational context.

With respect to availability, some of these datasets are inherently
publicly available; in other cases, their access is restricted to within
an organization. However, just because the underlying data is public
does not necessarily mean that it exists in an aggregate form
immediately suitable for research. Even when aggregated, questions arise
from the use of public datasets for unintended purposes, such as the use
of social media as a source of forensic data; these questions are
already the focus of debate in generative AI, which use publicly
available data to train models for both commercial and research use
\cite{avrahamiOwnershipCreativityGenerative2021,eshraghianHumanOwnershipArtificial2020,rooseAIgeneratedPictureWon2022}.

That said, the primary barriers to the use of real forensic datasets are
privacy and legal issues. For example, between 1998 and 2006, Garfinkel
acquired over 1,000 hard drives through secondary markets, allowing
researchers to perform a range of studies on a large set of real-world
data \cite{garfinkelForensicCorporaChallenge2007}. Although legal
for use by private institutions, the same dataset was barred from use at
the Naval Postgraduate School due to concerns with federal privacy
legislation. Similarly, Grajeda et al.~noted that nearly half of all
digital forensic literature reviewed using a novel dataset did not
publish the dataset due to legal restrictions or NDAs. This was
typically because the datasets were obtained from government agencies,
corporations, and law enforcement agencies, and therefore could not be
publicly released \cite{grajedaAvailabilityDatasetsDigital2017}.
Another reason is that the dataset may contain objectionable or illicit
material, such as licensed software or pornography, preventing its
public distribution.

Certain organizations and researchers have made efforts to make real
datasets accessible to the public through various means. For example,
the emails seized during the Federal Energy Regulatory Commission's
investigation of Enron were purchased by the Massachusetts Institute of
Technology, which anonymized emails and attachments before distributing
the dataset to the public
\cite{yannikosDataCorporaDigital2014,garfinkelForensicCorporaChallenge2007}.
In other cases, institutions have aimed to ensure the data can be made
publicly available ahead of time, such as by requesting that individuals
sign an agreement before any data is collected.

Indeed, some educational institutions do use real-world datasets for
forensic labs. Besides the sources mentioned above, students may also
opt to image their own devices or the device of a friend with
permission; these approaches are mentioned in older works
\cite{andersonComparativeStudyTeaching2006,mochForensicImageGenerator2009},
prior to the advent of online platforms such as CFReDS that provided
easier access to education-focused datasets. Naturally, this method of
acquiring real-world images suffers from other issues as well; students
know exactly what they will find on their own computer, and although
individuals may consent to the use of their images for educational
purposes, there remains the risk of highly personal data being leaked
\cite{garfinkelBringingScienceDigital2009}. Simultaneously, they
often lack the ground truth or annotations needed to assert that a
student has found everything of interest.

\subsection{Overview of synthetic
datasets}\label{overview-of-synthetic-datasets}

Because of the various issues associated with real-world data, many
instructors opt to use synthetic data (or ``manually created data'')
instead, in which the scenario and data are artificially generated based
on some predetermined procedure. That is to say, the data exists with
the explicit intent of being used in research or education.

From an educational perspective, an ideal dataset should accurately
reflect the problems and challenges faced by industry professionals
while avoiding the problems of real-world datasets. In turn, synthetic
datasets are often created with the express intent of allowing students
to explore forensic techniques in realistic scenarios while avoiding the
privacy and legal restrictions of genuine real-world scenarios. (The
same largely holds true in research, as well.)

Again, as described by Park and Garfinkel et al., synthetic data can be
categorized into two distinct groups
\cite{garfinkelBringingScienceDigital2009,parkTREDEVMPOPCultivating2018}:

\begin{itemize}
\tightlist
\item
  \emph{Synthetic test data}, which refers to forensic corpora that have
  been developed to test specific features in a group of tools. These
  are well-annotated datasets with extensive reference information and
  ground truth data, and are typically used to assert that a tool is
  able to analyze and identify data of interest. One example discussed
  so far is the Computer Forensics Tool Testing program administered by
  NIST
  \cite{nationalinstituteofstandardsandtechnologyComputerForensicsTool2017}.
\item
  \emph{Synthetic realistic data}, which is designed to mimic a
  situation that a forensic examiner might encounter in a real-world
  investigation. These are typically much more applicable to an
  educational context than test data, though test data can be used as
  educational material for learning new tools or very specific
  techniques.
\end{itemize}

Synthetic realistic data varies greatly in scope. Simple realistic data
might form a realistic emulation of how a user might use a particular
piece of software or anti-forensic technique, which creates forensic
artifacts directly associated with these actions. In contrast, realistic
data could form a full simulation of a scenario, whether based on real
events (such as the one developed by Moch and Freiling based on the Arno
Funke blackmail case in Germany
\cite{mochForensicImageGenerator2009}) or on common industry themes,
such as the NIST CFReDS data leakage case
\cite{nationalinstituteofstandardsandtechnologyCFReDSDataLeakage}.

REPLACE CODEBLOCK HERE

REPLACE CODEBLOCK HERE

\subsection{Motivation for synthetic
datasets}\label{motivation-for-synthetic-datasets}

Besides the issues mentioned in \autoref{overview-of-synthetic-datasets}, one major motivation for the development of synthetic
datasets is the need to fill various gaps in publicly available corpora.
The survey done by Grajeda et al.~found that some researchers created
new datasets not because of legal or privacy hurdles, but simply because
there was no available dataset for their needs
\cite{grajedaAvailabilityDatasetsDigital2017}. The researchers
interviewed in the survey also recognized the value of publishing their
datasets, but were hindered by a lack of available resources to maintain
or distribute the datasets. Other researchers were prevented from doing
so due to a non-disclosure agreement. The survey concluded that there
was no preference for building new datasets from scratch in research,
though there was clear agreement that private datasets made it more
difficult for researchers to reproduce results; in general, publishing
datasets contributed to the advancement of the field.

Similar sentiment also exists in an educational context, though the
motivation for creating new images from scratch differs from that of
research. For example, the direct reuse of existing datasets is often
undesirable, as it is often the case that answer keys and walkthroughs
have already been published online
\cite{woodsCreatingRealisticCorpora2011}; additionally, students in
the same class can simply replicate the exact methodology of other
classmates to come to the same conclusions without needing to perform
meaningful analysis. Even in labs where students' actions can be
monitored to provide insight into their methodology and possible
mistakes, such as in the \emph{CYDEST} platform developed by ATC-NY
\cite{bruecknerAutomatedComputerForensics2008}, it is still possible
to exactly copy the methodology of another student if the labs
themselves do not differ.

Additionally, many datasets that may be reflective of real-world
scenarios are often unsuitable for academic purposes. For example, the
forensic challenges of the Honeynet Project, DFRWS, and DC3 are suitable
training material for experienced forensic analysts, but are often too
difficult for students to solve
\cite{woodsCreatingRealisticCorpora2011}. As mentioned before, many
real-world datasets lack ``ground truth'' or annotated information that
is suitable for determining what the contents of an image are,
preventing the development of an answer key and qualitative grading of
students' work based on what they find and report. The same is true of
certain synthetic datasets, which may have been developed without a
focus on detailed documentation.

Clearly, there is a need for images that allow students to explore the
techniques and tools used by forensic analysts in industry in a
realistic setting. Preferably, these images should be built in such a
way that they challenge students to explore the same technique, but have
distinguishable differences to dissuade cheating (for example, different
files of interest may be placed, the specific disk sector written to may
vary, or relevant metadata may be changed). Simultaneously, instructors
need to know what the contents of the image are (including these
variations), so that they can judge the accuracy of students' findings
and determine which images are suitable hands-on material to complement
lecture material.

As a result, many instructors ultimately turn to manually developing
realistic forensic scenarios that students can analyze, much like
researchers. Some of these efforts have lead to publicly available
datasets that encompass realistic scenarios, such as the datasets and
scenario published by Woods et al.~as a direct response to the issues
noted above \cite{woodsCreatingRealisticCorpora2011}.

REPLACE CODEBLOCK HERE

\subsection{Challenges in developing synthetic
datasets}\label{challenges-in-developing-synthetic-datasets}

However, fulfilling all of the ideal characteristics of a forensic
dataset is difficult, especially if an instructor wants to create
variations of the same forensic scenario, multiplied by the number of
scenarios used throughout the course. There are three distinct issues
associated with the manual development of forensic datasets: the
significant amount of time involved in their creation, the inherent
non-determinism of human creation, and the lack of realistic background
usage.

The primary issue associated with manual development is the extensive
amount of work and time that must be put into not only planning and
developing the scenario, but also executing it. Virtually all prior
works recognize the time-consuming process involved in the manual
development of disk images suitable for education
\cite{adelsteinAutomaticallyCreatingRealistic2005,gobelForTraceHolisticForensic2022,guptaDigitalForensicsLab2022,mochForensicImageGenerator2009,russellForensicImageDescription2012,scanlonEviPlantEfficientDigital2017,woodsCreatingRealisticCorpora2011},
motivating research into possible solutions to streamlining their
development.

Depending on the artifacts involved in the scenario, it can also be
difficult to ``falsify'' metadata as a means of speeding development up.
More precisely, it can be challenging to produce the artifacts
associated with a scenario over a shorter period of time than is
suggested by the metadata attached to the artifacts. For example, it is
possible to directly change the time in virtualization software (the
approach often used by synthesizers
\cite{gobelForTraceHolisticForensic2022,mochForensicImageGenerator2009})
to allow time to pass without leaving any artifacts that suggest that
the system time has been tampered with. However, making this consistent
with online services can be challenging, since artifacts such as emails
or cached external websites would be based on ``real'' time; additional
work would then be needed to synchronize online content outside of the
control of the instructor. While there are workarounds that could be
considered -- such as saving the contents of various external websites
over a period of time and then ``replaying'' them at the time of
scenario development -- these are not trivial.

Finally, the long development period comes with the challenge of
addressing mistakes. In the event a mistake is made relative to the
desired scenario, instructors have several options. They may be able to
simply log the deviation and move on, or correct it after the fact by
directly editing the virtual hard drive after image creation. This was
the approach taken by Woods et al.~in which at least one researcher
logged into their personal email account while developing a fictional
scenario; the images were later scanned for personal identifiers and
stripped as needed \cite{woodsCreatingRealisticCorpora2011}.
However, consider a scenario in which it is absolutely essential that
certain files are deleted and written in a particular order, perhaps to
ensure the operating system behaves in a specific manner. In this case -
and possibly many others - the only option may be to start the process
of developing the image from the beginning.

The nondeterminism associated with human creation is a double-edged
sword. Given the same general scenario, this allows multiple images to
differ very slightly in nature (a desirable feature as described earlier
in this section); as described by Woods et al., the researchers all
acted out events in a pre-defined timeline, but did not do so at the
exact same time or in the exact same way
\cite{woodsCreatingRealisticCorpora2011}. This provides the
variability needed to prevent students from directly copying answers,
but allows students to explore the same techniques at a high level.

Simultaneously, however, this also adds a degree of uncertainty within
the images. Various artifacts or results on the images may be the result
of some confounding factor in the procedure; for example, suppose that
two researchers follow the same procedure to delete and overwrite a file
with the goal of making an image that allows students to explore file
carving in slack space. Due to variations in operating systems, drivers,
and other related software, it may be the case that the recoverable
contents in the slack space differ significantly between the two
researchers. Similarly, consider a case in which two researchers follow
the same procedure to develop volatile memory captures on a Linux
machine. It might be the case that the paging daemon evicts relevant
data for one image but not the other, causing significant differences.
Without careful preparation in creating a consistent environment for
development and testing, these differences might arise without a clear
reason. Where possible, sources of uncertainty should be minimized or at
least known to the scenario developer.

Finally, an inherent limitation of the long development time needed to
produce these images is the difficulty in populating realistic
background noise, in which a user performs normal activities on the
computer that are completely unrelated to the meaningful components of
the scenario. This must be done in a way that is realistic to the
scenario while avoiding any activity that could personally identify the
researcher or instructor. Although some of this background noise can be
automated - such as by scripting the process of browsing to websites or
sending emails - the fact remains that many realistic ``background''
activities, such as working in software such as Microsoft Office,
MATLAB, and SolidWorks, are not easy to perform without a human.
Furthermore, some of this background noise may comprise a significant
portion of the day, such as if these activities comprise a user's day
job in the scenario. Naturally, it is not the case that most instructors
or researchers have the time available to spend several hours each day
generating this information, much less over multiple variations of
multiple scenarios.

\section{Research objectives}\label{research-objectives}

At this point, we have clearly established a need for a more streamlined
method of developing scenarios for research and education, while still
providing the variability and content needed for images to be
interesting and adequately train students in forensic methodologies.
Additionally, these scenarios need to provide ground truth data to
determine the artifacts contained in each image, allowing instructors to
identify what should be contained in students' reports of the scenario.
Some research has been done into developing image synthesizers, which
aim to automate part or all of this work, as described in \autoref{chapter-two}.

This thesis aims to directly improve upon the foundations provided by
prior synthesizers, with the explicit goal of achieving feature parity
with all existing synthesizers. In particular, this thesis will describe
the components and architecture necessary to build an effective forensic
synthesizer for research and education with the following elements:

\begin{itemize}
\tightlist
\item
  The ability to create and export disk images, network captures, and
  volatile memory captures through arbitrary means;
\item
  the ability to generate forensic artifacts in a modular manner, both
  with and without operating system virtualization;
\item
  the ability to log every action in a standardized format, the
  Cyber-investigation Analysis Standard Expression (CASE)
  \cite{caseyAdvancingCoordinatedCyberinvestigations2017}, to
  address existing standardization concerns
  \cite{horsmanDatasetConstructionChallenges2021};
\item
  and the use of modern Python libraries, standards, and practices,
  promoting future development by reducing the overall complexity of the
  framework and abstracting specific concepts where possible.
\end{itemize}

Furthermore, this thesis aims to leverage recent advancements in
generative AI to streamline the development of full forensic scenarios,
as well as individual forensic artifacts to be added to a larger
forensic dataset. Finally, this thesis will evaluate the viability of
this framework in an actual classroom setting.

\section{Contribution}\label{contribution}

This thesis contributes the \emph{automated kinetic framework}, or AKF,
a modernized synthesizer framework that aims to provide the foundation
of a larger forensic dataset ecosystem. This framework can be used to
not only vastly reduce the time spent developing new datasets for
research and education, but also improve the discoverability of both
AKF-generated and non-AKF-generated datasets. Additionally, by focusing
on the long-term viability of AKF through its modular architecture, the
hope is that educators and researchers will have greater variety in the
datasets available to them, even as new developments and advancements in
technology occur.

\chapter{Literature review}\label{chapter-two}

We begin with a literature review of two topics: the publicly available
forensic datasets that currently exist and the contributions of prior
synthesizers. In particular, we describe the datasets that currently
exist for digital forensics in greater detail, as well as gaps
identified in these datasets by various authors. This will provide
context for both the need of synthesizers, as well as the specific gaps
that these synthesizers have gradually filled.

\section{Existing forensic
corpora}\label{existing-forensic-corpora}

Forensic datasets have been available for public use (or by request)
since the early days of the digital forensics field, though their
sourcing and qualities have changed greatly over time. This is explored
in far greater detail by Grajeda et al., who performed a survey of over
700 research articles to identify the various datasets used by the
digital forensics field over time. However, it is still important to
recognize certain datasets identified by prior literature as it is
relevant to synthesizers.

Early datasets were largely derived from real sources, whether made
available to the public or otherwise. The earliest collections of real
datasets include the used hard drives collected by Garfinkel from 1998
to 2006 and the Enron email corpus obtained during the federal
investigation of Enron \cite{garfinkelForensicCorporaChallenge2007}.
Other early real datasets identified by Grajeda et al.~include the
public Apache mailing archive, the Reuters news corpora, and various
facial recognition collections such as the MORPH corpus, all of which
were made in the early to mid-2000s
\cite{yannikosDataCorporaDigital2014,grajedaAvailabilityDatasetsDigital2017}.

There were also a variety of synthetic datasets constructed during this
early period. This includes the network captures obtained from simulated
attacks conducted by the MIT Lincoln Laboratory from 1998 to 2000
\cite{garfinkelForensicCorporaChallenge2007}, as well as standalone
datasets used for tool validation developed as part of the early CFTT
program developed by NIST. Notably, some synthetic datasets during this
period were generated as part of challenges, such as those produced for
DFRWS conferences, which are noted as being difficult for students to
solve \cite{woodsCreatingRealisticCorpora2011}.

The variety of forensic datasets increased considerably towards the late
2000s, which can be credited to both the overall growth of the field
(including the broader field of incident response) and computing as a
whole. Various notable sources described by Grajeda et al.~include
malware samples discovered ``in the wild'', large natural language
collections from multiple languages, and file-specific datasets such as
collections of Microsoft Office files. It was also during this time that
non-disk datasets began to be more prevalent, such as volatile memory
dumps and network captures. Although not explored by this thesis in
great detail, mobile datasets -- such as smartphone disk images, mobile
malware and applications, and SIM card images -- grew in prevalence as
well.

Many of these datasets were not maintained as part of a larger
collection of datasets with the explicit intent of providing them for
digital forensics research. This began to change towards the late 2000s;
Garfinkel's collection would eventually evolve into the Real Data
Corpus, growing to 30 terabytes by 2013
\cite{garfinkelBringingScienceDigital2009a,yannikosDataCorporaDigital2014}.
The collection then included hard disk images, flash drive images, and a
variety of optical discs sourced from real-world usage, requiring
institutional review board approval to use. This collection would
eventually be part of the Digital Corpora platform, which includes a set
of purely synthetic datasets. Separately, NIST began developing the CFTT
and CFReDS projects, both of which provide forensic datasets for a
variety of purposes. Digital Corpora, CFTT, and CFReDS are actively
maintained as of writing.

There are other datasets that are rather unique in nature, and are
maintained as part of a larger niche collection. Collections of malware
samples have grown significantly in size, in part because of the modern
threat intelligence ecosystem with platforms such as VirusTotal. There
exists datasets focusing on the dark web, such as those operating on the
Tor network, including ``black market'' sites on which illegal goods are
bought and sold. Finally, there exist network captures from various
novel sources, including various iterations of the Collegiate Cyber
Defense Competition and various network captures from university IT
departments \cite{grajedaAvailabilityDatasetsDigital2017}.

These datasets span a wide range of technologies -- in particular,
versions of the same technology -- that require different methodologies
to effectively analyze. In particular, there is the challenge of making
datasets available that reflect current advancements in technology, such
as new operating system versions or new applications. For example,
instant messaging applications have changed considerably over the
history of the field, ranging from MSN Messenger in the early 2000s to
Skype, Discord, Telegram, Signal, Slack, and more. Artifacts from each
of these applications must be handled differently, even if the
underlying service provided is largely the same. Similarly, the
strategies for analyzing Windows artifacts have changed significantly
from version to version, as new registry keys become relevant in
analyzing applications while others become unused.

It is this gradual ``aging'' of datasets, in which their relevance
degrades over time, that contributes to the continuous need for new
datasets. Furthermore, there have been a broad variety of datasets that
have proved to be relevant in digital forensics research, even if not
typical or immediately evident. Indeed, this is one of the reasons
identified by Grajeda et al.~for the manual development of new datasets
by research authors; it would often be the case that a modern dataset
simply did not exist for their needs. The other motivation is that the
dataset used was never made public, either due to the author not having
the resources to distribute the dataset themselves, or because there
were legal or privacy concerns.

How are these datasets relevant to the development of synthesizers?
Clearly, there has been a broad variety of datasets that have proved to
be relevant in digital forensics research, even if not immediately
evident. It should be the goal, then, that any effort to streamline the
develop of forensic datasets be able to cover as many use cases as
possible. To do so, a synthesizer should aim to achieve two separate
goals:

\begin{itemize}
\tightlist
\item
  A synthesizer should be able to replicate the features of existing
  datasets, provided that the underlying technologies are still
  available; and
\item
  A synthesizer should be able to account for developments in operating
  systems, applications, or other technologies, without requiring
  significant changes to the underlying architecture.
\end{itemize}

Indeed, the synthesizers described in the following section have
explicitly addressed these two concerns. For example, many of these
synthesizers have focused on implementing features that reflect the
qualities of real-world datasets. This includes the ability to execute
malware samples in a virtualized environment, send emails to arbitrary
email servers, and insert data on removable drives -- all of which
generate forensic artifacts that have been used as key datasets in
forensics-related research as shown here. Similarly, modern synthesizers
are capable of larger generating disk images, network captures, and
volatile memory dumps (in addition to extracting specific artifacts,
such as application-specific files), which are also reflective of the
focus of existing datasets. The gradual growth in the ability of
synthesizers to generate various artifacts can be seen in the specific
contributions made by each synthesizer in \autoref{analysis-of-existing-synthesizers}.

Of note is the generation of similar datasets that typically involve
significant human interaction, such as public email distribution lists,
photographs of human faces, and other transcripts of conversations.
While this has not been explored in significant detail by prior
synthesizers, it is explored as part of the generative AI work done as
part of AKF in \autoref{chapter-six}.

The second issue, in which synthesizers must be extensible in such a way
that they can support new applications, has been approached in several
ways. This is described in greater detail in \autoref{chapter-three}, but has been a major design consideration in the
development of most synthesizers.

\section{Analysis of existing
synthesizers}\label{analysis-of-existing-synthesizers}

REPLACE CODEBLOCK HERE

Numerous frameworks have been built over the last two decades that aim
to significantly reduce the effort involved in creating synthetic images
from scratch by automating various application- and OS-specific actions
at specified locations and times. The functionality and availability of
the frameworks described in literature have varied considerably over
time, but the goal of these frameworks has remained largely consistent:
they all aim to provide a rapid method for instructors to develop
forensic labs for students. Note that low-level studies are described in
later, more relevant sections throughout this thesis for clarity, rather
than providing detailed analyses as part of this chapter. Before
analyzing and adapting the low-level implementations of prior
synthesizers, it is best to first contextualize their broader
contributions.

The first identified effort to streamline the creation of forensic labs
through a high-level language was published by Adelstein et al.~in 2005
through the development of \emph{FALCON, the Framework of Laboratory
Exercises Conducted Over Networks}
\cite{adelsteinAutomaticallyCreatingRealistic2005}. It proposed an
architecture for not only the automatic creation of lab exercises, but
also their deployment to a virtual lab and the evaluation of students'
actions compared to the ground truth. Very few details of the
implementation are provided, though some examples of its usage are
provided. Although not primarily a synthesizer, a related work was in
the development of \emph{CYDEST, the CYber DEfenSe Trainer}
\cite{bruecknerAutomatedComputerForensics2008}, which similarly
provided complex virtualized lab environments to students via the
internet. Again, limited implementation details are provided, and it is
unclear if either \emph{FALCON} or \emph{CYDEST} are publicly available
or actively maintained.

\emph{Forensig2}, described by Moch and Freiling in 2009 and revisited
in 2012, appears to be the first detailed description of an image
synthesizer
\cite{mochForensicImageGenerator2009,mochEvaluatingForensicImage2012}.
Operators define scenarios through a Python 2 library provided by the
authors that abstracts various virtual machine operations (such as the
formatting of disks, the creation of partitions, and the copying of
files from the host to the virtual machine). Actions are performed live
through a Qemu-based VM, with the result being a ground truth report and
(effectively) an image to be analyzed by students. The source code for
\emph{Forensig2} is not currently maintained and appears to be
unavailable.

The \emph{Digital Forensic Evaluation Test (D-FET)} platform described
by William et al.~in 2011 provides a custom scripting language for users
to define system- and user-level actions
\cite{williamCloudbasedDigitalForensics2011}. Similar to
\emph{FALCON} and \emph{CYDEST}, it provided virtualized labs through
VMware ESXi. Unlike most other frameworks, it was primarily built to
provide a platform to efficiently evaluate digital forensic tools on a
scalable infrastructure, rather than streamline the process of building
and distributing labs for instructors (though it is mentioned that the
infrastructure was also used to host student labs). It is unclear if a
public implementation is available.

Russell et al.~define an XML specification, the \emph{Summarized
Forensic XML (SFX)} language, to describe scenarios as a sequence of
various actions on various partitions of a disk
\cite{russellForensicImageDescription2012}. This can be passed into
an interpreter to produce an image that can be analyzed by students.
Besides high-level operations such as disk partitioning and file
copying, it also provides Windows- and Linux-specific routines for
minimizing partition sizes with the goal of reducing the ``effective''
size of irrelevant files associated with the OS. It also briefly
describes approaches for updating web browsers and the Windows registry
to reflect certain actions. Limited implementation details are provided,
and it is unclear if a public implementation is available.

Yannikos et al.~define a framework for describing scenarios as a series
of Markov chains, allowing users to graphically define scenarios through
a graph view in which nodes and relationships can be easily edited
\cite{yannikosDataCorporaDigital2014}. Its design makes it
well-suited for quickly developing variations of the same scenario from
a large dataset of available artifacts to be placed on the image, though
it is not clear what application- or OS-specific routines are provided.

The remaining frameworks are all largely similar to each other in that
they are all built in Python, providing users with a Python library to
define and generate scenarios. Various functions in the library provide
abstractions to complex actions, such as Facebook browser activity or
sending emails on the Thunderbird application. Additionally, each of
these produce some form of detailed output that can be used as answer
keys or ground truth in an instructional setting. A brief summary of the
notable aspects of each of these frameworks relative to each other is
provided below:

\begin{itemize}
\tightlist
\item
  \emph{ForGeOSI} \cite{maxfraggMaxfraggForGeOSI2023} introduced the
  use of the VirtualBox SDK to automate various operations, forming the
  basis for much of the work done as part of VMPOP.
\item
  \emph{ForGe} \cite{vistiAutomaticCreationComputer2015} is
  specifically designed to generate NTFS and FAT32 images with a
  particular focus on placing data by maintaining and serializing custom
  data structures for supported filesystems.
\item
  \emph{EviPlant} \cite{scanlonEviPlantEfficientDigital2017}
  encompasses a novel method for the distribution of generated images in
  an educational context, whereby ``evidence packages'' are distributed
  to students to apply to a base image file. Then, an OS-native
  injection tool plants relevant artifacts according to the evidence
  package. Since the base image must only be downloaded once, each
  scenario is significantly smaller than if distributed as standalone
  images.
\item
  \emph{VMPOP} \cite{parkTREDEVMPOPCultivating2018} provides an
  architecture and routines for relatively elaborate VirtualBox control
  (such as attaching USB devices and starting video captures) in
  addition to a variety of OS-specific commands. Provided Windows
  routines include the ability to create restore points, install
  programs, map network drives, set registry values, and more. Although
  the architecture as a whole is platform-independent, the provided
  implementations operate with VirtualBox and Windows.
\item
  \emph{hystck} \cite{gobelNovelApproachGenerating2020} provides
  routines for automating OS- and application-specific commands through
  YAML configuration files (passed through a generator) and/or Python
  scripts (for low-level control). Similar to \emph{EviPlant}, it uses
  ``differential'' images to allow lightweight scenarios to be
  distributed relative to ``template'' images, and produces both network
  and disk captures.
\item
  \emph{ForTrace} \cite{gobelForTraceHolisticForensic2022} is the
  most recently developed synthesizer, which directly builds upon
  \emph{hystck} by providing volatile memory captures (alongside disk
  and network captures) in addition to various other new features (such
  as the ability to execute PowerShell scripts to create Windows
  artifacts), with a focus on a modular architecture. A variant focusing
  on Android artifact generation was developed in 2024
  \cite{demmelDataSynthesisGoing2024}.
\end{itemize}

Clearly, a considerable amount of work has been done in exploring ways
to streamline the process of developing images, although the
availability and functionality of each framework varies greatly.
Notably, with the exception of \emph{ForTrace}, none of these works are
direct extensions of past works, implying that at a fundamental level,
it was necessary for frameworks to be developed from scratch to support
the framework authors' needs. This is not directly stated in any of
these works with the exception of \emph{hystck}
\cite{gobelNovelApproachGenerating2020}, though it can be reasonably
concluded that the lack of maturity, availability, and maintenance of
prior works was a contributing factor to the independent development of
the other frameworks. This will be addressed in more detail in
\autoref{chapter-three}.

\chapter{Architecture and design}\label{chapter-three}

\section{Motivation}\label{motivation}

As described previously in \autoref{analysis-of-existing-synthesizers}, with the notable exception of
\textbf{ForTrace} and its related synthesizers, no synthesizer been an
extension of another synthesizer. This raises the question -- why
reinvent the wheel by developing yet another distinct architecture for
this thesis? This subchapter briefly explores the deficiencies in
existing synthesizers and explains why these are issues that warrant
building a new architecture from scratch, rather than extending an
existing synthesizer.

The motivations for developing completely new codebases instead of
extending existing synthesizers has varied considerably over the years.
One reason is that several synthesizers are not open-source and
therefore cannot easily be extended, as is the case with
\textbf{Forensig2} and \textbf{TraceGen}. Another reason is that the
focus of certain synthesizers resulted in a codebase that is simply
incompatible with the goals of newer works. For example,
\textbf{ForGe}'s architecture focuses largely on direct filesystem
manipulation to generate forensic artifacts, and is therefore not
suitable for a synthesizer that requires a virtualized operating system.
Synthesizers such as \textbf{VMPOP}, which exclusively leverages
agentless artifact generation as described in \autoref{agentless-artifact-generation}, require significant
architectural changes to support agent-based artifact generation.

However, perhaps the largest motivation for constructing new
synthesizers is simply the lack of ongoing support for virtually all
synthesizers. It appears that no synthesizer has gained significant
traction within the broader forensic community, possibly with the
exception of \textbf{ForTrace}; for the synthesizers that \emph{are}
open source, none of them are under active development and maintanance.
Additionally, the forensic datasets generated by these synthesizers have
not seen significant adoption in either education or research; many
instructors continue to use the human-generated datasets available on
public platforms.

The inflexibility of prior synthesizers, combined with the overall lack
of support and success of synthesizer-based datasets, contributes to the
disparate codebases that are now observed today. However, this is not to
say that the individual contributions of each prior synthesizer cannot
be merged into a single project that resolves many of the architectural
barriers that have reduced the adoption and extension of existing
synthesizers.

In turn, AKF is built on the following four pillars to help promote its
long-term usage. In particular, it allows it to generate datasets that
address several of the considerations raised by Horsman and Lyle in the
construction of various datasets, such as the need for comprehensive
documentation, an awareness of the ``realism'' of the resulting dataset,
and transparency in the scenario development process
\cite{horsmanDatasetConstructionChallenges2021}.

First, AKF conforms to modern Python development practices. This
includes the use of modern project management practices (such as the use
of \emph{uv} and \texttt{pyproject.toml}, rather than the use of
\texttt{setup.py} observed in older synthesizers), as well as static
linters (\emph{flake8}) and style enforcers (\emph{black}, \emph{isort})
to promote adherence to the PEP8 standard. Additionally, AKF's libraries
make heavy use of Python 3.11+ features, such as type hinting and
special type annotations, which allows for tools such as \emph{mypy} to
perform static type checking. In addition to greatly improving the
development experience, these practices also increase the likelihood of
discovering errors earlier in the development process.

Second, AKF takes a modular, agent-based approach to implementing
application-specific functionality. This allows AKF to use existing
automation frameworks for web browsers and other applications, greatly
simplifying the codebase when compared to the same features implemented
in other synthesizers. This focus on flexibility makes it significantly
easier to install the agent, implement new application-specific
features, and more - a particularly important design focus given AKF's
dependence on agents for the majority of its application-specific
functionality.

Third, AKF is architected to maintain feature parity with all prior
synthesizers. That is, although AKF deviates considerably from the
implementation details of prior synthesizers, this does not come at the
loss of prior advancements in the field. In particular, AKF supports all
three artifact generation techniques described in \autoref{chapter-four} using hypervisor-agnostic interfaces, reducing the tight
coupling that made certain features difficult to implement in prior
synthesizers. This reduces the likelihood that a new feature or
technique will require a significant architectural change to support it.

Finally, AKF is designed with the explicit intent of developing an
ecosystem of AKF-generated datasets, promoting long-term usage. This is
reflected in the design of AKF's logging and reporting mechanisms as
described in \autoref{chapter-five}; the use of an
RDF-based standard, CASE, allows for arbitrarily complex queries to be
made against AKF datasets. These queries can be made in bulk,
significantly improving the ability of researchers to find and use
datasets that may be relevant to them.

These four pillars reflect throughout the design of AKF's architecture,
which is described in the following section and the next three chapters
of this thesis.

\section{Overview}\label{overview}

At a minimum, every synthesizer must fulfill these high-level
requirements through some means, largely derived from the criteria
developed by Horsman and Lyle
\cite{horsmanDatasetConstructionChallenges2021}:

\begin{itemize}
\tightlist
\item
  The synthesizer must accept commands on how it should operate. These
  commands should serve as a form of self-documentation, in which it is
  clear to another person \emph{what} the expected contents of the image
  are, as well as the intent behind these commands.
\item
  The synthesizer must be able to accept external inputs, such as files
  and other binary data, to be included in any final outputs.
\item
  The synthesizer must implement the mechanisms and technologies
  necessary to carry out the commands it has been provided -- that is,
  it must be able to generate artifacts. Such mechanisms should be
  independently verifiable, and preferably leave as little extraneous
  information as possible.
\item
  The synthesizer must be able to generate a final output (such as a
  disk image), along with ground truth and reporting that describes the
  expected contents of that image. This should include a well-structured
  description of the dataset, a unique identifier for the dataset, and
  an explicit identification of any data of ``evidential value'' where
  possible.
\end{itemize}

These four requirements are fulfilled by various mechanisms throughout
AKF. A complete architecture diagram is shown in the figure below.

!\textbf{Architecture 2025-02-07 17.07.24.excalidraw}

This can be simplified to the following high-level diagram, which
broadly groups AKF into a set of seven modules:

!\textbf{Architecture 2025-02-08 16.35.46.excalidraw}

At a high level, this architecture can be broken up into three distinct
concepts, each of which covers a distinct chapter.

The first set of modules are responsible for artifact generation. This
encompasses three major systems - a hypervisor and its associated SDK,
an OS-specific agent, and \texttt{akflib}. \texttt{akflib} is a Python
library containing the abstract interfaces and concrete implementations
necessary to generate individual artifacts and complete datasets. This
includes routines for directly interacting with hypervisors, issuing
commands to virtual machines, and directly modifying virtual hard
drives. This library is also the foundation for OS-specific agents,
which carry out actions on the virtual machine on behalf of the host.
This is described in greater detail in \autoref{chapter-four}, and corresponds to the \emph{action automation library},
the \emph{OS-specific agent}, and the \emph{virtual machine} (as well as
the \emph{user applications} running on the machine).

The second set of modules are responsible for logging and reporting.
This encompasses both independent libraries for generating outputs and
ground truth, as well as the various logging-related mechanisms that are
contained throughout the artifact generation libraries. These modules
are responsible for exporting and documenting artifacts generated by
AKF; in particular, it makes heavy use of CASE, a standardized ontology
for documenting the contents of forensic datasets
\cite{caseyAdvancingCoordinatedCyberinvestigations2017}. This is
covered in \autoref{chapter-five}, and corresponds to the
\emph{output and validation library} and any associated components
within the \emph{action automation library}.

The final set of modules are responsible for invoking AKF itself and
supporting scenario development. AKF is an \emph{imperative}
synthesizer, which means that commands are written and executed using an
imperative language (here, Python 3) that dictates exactly \emph{how}
scenarios should be constructed. However, AKF also supports a
\emph{declarative} syntax, which allows users specifies \emph{what}
forensic artifacts and datasets are generated without the need to learn
the AKF libraries and write Python code. Additionally, AKF contains
generative AI tools for constructing scenarios in the declarative syntax
as well as individual artifacts. This is described in \autoref{chapter-six}, and covers the \emph{scenario construction library}
and the \emph{translation unit}, as well as any scripts that leverage
AKF libraries.

The following three chapters will focus on these module groups. In
simpler terms, this thesis addresses the following questions in order:

\begin{itemize}
\tightlist
\item
  How do we automate or streamline the generation of artifacts?
\item
  How do we document and report on the artifacts and datasets that are
  generated?
\item
  Given the solutions that address these two challenges, how do we
  actually use them to build scenarios?
\end{itemize}

\chapter{Action automation}\label{chapter-four}

This chapter addresses the modules responsible for automating artifact
generation, depicted in the partial architectural diagram below.

!\textbf{39.4 - Action automation 2025-02-07 17.15.30.excalidraw}

These implement AKF's ability to generate individual artifacts as part
of a larger scenario dataset. This chapter begins with an overview of
the techniques used by prior synthesizers for generating artifacts. It
then moves to a low-level analysis of the actual implementation used by
AKF, while comparing these techniques to other synthesizers.

In particular, this chapter addresses the role of the action automation
library in generating artifacts by interacting with a live virtual
machine and disk images stored on the host. It describes how logical
artifacts are generated at a lower level, both directly through
hypervisor APIs and through an OS-specific agent installed on the
virtual machine. It concludes by describing the generation of physical
artifacts through direct filesystem and disk image editing.

\section{Overview}\label{overview}

At the core of every synthesizer is the ability to place or otherwise
generate forensic artifacts. Each of the synthesizers described in
\autoref{analysis-of-existing-synthesizers} takes one of three approaches to artifact generation, as
partly described by Scanlon et al.
\cite{scanlonEviPlantEfficientDigital2017}:

\begin{itemize}
\tightlist
\item
  \textbf{Physical}: No virtualization of software or hardware ever
  occurs; data is written directly to the target medium, such as a disk
  image or virtual hard drive.\\
\item
  \textbf{Agentless logical}: The synthesizer interacts with a live VM
  to generate artifacts. Interaction is achieved without the need for
  custom software to be installed on the VM, and is typically achieved
  through the hypervisor itself or a remote management tool native to
  the virtualized operating system.\\
\item
  \textbf{Agent-based logical:} The synthesizer interacts with a
  dedicated client (agent) on a live VM to carry out actions. The VM
  must have the agent installed before any interaction can occur.
\end{itemize}

These three approaches are not mutually exclusive within a single
synthesizer, though it is the case that most prior synthesizers have
used exactly one approach to generate artifacts. The table below denotes
the approaches used by each of the synthesizers previously discussed.
Where source code is not available, a best effort is made to identify
the approach used by a particular synthesizer based on its published
paper, if one exists; otherwise, the entire row contains question marks.

\begin{longtable}[]{@{}
  >{\raggedright\arraybackslash}p{(\linewidth - 6\tabcolsep) * \real{0.1799}}
  >{\raggedright\arraybackslash}p{(\linewidth - 6\tabcolsep) * \real{0.4604}}
  >{\raggedright\arraybackslash}p{(\linewidth - 6\tabcolsep) * \real{0.1942}}
  >{\raggedright\arraybackslash}p{(\linewidth - 6\tabcolsep) * \real{0.1655}}@{}}
\toprule\noalign{}
\begin{minipage}[b]{\linewidth}\raggedright
Synthesizer
\end{minipage} & \begin{minipage}[b]{\linewidth}\raggedright
Physical
\end{minipage} & \begin{minipage}[b]{\linewidth}\raggedright
Agentless
\end{minipage} & \begin{minipage}[b]{\linewidth}\raggedright
Agent-based
\end{minipage} \\
\midrule\noalign{}
\endhead
\bottomrule\noalign{}
\endlastfoot
\textbf{FALCON}, 2005 & ? & ? & ? \\
\textbf{CYDEST}, 2008 & ? & ? & ? \\
\textbf{Forensig2}, 2009 & Yes (mounting filesystem, writing directly to
disk using offset) & Yes (over SSH only) & No \\
\textbf{D-FET}, 2011 & Yes (mounting filesystem, writing directly to
disk using offset) & No & No \\
\textbf{SFX}, 2012 & Yes (mounting filesystem) & No & No \\
\textbf{Yannikos et al.}, 2014 & ? & ? & ? \\
\textbf{ForGeOSI}, 2014 & No & Yes (hypervisor interfaces) & No \\
\textbf{ForGe}, 2015 & Yes (emulates complete filesystem in memory as
Python objects) & No & No \\
\textbf{ForGen}, 2016 & No & No & No \\
\textbf{EviPlant}, 2017 & Yes (write directly to image file using
offset) & No & Yes (unknown mechanism) \\
\textbf{VMPOP}, 2018 & No & Yes (hypervisor interfaces) & No \\
\textbf{hystck}, 2020 & No & No & Yes (Python agent) \\
\textbf{TraceGen}, 2021 & No & No & Yes (unknown mechanism) \\
\textbf{ForTrace}, 2022 & No & No & Yes (Python agent) \\
\end{longtable}

There are advantages and disadvantages to each approach, in addition to
requiring distinct implementation techniques for each. The remainder of
this chapter analyzes these each of these three approaches in greater
detail, describing the implementation details of prior synthesizers and
comparing them to those of AKF's.

\section{Agentless artifact
generation}\label{agentless-artifact-generation}

\subsection{Hypervisor interfaces}\label{hypervisor-interfaces}

\textbf{Agentless artifact creation} describes one of two general
techniques. The first is emulating ``normal'' human interaction by
leveraging hypervisor interfaces -- such as the monitor, keyboard, and
mouse -- to manipulate a GUI-based operating system directly. This is
the approach taken by \textbf{VMPOP} and \textbf{ForGeOSI}; for example,
VMPOP leverages the VirtualBox API to interact with supported
applications. More precisely, it typically uses a sequence of keyboard
strokes to focus and interact with UI elements, such as clearing User
Account Control dialogs on Windows or starting applications with Win+R.

Strictly speaking, this approach most accurately reflects how a real
human would interact with a machine. In many cases, this greatly reduces
the \textbf{Synthesis pollution} that occurs. However, this tends to
lead to verbose scripts that are only capable of performing very
specific actions. The keyboard and mouse actions required to fulfill a
particular action can change significantly between versions of the same
application, versions of the same operating system, and varying screen
sizes.

For example, VMPOP handles UAC prompts on machines prior to Windows 7 by
sending an Alt+Tab keystroke to the machine, clicking the mouse at the
center of the screen to focus the UAC prompt, and then sending Alt+C to
accept the prompt. On machines running Windows 7 or later, this is
instead achieved by focusing the UAC prompt with the mouse and then
sending Alt+Y to accept the prompt.

Similarly, VMPOP interacts with browsers exclusively through the use of
keyboard shortcuts. VMPOP supports simple interactions with a small
number of websites, such as logging into Microsoft or Google, but is
dependent on the form elements remaining the same over time. That is,
VMPOP does not inspect the contents of the current webpage to perform
actions, and cannot react to design changes. If there are new focusable
elements on the page, the same keystroke sequence may no longer achieve
the desired effect. This lack of runtime logic, which amounts to
operating the machine with the monitor off, leads to brittle scripts
that can be tedious to fix.

It is worth noting that throughout this thesis, the use of
hypervisor-specific guest software, such as VirtualBox Guest Additions
and VMWare Tools, is treated as an agentless approach throughout this
thesis. Although this inherently requires the installation of
``unusual'' software on the virtual machine, it is sufficiently distinct
from typical user software that it is unlikely to generate artifacts
that complicate a scenario. In most cases, artifacts generated by
hypervisor guest software can be noted, isolated, and ignored.

\textbf{TraceGen} notes that the ideal future is to use some combination
of computer vision and AI to generalize user actions. Currently,
completing the action of ``performing a Google Search in Microsoft
Edge'' would likely be achieved through a sequence of predefined
keystrokes. While this is explored in greater detail in \autoref{chapter-eight}, recent advancements in LLMs may make it possible to allow
a machine to perform arbitrarily complex tasks on a GUI-based operating
system using natural language - an approach that can be integrated into
AKF.

\subsection{Management utilities}\label{management-utilities}

The alternative is to use existing management utilities, typically a
shell, which are native to the virtualized operating system and are
capable of carrying out commands. This can be further broken down into
two categories: local management utilities, such as Bash and PowerShell,
and remote management utilities, such as SSH and WinRM.

Local management utilities typically refer to scripting languages that
are available as part of the operating system, and can be used to manage
most or all operating system resources. For example, PowerShell allows
users to modify registry keys, invoke applications, create users, and
more. Similarly, any standard Linux terminal program, such as Bash or
Zsh, can be used to install packages and run a variety of command-line
applications. This is typically invoked by either opening and focusing a
terminal window (such as the Win+R shortcut on Windows), or by directly
executing scripts through the hypervisor guest additions.

In particular, \textbf{VMPOP} makes heavy use of local management
utilities; it implements much of its functionality through a collection
of PowerShell and batch scripts. For example, VMPOP allows users to
focus a window by PID, process name, or window title. This is achieved
by getting a PowerShell handle to the process using
\texttt{Get-Process}, using \texttt{Add-Type} to add a local C\#
function that is capable of sending keyboard events through
\texttt{user32.dll}, and then holding the Alt key while using
\texttt{AppActivate} to focus the window and bring it to the foreground.
VMPOP leverages similar scripts for launching and terminating processes
by name, uninstalling programs, creating a Windows restore point, and
more.

In Similarly, remote management utilities allow a remote device to
invoke local management utilities, typically over an SSH server or
WinRM. In most architectures, the use of remote management utilities
requires that the host and guest machines can communicate with each
other, which is typically achieved through a NAT or host-only interface
managed by the hypervisor. This approach is taken by \textbf{Forensig2},
which exclusively connects to a running SSH server to carry out user
actions.

The use of remote management utilities to automate actions typically
undertaken by users is not uncommon, especially among general
infrastructure as code solutions. For example, the open-source Ansible
framework simplifies the configuration of Windows and Linux devices to
simple, YAML-based files called ``playbooks''. These playbooks typically
contain a sequence of high-level tasks to perform, such as installing
packages, managing local user accounts, running scripts, and more.

It is worth noting that although this approach does not require
installing new software on the machine, most operating systems perform
some degree of logging when their management utilities are invoked. For
example, the \texttt{sshd} daemon logs connections regardless of which
network interface is used, which may make it difficult to separate or
remove logs not related to SSH connections as part of the scenario
itself. Similarly, the invocation of PowerShell scripts causes event
logs to be generated, which can be particularly noisy if Script Block
Logging (which logs the execution and content of all PowerShell scripts)
is active.

\subsection{AKF agentless
generation}\label{akf-agentless-generation}

AKF allows users to perform agentless artifact creation through a
dedicated hypervisor-agnostic interface (`akflib.\textless something

(What functionality is available through the hypervisor interface, as
well as the VirtualBox API in particular?)

(What agentless functionality in \textbf{VMPOP} do we instead delegate
to the agent, such as interacting with browsers? why is this a good
thing, and what do we lose as a result?)

\begin{itemize}
\tightlist
\item[$\square$]
  \#todo explain the hypervisorbase
\end{itemize}

\section{Agent-based artifact
generation}\label{agent-based-artifact-generation}

\textbf{Agent-based artifact creation} involves the use of a dedicated
program on the VM that serves as an interface between the host machine
and the guest machine. This program runs commands natively on the
virtual machine on behalf of the host machine, typically accepting
commands over a dedicated network interface. This allows for greater
flexibility and more complex actions to be taken. In particular, it
allows for existing automation frameworks such as Playwright and
PyAutoGUI to be used in implementing application-specific functionality.
However, this approach often leads to \textbf{Synthesis
pollution\textbar synthesis pollution}; besides the presence of software
that would never exist on a typical user's machine, agents often do not
interact with applications the same way that a human would. In this
case, the synthesizer should document known pollution that can be
ignored for educational and testing purposes.

\subsection{Analysis of the ForTrace
agent}\label{analysis-of-the-fortrace-agent}

Agent-based artifact creation is the approach taken by
\textbf{hystck}/\textbf{ForTrace}, which refers to its agent as an
``interaction manager''. Because ForTrace provides the largest set of
functionality for its agents, and is by far the most mature synthesizer,
AKF's agents borrow heavily from ForTrace's approach. As a result, a
detailed analysis of ForTrace's design is provided here.

ForTrace agents are simply an entire copy of the ForTrace codebase
copied over to the virtual machine -- that is, the ``agent'' and
``server'' share the same codebase, but have different entry-points and
call different functions. Application-specific functionality for agents
is organized into individual modules (files), where each file separates
a set of commands or actions for a single application.

The server, or ``virtual machine monitor'' (VMM), issues commands over
TCP to a running instance of the agent on the VM. Invoking commands
entails issuing a simple string to the agent running on the VM. For
example, invoking a command to add a new user to the VM may look like
the following:

REPLACE CODEBLOCK HERE

The agent's main loop, which receives and interprets these commands, is
extremely simple:

REPLACE CODEBLOCK HERE

When the \texttt{Agent} is first instantiated, it forms a permanent TCP
socket out to the configured port and IP address where it expects the
server (the VMM) to be issuing commands.
\texttt{Agent.receiveCommands()} receives every message available by
reading the socket, parsing it, and then invoking
\texttt{Agent.do\_command()}, which converts the command message into a
specific Python function call with arguments.

Each module under \texttt{fortrace.application} can be thought of as a
coherent group of commands associated with a particular \emph{real}
application (like Firefox or Thunderbird). Recall earlier how we found
our \texttt{userManagement} application and invoked \texttt{addUser}:

REPLACE CODEBLOCK HERE

At a high level, the \texttt{application()} call attempts to import
\texttt{fortrace.application.\{application\_name\}}. Although not
enforced by a higher-level interface, each of these modules contains
subclasses of the following four classes (defined in
\texttt{fortrace.application.application}) at minimum, with possibly
more helper classes for OS-specific functionality or other modularity as
needed:

\begin{itemize}
\tightlist
\item
  \texttt{ApplicationVmmSide}: Contains one function for each command
  implemented in \texttt{ApplicationGuestSide}, building a message that
  will be interpreted and acted upon by the corresponding
  \texttt{ApplicationGuestSideCommands} class.
\item
  \texttt{ApplicationVmmSideCommands}: Accepts and interprets
  module-specific messages returned by the agent, which may be used to
  update the remote state as tracked by the host.
\item
  \texttt{ApplicationGuestSide}: Implements the actual
  application-specific functionality for the agent, providing one
  function for each separated command by this module.
\item
  \texttt{ApplicationGuestSideCommands}: Interprets commands and
  arguments, calling the respective function in the corresponding
  \texttt{ApplicationGuestSide} subclass. This allows the actual
  dispatch of commands to be delegated to this class, which is free to
  choose how actions are performed (threading, multiprocessing, etc.) as
  well as any module-wide state it may need to maintain.
\end{itemize}

This naming convention is intentional. As will be shown later, the word
\texttt{Application} is replaced with the name of the module in
camelcase. For example, \texttt{fortrace.application.userManagement}
contains \texttt{UserManagementVmmSide},
\texttt{UserManagementVmmSideCommands}, and so on. Discovering and
getting handles to these classes is performed through string
manipulation of the relevant application's module name, as shown in the
example below:

REPLACE CODEBLOCK HERE

In the example above, \texttt{UserManagementGuestSide.addUser()}
contains the literal code to execute on the guest when
\texttt{addUser()} is called, such as adding the registry keys needed
for a user to be created. On the other hand,
\texttt{UserManagementVmmSide.addUser()} contains the code to send a
message over TCP that the agent will understand, eventually leading to
the execution of the agent's version of \texttt{addUser()}.

More precisely, the \texttt{ApplicationVmmSide} subclass effectively
serves as the API for calling the associated functions in the
\texttt{ApplicationGuestSide} class running on the virtual machine. This
subclass, along with the complete agent-side code, is stored in a single
file. For example, the API and code for opening a Firefox browser window
is as follows:

REPLACE CODEBLOCK HERE

Calling the \texttt{open()} command from the host causes a message of
the following form to be sent to the agent:

REPLACE CODEBLOCK HERE

Upon receiving this message, the agent's main loop will search for the
\texttt{webBrowserFirefox} module and import its corresponding
\texttt{ApplicationGuestSideCommands} subclass. After any state
management, the subclass will then search for a function called
\texttt{open} in its corresponding \texttt{ApplicationGuestSide} class,
as implemented below:

REPLACE CODEBLOCK HERE

There are two major implementation details that should be observed here.
As mentioned previously, commands are sent through a simple string-based
protocol. This simplicity makes it easy to debug issues as a result of
the protocol, but is relatively inflexible. In particular, it is
difficult to send complex objects that cannot easily be serialized to a
string without loss of information, such as Playwright browser objects
with state that is difficult to reconstruct using strings alone.

The second implementation detail involves the discovery and execution of
commands indicated by the protocol. As shown previously, ForTrace makes
heavy use of Python's runtime introspection to discover the correct
modules and functions to call based on the contents of a command string.
While this is a valid approach, it is significantly more complex than
the approach taken by AKF.

\subsection{The AKF agent}\label{the-akf-agent}

AKF borrows heavily from the Python agent-based approach of
\textbf{ForTrace}, but improves on its architecture in several respects.

Perhaps the most significant change is the use of RPyC for
communication, a library for symmetric remote procedure calls
\cite{TomerfilibaorgRpyc2025}. Although the RPyC protocol is
symmetric, it is often used in client-server architectures to allow
clients to manipulate remote Python objects as if they were local
objects, as well as invoke remote functions using local parameters. By
delegating the serialization and deserialization of complex objects to
RPyC, this allows us to perform complex operations that would have been
difficult to implement with the simple string-based protocol of
ForTrace.

In ``new-style'' RPyC, this is achieved by running a \emph{service} on
the device where remote operations should be performed; this service
simply listens on an ephemeral TCP port. Services expose a set of
functions and attributes that may be accessed remotely by an RPyC
client. This is analogous to the \texttt{ApplicationGuestSideCommands}
and \texttt{ApplicationGuestSide} classes of individual ForTrace
modules.

Because the exposed functions (as well as their signatures) and
attributes cannot be inferred from the raw RPyC connection alone,
clients can implement a typed, concrete API that allows type-hints and
autocompletion features to continue working in development environments.
This simultaneously abstracts the raw RPyC call (and the existence of an
RPyC connection) away from the user, while also providing the remote
function signatures where there would otherwise be none. This is
analogous to the \texttt{ApplicationVmmSide} and
\texttt{ApplicationVmmSideCommands} classes of ForTrace modules.

For example, a simple RPyC service and its corresponding API may be
implemented as follows:

REPLACE CODEBLOCK HERE

This syntax provides three major improvements over the ForTrace
protocol. First, the routing of functions is wholly delegated to RPyC.
Instead of manually constructing a message with the function name and
its associated parameters (as strings) over the network, implementing
support for a particular service on the client-side API is as simple as
calling an untyped function \texttt{rpyc\_conn.root.set\_browser()}. By
wrapping it around a typed function,
\texttt{ChromiumServiceAPI.set\_browser()}, users regain the ability to
use code completion and static linting tools.

Second, this allows us to pass and return arbitrarily complex objects
for which we do not have to write the serialization and deserialization
logic for. When passing complex objects from the agent to the server or
vice versa, a reference to the object is sent over the network and
wrapped by a \emph{proxy object}, which behaves like the original object
\cite{TheoryOperationRPyC}. Importantly, it is not necessary to
distinguish between local and remote/proxy objects of the same type when
writing code, which elements the extra complexity of using proxies.

Finally, the ability to interact with complex remote objects allows us
to significantly reduce the actual code that is written as part of the
API exposed to the host. For example, there is no need to implement a
wrapper for every method available as part of a Playwright page object;
instead, a reference to the Playwright object \emph{running on the
virtual machine} can be given to the host machine. Instead of writing
individual methods for opening pages, navigating to specific elements,
and so on, we can simply use the methods that already exist in the
Playwright object -- any local calls on the host's proxy object will
lead to remote outcomes on the host, as desired.

Application specific-functionality is broken up into individual RPyC
``subservices'', which are created on demand. The agent's main loop is
itself a ``root'' RPyC service that is responsible for creating and
destroying these subservices upon request; all subservices are known to
the root service at program start, eliminating the need to perform
runtime introspection to find application-specific modules.

Additionally, because the service and the API to an RPyC service are
separable, this allows us to break the API (which is used in scripts)
and the agent logic itself into two separate libraries. This has two
advantages -- it may slightly reduce the size of the agent when
installed onto the virtual machine, and it also makes it significantly
easier to build and generate standalone executables using tools like
PyInstaller. Unlike ForTrace, whose agent installation process requires
a batch script installing various libraries and Python through the
Chocolatey package manager, AKF's agent requires only that a single
executable is copied over and configured to run on startup (in addition
to setting relevant firewall rules).

The list of applications supported by AKF's agent, as well as
implementation-specific details, is described in the table below.

\begin{itemize}
\tightlist
\item[$\square$]
  \#todo shoot for feature parity with ForTrace; see page 11 of its
  paper for what it supports
\end{itemize}

\begin{longtable}[]{@{}
  >{\raggedright\arraybackslash}p{(\linewidth - 4\tabcolsep) * \real{0.0376}}
  >{\raggedright\arraybackslash}p{(\linewidth - 4\tabcolsep) * \real{0.2019}}
  >{\raggedright\arraybackslash}p{(\linewidth - 4\tabcolsep) * \real{0.7606}}@{}}
\toprule\noalign{}
\begin{minipage}[b]{\linewidth}\raggedright
Module
\end{minipage} & \begin{minipage}[b]{\linewidth}\raggedright
Dependencies
\end{minipage} & \begin{minipage}[b]{\linewidth}\raggedright
Details
\end{minipage} \\
\midrule\noalign{}
\endhead
\bottomrule\noalign{}
\endlastfoot
Chromium & Playwright \cite{MicrosoftPlaywrightpython2025} & Allows
arbitrary webpages to be visited on Chrome and Edge, as well as perform
complex actions such as completing forms and clicking links based on
HTML selectors \\
\end{longtable}

\section{Physical artifact
generation}\label{physical-artifact-generation}

\subsection{Overview}\label{overview-1}

\textbf{Physical artifact creation} encompasses any technique in which
virtualization of an operating system is not required to generate
artifacts. This gives the synthesizer the ability to bypass operating
system or driver behavior, which may lead to non-deterministic behavior
for certain actions. For example, a scenario developer may want to
guarantee that a particular deleted file is partially overwritten by
another file, allowing that the deleted file is recoverable from the
slack space of the newly placed file.

However, there is no known method for reliably reusing the same logical
block from userspace. Operating systems do not expose low-level
filesystem functionality to applications; furthermore, they are still
subject to hardware drivers that regularly rearrange physical space,
such as those that engage in wear leveling. While it may be possible to
predict the outcome of these wear leveling techniques as demonstrated by
Neyaz et al., this is far from the determinism that may be necessary of
research and tool validation \cite{neyazForensicAnalysisWear2018}.
Additionally, background programs and services may perform file
operations that are difficult to predict.

In turn, it is sometimes necessary to bypass the operating system
entirely to reliably place data on a disk. There are two primary
techniques used to achieve this -- mounting the filesystem and
performing direct read/write operations, and bypassing the filesystem
entirely and performing direct writes to the underlying disk image.

Filesystem mounting is leveraged by several synthesizers, including
\textbf{Forensig2} and \textbf{SFX}. For both of these synthesizers,
physical artifact creation is achieved by simply allowing the user to
specify a file to copy from the host machine to a specific filepath on
the disk image. Other synthesizers, such as \textbf{ForGe} and
\textbf{EviPlant}, instead write to the disk image directly.
\textbf{ForGe} maintains its own representation of supported
filesystems, using its own data structures to represent a FAT32/NTFS
filesystem. This allows it to quickly identify and place data in
unallocated or slack space. In contrast, synthesizers such as
\textbf{EviPlant} simply allow the user to provide an offset into the
disk image where arbitrary data will be written to.

It is important to note that the majority of the artifacts generated by
physical techniques can still be created non-deterministically through
logical means. Users can create files of interest, delete them, and then
write new files to disk through a virtual machine; this \emph{may}
result in the desired outcome, but there is no strict guarantee that
particular files have overwritten data in now-unallocated space. In some
cases, this may be acceptable, such as a scenario in which an instructor
simply wants to demonstrate that file recovery from slack is possible.

It is also worth noting that artifact creation does not lend itself well
to generating network captures and volatile memory, since no operating
system is virtualized and therefore no applications (that would normally
perform requests over the internet) are executed. However, it is
theoretically possible to build forensic datasets through simulation
alone; if a user knows exactly what artifacts are generated through the
execution of an application and \emph{how} these artifacts are
generated, it may be possible to artificially generate these artifacts
as if the application had actually been run.

An example may be editing the history database of a particular browser
contained on a filesystem by mounting it, parsing it, and then adding
new entries; this would make it appear as if the machine was used to
browse to these websites without requiring virtualization. However, this
technique is difficult to implement in practice. For example, a scenario
developer would need to ensure that the artifact is consistent with
other artifacts on disk, since typical Windows application usage may
generate relevant artifacts in prefetch files and jump lists.

\subsection*{4.4.2 - AKF implementation}\label{akf-implementation}
\addcontentsline{toc}{subsection}{4.4.2 - AKF implementation}

\chapter{Output and validation}\label{chapter-five}

This chapter addresses the mechanisms through which AKF generates and
documents its outputs, depicted in the partial architectural diagram
below.

!\textbf{39.5 - Output and validation 2025-02-08 17.13.11.excalidraw}

This chapter addresses the role of the output and validation library in
providing several services to AKF, including a centralized logging
system, the use of CASE objects, and invoking various commands to
generate and export outputs such as disk images, network captures, and
volatile memory dumps. It also describes high-level reporting and
validation functionalities, both those as part of the framework itself
and options available through external tools.

\section{Overview}\label{overview}

Whether generating artifacts through physical or logical means, these
artifacts must ultimately be exported and documented. In most cases,
this involves generating disk images, volatile memory captures, or
network captures; additionally, specific artifacts, such as browser
artifacts, may be selectively copied from a filesystem.

In either case, the contents and details of these artifacts, as well as
the means through which they were generated, should be documented in
some format. The term ``ground truth'' describes the contents of a
single specific dataset in full. That is, it ideally provides a
reference for every artifact that can be discovered within a dataset, as
well as every action taken to plant those artifacts.

From an educational perspective, the ground truth represents an ``answer
key'' to the dataset; it details every artifact of interest that an
analyst could be expected to discover. For research, it allows for
well-labeled datasets that can be used for tool development, validation,
and testing. Importantly, this should be generated in a manner
independent of the input script used to construct the scenario, allowing
\emph{all} artifacts to be documented, including those not explicitly
declared or deemed important by the scenario creator.

The remainder of this chapter describes the mechanisms through which
outputs are exported from the synthesizer and the mechanisms that allow
for real-time logging and documentation.

\section{Core outputs}\label{core-outputs}

\subsection{Extracting datasets}\label{extracting-datasets}

\subsection{Distribution and
storage}\label{distribution-and-storage}

\section{Metadata and ground
truth}\label{metadata-and-ground-truth}

\subsection{Overview}\label{overview-1}

There exists a gap in the ability of instructors and researchers to
perform bulk searches for specific forensic artifacts in public
datasets. For example, the NIST CFReDS repository
\cite{nationalinstituteofstandardsandtechnologyCFReDSPortal}, one of
the largest listings of forensic datasets, does not have a unified
standard for describing uploaded images. While it is possible for users
to search by keywords and human-applied tags, these are not presented in
a standardized format.

For most datasets, an analyst must read through a PDF answer key (if one
exists) or analyze the image themselves to determine if a particular
artifact is present. Such formats are not immediately machine-readable,
and are therefore difficult to use in bulk searches. Additionally, the
content of human-made reports may be limited to what the author believes
is significant, even if other artifacts of interest are present in the
image. In turn, it may be difficult to quickly determine if a dataset is
useful in demonstrating a particular technique to students, or in
validating a specific feature of a newly-developed tool.

A rigid, well-defined format for ground truth is invaluable to
researchers engaging in tool validation and development. It is easy to
write an automated converter for well-structured data into natural
language; it is likely more difficult to perform the reverse operation.
Perhaps the lack of labeled forensic datasets on major repositories may
be attributable to the lack of a need for one; Grajeda et
al.~demonstrated that few scenarios are shared between researchers to
begin with, so there is rarely a need to label them for general-purpose
usage. AKF has the opportunity to solve this issue by allowing for the
mass production of labeled datasets, adopting a single major standard.

Many forensic analysis tools support exporting case data in both
proprietary and language-agnostic formats. For example, Cellebrite's
UFED supports exporting to UFDR and XML files, Magnet Axiom supports
exporting to XML, and Autopsy supports a variety of formats including
Excel, STIX, and HTML. Each of these vary in structure and format, and
do not necessarily contain equivalent information for the same analyzed
disk image using default settings. This is the primary challenge with
using an existing format, especially a proprietary format that is
subject to vendor changes; certain details may be missing, and may
change at an arbitrary point in time.

There has been extensive work in other fields towards developing a
structured ontology that describes relationships and low-level details.
This includes the EVIDENCE project for criminal justice and the
Structured Threat Information Expression (STIX) format for conveying
cyber threat intelligence
\cite{caseyLeveragingCybOXStandardize2015}. For example, STIX
provides a standard set of objects that allows organizations to describe
observed attacker techniques and associate them with specific pieces of
malware, attack campaigns, or threat actors.

However, there is limited work that aims to document the contents of a
forensic scenario (disk images and related metadata) in a vendor-neutral
manner. Besides their lack of adoption, Casey et al.~found that existing
formats lacked features such as parent-child relationships, user
actions, and non-technical case information such as a chain of custody.
In response, the same authors introduced the Digital Forensic Analysis
eXpression, or DFAX, a language extending CybOX (the predecessor to
STIX) for use in the digital forensics community. DFAX eventually
evolved to become the Cyber-investigation Analysis Standard Expression
(CASE) \cite{caseyAdvancingCoordinatedCyberinvestigations2017},
which we leverage as \emph{fastlabel}'s standard output format. CASE is
perhaps the most comprehensive and actively supported ontology available
for digital forensics; contributors include NIST with support from the
Linux Foundation.

\subsection{CASE and Python
bindings}\label{case-and-python-bindings}

CASE is a vendor-neutral format that aims to document both technical and
non-technical information about a digital forensics case
\cite{caseyAdvancingCoordinatedCyberinvestigations2017}. It aims to
cover as many OS-specific and application-specific artifacts as
possible, while still providing the flexibility to describe artifacts
from uncommon applications. In theory, data exported from any major
vendor, such as Cellebrite, Magnet, or FTK, can be converted into a
valid CASE file. CASE is an extension of the Unified Cyber Ontology, or
UCO, which simply provides basic objects that are not specific to
digital forensics (such as applications or users). Consistent with
existing documentation, CASE and CASE/UCO refer to the same format.

CASE is built on the Resource Description Framework (RDF), a model for
describing information using relationships. Two objects are linked using
relationships, each of which is described as a ``triple''. This pattern
allows for directed, labeled graphs to be expressed using RDF. The
ontology of CASE objects is defined using the Terse RDF Triple Language,
or Turtle, which allows these triples to be written in a simple text
format. For example, the following set of triples describes an object
called \texttt{ApplicationFacet} with two properties,
\texttt{numberOfLaunches} and \texttt{applicationIdentifier}:

REPLACE CODEBLOCK HERE

An instance of a \texttt{Application} object may thus be represented in
the JSON-LD format using the \texttt{ApplicationFacet} as follows
(including attributes omitted above):

REPLACE CODEBLOCK HERE

Because the CASE format itself is language-agnostic, it is necessary to
write a library for each language that leverages CASE. As of writing,
the CASE project provides Python bindings for UCO/CASE 1.4
\cite{CaseworkCASEMappingPython}, in which each unique object is
represented as a Python class, which can be instantiated to produce
individual objects. However, this library has several limitations due to
its design. For example, CASE objects are internally represented as
dictionary of strings, rather than a set of instance variables. While
this makes it easier to serialize these objects to JSON-LD, it also
makes it extremely difficult to work with objects after they have been
instantiated:

REPLACE CODEBLOCK HERE

Additionally, it appears that each of the objects have been manually
translated from the Turtle definitions to their corresponding Python
class. This is slow and time-consuming, especially given the context
that a significant overhaul of UCO/CASE to version 2.0 is underway, with
new object definitions; there appears to be no active effort to update
the 1.3 bindings to 2.0.

AKF leverages (and contributes, as part of a project independent of this
thesis) Pydantic-based bindings for CASE. The foundation for this system
will be the Pydantic library for Python, which allows developers to
quickly define classes (referred to as ``Pydantic models'', or simply
``models'') with built-in schema validation and serialization based on
Python type hints \cite{colvinPydantic2024}. More broadly, it allows
us to simplify the declaration of individual objects while providing
runtime type validation and automatic casting.

For example, we can write the same object above as follows:

REPLACE CODEBLOCK HERE

This declaration is only three lines, 40 lines shorter than the existing
declaration of \texttt{ApplicationFacet} - a 93\% reduction in code
written. This is because the conversion of instance variables to valid
JSON-LD keys is deferred until serialization. This design choice allows
us to centralize the serialization logic in a single parent class that
all CASE objects inherit from. In exchange for slightly increasing the
complexity of converting \texttt{numberOfLaunches} to a dictionary with
the correct key name, we can massively simplify the logic for declaring
CASE objects.

A simple CASE bundle, representing the complete contents of a forensic
scenario, can be observed below:

REPLACE CODEBLOCK HERE

A script is provided with the Pydantic-based CASE bindings to
automatically parse the RDF files and convert them to valid Pydantic
models. It automatically converts XSD datatypes to their native Python
types (or a custom wrapper type if a native type does not exist),
correctly inherits classes, and automatically generates docstrings and
Pydantic fields as applicable. Additionally, the script also
topologically sorts dependencies in the same file; the parent class of
an RDF object may be declared \emph{after} its child class, which is
disallowed in Python. This greatly simplifies the process of maintaining
Python bindings for UCO/CASE, as well as the overall design of the
library for future needs.

\subsection{CASE integration in
AKF}\label{case-integration-in-akf}

With these Python bindings, we can integrate them throughout
artifact-generating libraries in AKF.

(How does one actually generate CASE entries? the answer is that various
things accept CASE bundles, and they can optionally attach extra data to
the bundle as needed\ldots)

\section{Human readable reporting}\label{human-readable-reporting}

as previously mentioned, it's easier to go from rigid and well-defined
to human reports than it is to go the other way around.

\chapter{Building scenarios}\label{chapter-six}

This chapter addresses the modules responsible for allowing users to
invoke the framework, both through a standard Python script and through
a high-level YAML file. It also addresses the generative AI modules that
can assist a user in building a scenario, as depicted in the partial
architectural diagram below.

!\textbf{39.6 - Building scenarios 2025-02-08 17.23.40.excalidraw}

At this point, we have provided the implementations for automating
artifact generation in a near-deterministic manner with comprehensive
logging and reporting. However, there is still the challenge of exposing
this functionality in a user-friendly manner. Furthermore, even if the
\emph{process} of placing artifacts or performing actions can be
simplified, the challenge of deciding what actions to perform still
remains.

It is important to note that it is still largely the responsibility of
instructors to provide the actual background noise to populate the
images with. Although the high-level languages provided by many of these
frameworks make it easy to place files at desired locations or visit
websites that are part of a scenario, these must all be defined and
created ahead of time.

This chapter addresses the challenges of creating background noise and
providing simple APIs for complex GUI-driven applications. More
precisely, we address two questions -- how do we invoke AKF's automation
systems, and how does AKF assist a user in building a scenario? Here, we
explore AKF's imperative and declarative APIs, as well as the viability
of using large language models (LLMs) to assist in building individual
files and complete scenario descriptions.

\section{Background}\label{background}

We begin by analyzing how synthesizers accept user input to decide how
to operate, as well as what specific forms of data are accepted by these
synthesizers. In particular, we address the following questions:

\begin{itemize}
\tightlist
\item
  How do users define the high-level ``scenario'', or sequence of
  operations that the synthesizer should take to form the image?
\item
  How do users pass data, such as images to be placed or emails to be
  sent, into the synthesizer?
\end{itemize}

For many of the frameworks created in the last decade, users typically
define scenarios by using a Python library to interact with the
framework, setting up the virtualized environment and perform high-level
actions on the environment. This abstracts the underlying calls to the
virtualized environment away from the user. This code-based approach
represents an \emph{imperative} strategy to scenario creation, where the
user describes how the image should be created by describing the exact
order and methodology by which actions should be taken.

It is worth noting that, like many automation frameworks such as
Playwright \cite{MicrosoftPlaywrightpython2025}, the language used
to interact with the synthesizer's API does not need to match the
language used to implement the synthesizer itself (although this is
often the case). For example, Playwright itself is implemented in
TypeScript, and therefore began with a Node.JS API. Today, Playwright
provides APIs in Python, Java, and C\#.

In contrast, custom scenario formats provided by \emph{D-FET}
\cite{williamCloudbasedDigitalForensics2011}, \emph{SFX}
\cite{russellForensicImageDescription2012}, and Yannikos
\cite{yannikosDataCorporaDigital2014} follow a \emph{declarative}
strategy. Here, a custom high-level language describes what the final
state of the image should be. For example, consider the following
example \emph{SFX} code taken from
\cite{russellForensicImageDescription2012}, in which a partition is
created on which a 64-bit version of Windows 7 is installed and a user
called ``Gordon'' uses Firefox to browse to the internet:

REPLACE CODEBLOCK HERE

The same might be partially expressed in \emph{ForTrace} as the
following, excluding additional overhead for ground truth generation:

REPLACE CODEBLOCK HERE

Note that \emph{hystck} - and by extension, \emph{ForTrace} - also
allows users to express scenarios as YAML files, such that both a
declarative and an imperative approach to defining scenarios is
available. This reduces the technical difficulty of using the framework,
while still allowing users experienced with both Python and the
framework to perform lower-level customization as needed. This
highlights the fact that both declarative and imperative approaches can
be used simultaneously; in particular, it demonstrates that
declarative-to-imperative translators can be written to support
arbitrary declarative languages, such as those of both SFX and ForTrace.
While not explored in this thesis, it is also worth noting the GUI-based
interfaces provided by \textbf{Yannikos et al.} and \textbf{ForGe} for
building scenarios.

\section{Setup and imperative
usage}\label{setup-and-imperative-usage}

Like many of its predecessors, AKF implements its functionality and
exposes its API in the same language -- Python 3. There are numerous
advantages to a Python-based API; besides the low difficulty of setting
up and using Python, its rich ecosystem allows scenarios to be extended
through the use of other libraries from the Python ecosystem. For
example, if a user wanted to conditionally execute certain parts of a
scenario by testing if a particular service is currently online, a user
could use the \emph{Requests} library \cite{Requests31Documentation}
to issue an HTTP request out-of-band before doing the same using
synthesizer-provided routines in the virtualized environment.

\section{Declarative usage}\label{declarative-usage}

\subsection{Overview}\label{overview}

\subsection{The AKF declarative
syntax}\label{the-akf-declarative-syntax}

\section{Using generative AI for individual
artifacts}\label{using-generative-ai-for-individual-artifacts}

As it currently stands, users of synthesizers must still perform a
significant amount of work towards generating the artifacts to be
planted. While the process of generating an image based on some
predefined scenario has been streamlined through existing synthesizers,
users still need to define all of the data that they want to plant. For
example:

\begin{itemize}
\tightlist
\item
  If a user wants to place 100 photos on the drive to simulate real
  usage, the user needs to pass in 100 realistic images;
\item
  If a user wants to simulate an email or other online conversation, the
  user needs to pass in the entirety of the conversation to simulate;
\item
  If a user wants to generate ``proprietary'' documents to emulate some
  form of corporate sabotage, the user would need to generate a variety
  of Microsoft Office, PDF, or other files in these formats ahead of
  time.
\end{itemize}

The net result is that although creating images for the purposes of tool
validation and research can be accomplished with existing frameworks,
creating realistic images that are more reflective of real-world
scenarios that a forensic analyst might encounter still requires
extensive work. While true that images should often be small enough in a
classroom setting to allow the student to explore a single specific
technique, real-world scenarios encountered by analysts are typically
not limited by time or size. An analyst might have to deal with a drive
used over the course of a decade to store many photographs and send many
messages. Such scenarios are valuable training material for courses that
encapsulate a long period of forensic study, allowing a student to apply
many different techniques in reconstructing a large-scale scenario.

With recent advancements in generative AI, popularized by services such
as Midjourney and ChatGPT, it is now significantly easier to generate
realistic images and text content from short, high-level descriptions.
Additionally, various services exist for creating realistic audio and
video files that emulate a particular person's voice or facial
movements; these can be used to generate additional scenario content of
interest, especially if the scenario is based on a real-world event.

It holds that generative AI can be used to quickly populate forensic
datasets with realistic conversations and images consistent with an
arbitrary scenario. For example, a corporate espionage case could be
built by providing a large language model such as ChatGPT with prompts
to describe complex machinery in both a technical writing and a
conversational style. Simultaneously, similar prompts can be passed into
an image synthesizer such as Midjourney to produce related images. The
images and text produced can then be used to create documents describing
an unreleased product of high value, providing a pipeline through which
significant artifacts can be planted onto a forensic image.

This idea can be extended further by training models on specific
datasets; for example, if an instructor wished to create a fictional
scenario in which a user frequently interacts with users of a particular
online community, a large language model could be trained on available
conversations to provide a degree of realism to the scenario. However,
as mentioned before, this faces the challenges of ownership, privacy,
and legality behind works derived from publicly available information
that was (likely) not published with the expectation of its usage in an
AI model.

It is important to note that the inclusion of generative AI into
synthesizers does not necessarily require deep integration with the
framework itself. Many existing frameworks could be extended to use
documents, images, or other data sourced from generative AI instead of
user-defined files without the need to change the architecture of the
framework. However, as advancements in AI continue, it may make sense to
directly integrate AI-driven actions into synthesizers. For example,
there may come a time in which synthesizers can be provided natural
language prompts (such as ``Open Firefox and browse to news-related
websites'') that directly lead to the generation of relevant artifacts,
without the need to explicitly program the process of browsing to a
website in advance.

\chapter{Evaluation and observations}\label{chapter-seven}

\section{Context and scenario}\label{context-and-scenario}

\section{Student analysis}\label{student-analysis}

\chapter{Future work}\label{chapter-eight}

\section{Integration of recent
advancements}\label{integration-of-recent-advancements}

\begin{itemize}
\tightlist
\item
  So OpenAI has their
  \href{https://openai.com/index/introducing-operator/}{Operator} thing,
  it's important to take note that it exists and could likely be
  generalized to automate user actions more broadly -- however, there's
  three things to keep in mind:

  \begin{itemize}
  \tightlist
  \item
    it likely can't substitute (right now) all of the features provided
    by AKF, such as the ability to manipulate low-level bytes or change
    the BIOS time, even if it had full knowledge of how to interact with
    the operating system
  \item
    Present models are
    \href{https://152334h.github.io/blog/non-determinism-in-gpt-4/}{known
    to be non-deterministic}, which is acceptable for educational
    purposes (so long as you have a framework that can log all of the
    actions it takes, or identify ground truth after the fact), but a
    little less so when building images that \emph{need} to be done
    deterministically, such as when developing images for research or
    tool validation

    \begin{itemize}
    \tightlist
    \item
      not to mention that it apparently
      \href{https://www.reddit.com/r/ChatGPTPro/comments/1i8jln3/i_am_among_the_first_people_to_gain_access_to/}{sucks
      at doing its job right now}
    \end{itemize}
  \item
    Operator is not \emph{incompatible} with our architecture or overall
    framework -- it could likely be built as yet another \texttt{akflib}
    module or agent module, which would give users the option to choose
    between a simple, non-deterministic option for generating browser
    artifacts from human language, or a more complex, deterministic
    option by manually declaring actions.
  \end{itemize}
\end{itemize}

\section{Framework limitations and
extensions}\label{framework-limitations-and-extensions}

AKF was built with the expectation that it would be easy to maintain,
develop, and extend. Indeed, there are several use-cases that AKF does
not fulfill as of writing. This includes: - support for generating
mobile device datasets, although

\chapter{Conclusion}\label{chapter-nine}



% == End thesis content


%glossary & acronym lists
\printglossary[type=\acronymtype]
\printglossary
\bibliography{lloyd_gonzales_thesis}

\appendix

\chapter{Results Data} \label{appendix-B}
Temporary...



\end{document}
