\documentclass[preprint,12pt]{elsarticle}

%% Use the option review to obtain double line spacing
%% \documentclass[authoryear,preprint,review,12pt]{elsarticle}

%% Use the options 1p,twocolumn; 3p; 3p,twocolumn; 5p; or 5p,twocolumn
%% for a journal layout:
%% \documentclass[final,1p,times]{elsarticle}
%% \documentclass[final,1p,times,twocolumn]{elsarticle}
%% \documentclass[final,3p,times]{elsarticle}
% \documentclass[final,3p,times,twocolumn]{elsarticle}
%% \documentclass[final,5p,times]{elsarticle}
% \documentclass[final,5p,times,twocolumn]{elsarticle}

%% Fixing DOIs in the bibtex file, which otherwise don't play well with elsarticle
\usepackage{doi}

% Code listings. We use inconsolata for the font.
\usepackage{listings}
\usepackage{inconsolata}
\usepackage{color}
\usepackage{setspace}
\definecolor{codegreen}{rgb}{0,0.6,0}
\definecolor{codegray}{rgb}{0.5,0.5,0.5}
\definecolor{codepurple}{rgb}{0.58,0,0.82}
\definecolor{backcolour}{rgb}{0.95,0.95,0.92}

% Pandoc generates `passthrough' commands for inline code as described in 
% https://github.com/jgm/pandoc/issues/5696, we can just effectively ignore it
\newcommand{\passthrough}[1]{#1}

\lstset{
    backgroundcolor=\color{backcolour},   
    commentstyle=\color{codegreen},
    keywordstyle=\color{magenta},
    numberstyle=\ttfamily\tiny\color{codegray},
    stringstyle=\color{codepurple},
    basicstyle=\ttfamily\footnotesize\setstretch{1}, 
    breakatwhitespace=false,         
    breaklines=true,                 
    captionpos=b,                    
    keepspaces=true,                 
    numbers=left,                    
    numbersep=5pt,                  
    showspaces=false,                
    showstringspaces=false,
    showtabs=false,                  
    tabsize=2,
    % aboveskip=10pt, % Adjust spacing above listings
    % belowskip=0pt, % Adjust spacing below listings
}

% Less spacing between items in \begin{itemize} environments.
\usepackage{enumitem}
\setitemize{noitemsep}

% Better table appearance, particularly for the long ones
% https://tex.stackexchange.com/questions/373362/how-to-avoid-justified-text-in-tabularx
\usepackage{tabularx}
\usepackage{ragged2e}
\usepackage{booktabs}
\newcolumntype{L}[1]{>{\hsize=#1\hsize\RaggedRight} X}

% No footnotes across pages.
\interfootnotelinepenalty=10000

% Fix hyperref warnings related to author definitions.
% https://tex.stackexchange.com/questions/504814/package-hyperref-warning-token-not-allowed-in-a-pdf-string-pdfdocencoding
\hypersetup{pdfauthor=author}

% Set asset path.
% \graphicspath{{assets/}}

\journal{Forensic Science International: Digital Investigation}

\begin{document}

\begin{frontmatter}

\title{AKF: A modern synthesis framework for building datasets in digital forensics}

\author[unr]{Lloyd Gonzales\fnref{fn1}}
\ead{lloydg@unr.edu}
\fntext[fn1]{Present affiliation. The contents of this paper have been derived from this author's master's thesis of the same name.}
\author[unr]{Nancy LaTourrette\corref{cor1}}
\cortext[cor1]{Corresponding author.}
\ead{nancy@unr.edu}
\author[unr]{Bill Doherty}
\ead{wdoherty@unr.edu}
%% Author affiliation
\affiliation[unr]{organization={Department of Computer Science and Engineering, University of Nevada, Reno},%Department and Organization
            addressline={1664 North Virginia Street}, 
            city={Reno},
            postcode={89557}, 
            state={Nevada},
            country={USA}}

%% Abstract
\begin{abstract}
%% Text of abstract
{{substitute_abstract}}
\end{abstract}

%% Graphical abstract
% \begin{graphicalabstract}
%\includegraphics{grabs}
% \end{graphicalabstract}

%% Research highlights: 3-5 points of no more than 85 characters
\begin{highlights}
\item Broad improvements to the automation of creating digital forensic artifacts
\item Use of modern libraries and technologies to simplify artifact generation
\item Automatic comprehensive dataset documentation through the CASE ontology
\item Simplified dataset creation using a custom scripting language and generative AI
\end{highlights}
    
%% Keywords
\begin{keyword}
%% keywords here, in the form: keyword \sep keyword
Forensic image creation \sep
User emulation \sep
Artifact documentation \sep
Dataset documentation \sep
\end{keyword}

\end{frontmatter}

%% Beginning of main text
{{substitute_content}}


% Final sections, not numbered
\section*{CRediT authorship contribution statement}

\textbf{Lloyd Gonzales}: Conceptualization, Methodology, Software, Validation, Visualization, Writing -- original draft, Writing -- review \& editing. 
\textbf{Nancy LaTourrette}: Conceptualization, Methodology, Supervision, Writing -- review \& editing. 
\textbf{Bill Doherty}: Conceptualization, Writing -- review \& editing.

\section*{Declaration of generative AI and AI-assisted technologies in the writing process}

The authors used Grammarly and ChatGPT in making grammar and clarity improvements
during proofreading. The authors have reviewed the content produced by these tools
and take full responsibility for the content of this publication.

\section*{Declaration of competing interest}

The authors declare that they have no known competing financial interests or
personal relationships that could have appeared to influence the work reported
in this paper.

\section*{Acknowledgments}

This material is based upon work supported by the National Science Foundation under Award No. 2146146.

\section*{Data availability}

All data is hosted in code repositories that have been linked in the text.

%% Beginning of appendices
\appendix

%% Using bibtex to generate items, expecting article_bib.bib
% \bibliographystyle{elsarticle-num} 
\bibliographystyle{elsarticle-harv}\biboptions{authoryear}
\bibliography{article_bib}

\end{document}