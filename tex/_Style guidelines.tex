\begin{itemize}
\item
  Synthesizers are not to be italicized. Every mention of a synthesizer
  in a paragraph not focusing on that synthesizer should have a
  citation.
\item
  Bulleted lists should have their key element \textbf{bolded} where
  applicable.
\item
  Tools or libraries may be \passthrough{\lstinline!monospace!} to
  distinguish them from regular text. Use this sparingly; tools with
  ``actual names'' should have no formatting applied. (for example:
  \passthrough{\lstinline!uv!}, \passthrough{\lstinline!dfvfs!}, the
  Digital Forensics Virtual Filesystem, Pydantic, CASE,
  \passthrough{\lstinline!caselib!})
\item
  Italicization \emph{for emphasis} is allowed. Bolding for emphasis is
  not.
\item
  Italicization for ``proper'' terms, such as the names of modules, is
  allowed but should be limited in usage.
\end{itemize}

\passthrough{\lstinline!\\autoref!} is used to stick the word
``chapter'' and ``section'' into the actual reference. Don't do
hyperlinks like ``refer to chapter \textbf{39.0 - Abstract}'', just use
``refer to \textbf{39.0 - Abstract}''.

There's a special syntax for making code listings appear. Anything of
the form \passthrough{\lstinline"**!lst:(.*?)**"} will turn into an
\passthrough{\lstinline!\\autoref\{lst:label\}!}. Also, if the first
line of an \passthrough{\lstinline!lstlisting!} is of the form
\passthrough{\lstinline"!label|caption"}, it'll make the listing
accordingly. You can see the syntax towards the start of \autoref{scripting-background}.

For now, figures are just going to be done manually. We don't really
even have any yet, so I guess it's not an issue lol
