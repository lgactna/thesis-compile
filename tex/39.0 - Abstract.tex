\section{Dedication}\label{dedication}

To those in the osu! tournament community, without whom I would have
never embarked on this journey;

To my numerous teachers and professors, especially Keith Lightfoot,
Rodney Rogers, Marc Miller, and Gabbi Bachand, who I have limitless and
appreciation and admiration for;

To those part of the United States Cyber Team and the broader CTF
community, for igniting my interest in digital forensics and supporting
me even when I flailed like a fish out of water;

And, of course, to my friends, family, and bed, who provided motivation
when there was none.

\section{Acknowledgments}\label{acknowledgments}

\st{i acknowledge i am lazy\ldots{}}

I want to express my immense gratitude to Nancy Latourrette for her
support, guidance, and mentorship throughout the development of this
thesis. This thesis would be nowhere without her ideas and experience,
and I am truly grateful and honored to have been able to work with her
throughout this experience.

I would also like to thank Bill Doherty for his review of a prior paper,
from which some of this content is derived.

\section{Abstract}\label{abstract}

As our world becomes increasingly dependent on technology, the
advancement of digital forensics has become a key focus in the fight
against cybercrime. The field depends greatly on the availability of
disk images, network captures, and other forensic datasets for
education, tool validation, and research. However, real-world datasets
often contain sensitive information that may be difficult to remove,
making them difficult to distribute publicly. As a result, researchers
and educators can encounter gaps in available datasets, often leading to
the manual development of new, suitable datasets. While viable, this
approach is time-consuming and rarely produces datasets that accurately
reflect real-world scenarios suitable for comprehensive training and
education. In turn, there is ongoing research into forensic
synthesizers, which automate the process of creating unique synthetic
datasets that can be publicly distributed without legal and logistical
concerns.

This thesis introduces the \emph{automated kinetic framework}, or AKF, a
modular synthesizer for creating and interacting with virtualized
environments to simulate user activity. AKF significantly improves upon
the architectural designs of prior synthesizers while maintaining
feature parity and usability. Additionally, AKF leverages the CASE
standard to provide human- and machine-readable reporting, exposing
low-level details in a searchable format. Finally, AKF provides options
for leveraging generative AI to develop high-level scenarios as well as
individual artifacts. These contributions are intended to not only
improve the speed at which synthetic datasets can be created, but also
ensure the long-term usefulness of AKF-generated datasets and the
framework as a whole.
