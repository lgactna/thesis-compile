AKF was built with the expectation that it would be easy to maintain,
develop, and extend. Indeed, there are several use-cases that AKF does
not fulfill as of writing, primarily due to time constraints. Future
work related to AKF can be divided into two groups -- tasks that
integrate cutting-edge advancements to implement new features, and tasks
that extend or improve existing functionality. \autoref{open-ended-automation-with-ai} describes the application of recent AI
developments towards addressing tasks relevant to forensic dataset
development; the remaining sections focus on extending existing concepts
in AKF.

\section{Open-ended automation with
AI}\label{open-ended-automation-with-ai}

During the development of AKF, OpenAI announced the release of Operator,
an ``agent'' capable of automating tasks on webpages using natural
language prompts \cite{openaiIntroducingOperator2025}. Rather than
using a browser automation framework like Playwright or Selenium, it
leverages its own browser to interact with webpages. This allows users
to provide Operator with a high-level goal, which it can then convert to
concrete webpage actions to achieve that goal.

The example provided by OpenAI involves Operator searching Allrecipes
for a clam linguine recipe, then ordering the ingredients for the
linguine through Instacart. Various sensitive actions, such as inputting
payment information or logging into a service, are delegated to the user
to complete. Operator is also capable of identifying ambiguity in a task
and prompting the user, such as asking the user which store to use for
ordering Instacart items. This demonstrates multiple notable features
that are relevant to forensic synthesizers; in particular, it
demonstrates the ability to both \emph{interpret} and \emph{interact}
with arbitrary GUIs, as well as the ability to convert human prompts
into a sequence of automated actions that may change as the agent
discovers new information or encounters unexpected issues.

It is powered by what OpenAI calls its ``Computer-Using Agent'', or CUA,
which is trained to interact with a virtual machine by accepting natural
language and a screenshot of a virtual monitor
\cite{openaiComputerUsingAgent2025}. It leverages OpenAI's GPT-4o
model, following a three-step process in which it analyzes screenshots
of the virtual desktop, conducts reasoning to determine the necessary
steps to achieve a task (using its own prior context), and executes
actions on the virtual machine. OpenAI notes that CUA can reliably
perform simple tasks that a human would normally take, such as
navigating to specific categories of websites or repeating UI
interactions. However, it struggles with UIs that it has not encountered
before and tends to be inefficient or hallucinate on more complex
tasking.

The challenge of automating open-ended tasks through AI is not new.
Multiple benchmarks (mentioned in OpenAI's articles), including
WebVoyager, WebArena, and OSWorld, were developed in early 2024 to
provide examples of typical webpage and OS interaction tasks done by
humans
\cite{zhouWebArenaRealisticWeb2024,heWebVoyagerBuildingEndtoEnd2024,xieOSWorldBenchmarkingMultimodal2024}.
CUA is stated to achieve state-of-the-art results in these benchmarks,
and therefore represents the current ability of AI to address open-ended
tasking. Although CUA has clear limitations and falls well behind human
performance on these benchmarks, Operator demonstrates significant
progress in the ability of AI models to replicate actions that are often
performed by real users. Even in its current state, CUA is likely able
to automate the ``simple'' tasking that is characteristic of generating
background noise for forensic datasets, such as browsing news sites,
interacting with social media platforms, and more.

In the near future, it is likely that works similar to CUA or Operator
will be capable of fully automating human actions as part of a larger
scenario with a high degree of reliability and accuracy. Not only does
this accelerate the process of building forensic datasets, it also
improves the variety of applications that can be used as part of a
scenario due to the flexibility of these models. For educators whose
focus is to build scenarios with investigative and analytic value to
students, this may be exactly what is needed to streamline scenario
development. What does this mean for synthesizers?

First, the verbosity of writing a script and passing it to a synthesizer
is still valuable, especially in contexts where the non-determinism and
opacity of an AI model may not be acceptable. Non-determinism is often
acceptable in educational contexts, so long as the actions taken by the
model are logged and can be verified after the fact. However, the
development of datasets for research and tool validation may require
that actions are taken in a specific manner every time the dataset is
generated. Several works have described the non-deterministic nature of
LLMs in multiple distinct tasks
\cite{astekinExploratoryStudyHow2024,songGoodBadGreedy2024,ouyangEmpiricalStudyNonDeterminism2025},
which negatively impacts the reproducibility of results -- an important
quality of forensic datasets as described by Grajeda et al.
\cite{grajedaAvailabilityDatasetsDigital2017}.

Second, these developments are not \emph{incompatible} with
synthesizers, and should instead be seen as an option to complement
them. The capabilities provided by Operator could likely be built into
AKF as part of its internal library or an OS-specific agent, providing
users with access to both verbose imperative/declarative scripts as well
as simpler natural language prompts when automating actions and building
scenarios. Additionally, these agents cannot (currently) act as a
substitute for physical artifact generation, in which the underlying
filesystem or disk image must be edited to fulfill a certain task.

\section{Alternative platform
support}\label{alternative-platform-support}

Perhaps the most impactful example is implementing support for other
desktop environments, such as Mac and Linux. At a high level, this
requires implementing all three artifact generation methods described in
\autoref{chapter-four}.

Consistent with AKF's focus on using existing automation frameworks
through agents to implement application-specific functionality, much of
the effort for supporting other platforms requires writing and deploying
a new OS-specific agent. It is likely that much of the same code and
overall design can be shared between the Windows agent and other
platforms. This is largely possible through the portability of Python
and the lack of OS-specific assumptions made by RPyC. Certain automation
frameworks may have significant cross-platform support (such as
Playwright and \passthrough{\lstinline!pywinauto!}, which supports
Windows, Mac, and Linux), though implementing functionality for other
applications may require more effort.

In general, implementing logical agentless generation through VirtualBox
is likely straightforward, in large part because Oracle supports
VirtualBox Guest Additions on MacOS and a large variety of Linux
distributions. Similarly, the physical artifact generation implemented
in AKF for FAT32 and NTFS (the most common filesystems for Windows) may
be extensible to other disks, such as ext4 (the default for modern Linux
distributions).

The challenge here lies in accounting for artifacts or mechanisms unique
to other operating systems. Some features, such as the use of
PowerShell/WinRM to carry out actions on Windows, are analogous to
Bash/SSH on Unix-based systems and can be adapted accordingly. However,
some concepts are truly unique to an operating system, such as
performing modifications to the operating system through registry keys,
and may require more effort to adapt the same outcomes to other
platforms.

\section{Additional interface
implementations}\label{additional-interface-implementations}

Another example is extending the various concrete implementations of AKF
interfaces to other technologies that may have better support for
specific host and guest platforms. AKF currently provides a
VirtualBox-based implementation of the hypervisor-agnostic interface
provided as part of \passthrough{\lstinline!akflib!}, largely depending
on the VirtualBox Guest Additions software to function. Indeed, this is
the approach taken by virtually all prior synthesizers that require
virtualization -- existing works in implementing synthesizer
functionality in VirtualBox have contributed significantly to its
adoption in these synthesizers, as well as AKF.

However, prior synthesizers have also considered KVM/Qemu and VMWare as
hypervisors; in particular, Forensig2 leveraged Qemu as its
virtualization platform \cite{mochForensicImageGenerator2009}. There
are several motivations for using other synthesizers, including support
on different host platforms, performance, and available functionality.
For example, Hwang et al.~compared the performance and low-level
features of multiple hypervisors, including Hyper-V, KVM/Qemu, and
VSphere, finding significant variability between hypervisors when
running workloads and applications
\cite{hwangComponentbasedPerformanceComparison2013}. It may be the
case that certain scenarios or host machines are better suited for other
hypervisors, or even a multi-hypervisor environment. It is worth noting
that VMWare has similar guest-specific software to VirtualBox, with
multiple Python libraries for interacting with guests on VSphere such as
\passthrough{\lstinline!pyvmomi!} \cite{VmwarePyvmomi2025}.

Similarly, rather than create new implementations of existing
interfaces, it may also be worth building new interfaces entirely. One
particular example is the implementation of agents in languages other
than Python. This may cover needs for specific tasks (such as actions
that can be automated using a library for which no Python equivalent
exists) or specific deployment restrictions (such as avoiding specific
forms of synthesis pollution). In general, the implementation for an
agent can be written in any language, so long as a Python API exists;
the burden would be on the author to handle any communication or
discrepancies across disparate languages, such as writing a TCP-based
protocol for issuing commands.

\section{Distribution}\label{distribution}

Depends a lot on what we do implement, but this would be stuff like
Packer/Vagrant and any image ``squeezing'' that's done as part of
\autoref{chapter-five}

\section{Mobile synthesis}\label{mobile-synthesis}

Finally, while outside the scope of this thesis, there is also the
challenge of building a synthesizer for non-desktop platforms,
particularly mobile devices. For many people, mobile devices are the
primary means through which we interact with the digital world, causing
them to play a unique role in investigations
\cite{chernyshevMobileForensicsAdvances2017}. They can contain data
from many facets of our lives, including text messages, location
history, images, application logs, and other information that can
provide insight into an individual's actions during a period of interest
\cite{sutiknoCapabilitiesCellebriteUniversal2024}. This information
can be combined with other investigative methods, such as desktop and
conventional forensics, to form a better understanding of a larger
scenario.

Naturally, this means that there is a need for datasets in digital
forensics, as well. Grajeda et al.~identify a small number of existing
mobile dataset collections, including those containing Android malware,
Android application files, and smartphone disk images
\cite{grajedaAvailabilityDatasetsDigital2017}. However, these
datasets are far less common than their desktop counterparts. When
compared to the hundreds of desktop disk images hosted on Digital
Corpora and CFReDS, there are only around 25 mobile disk images on the
same platforms. A mobile-specific survey conducted by Gonçalves et
al.~in 2022 found not only a low availability of mobile images, but also
lack the recency and variety of data that would be present in a modern
investigation \cite{goncalvesRevisitingDatasetGap2022}.

There has been limited work in developing forensic synthesizers for
mobile platforms. Two notable examples are FADE
\cite{ceballosdelgadoFADEForensicImage2022}, developed by Delgado et
al.~in 2022, and an unnamed framework developed by Demmel et al.~in 2024
\cite{demmelDataSynthesisGoing2024}. FADE operates largely through
physical artifact generation, which it achieves by extracting and
mounting partitions from an emulated, rooted Android device using the
Android Debug Bridge. It then modifies application-specific database
files to create artifacts such as phone call entries and text messages.

In contrast, Demmel et al.~generate data using a logical agentless
approach, using a tool called AndroidViewClient (AVC) to send keypresses
and touch gestures. Unlike VMPOP, which uses hardcoded mouse movements
to click on GUI-based elements, this framework leverages AVC to dump all
interactable UI elements and directly ``touch'' these elements once the
desired element has been found. This allows it to implement more
application-specific functionality, such as using WhatsApp and opening
Google Chrome. There has also been broader work in automating actions as
part of testing pipelines for Android applications
\cite{janickiObstaclesOpportunitiesDeploying2012,nagowahNovelApproachAutomation2012,linares-vasquezHowDevelopersTest2017},
though these largely address actions at the application scope, rather
than automating actions across the entire device. Of note is the lack of
iOS-relevant synthesizer functionality, perhaps due to the greater
difficulty of working on a closed-source operating system with fewer
options for exposing internal functionality.

It is possible that \emph{some} of AKF's architecture could be adapted
to support mobile dataset generation, primarily by interacting with
Android emulators and existing developer tools. However, the isolation
around individual Android applications suggests that it may be difficult
to use a logical agent-based approach to automating application
activity; this simultaneously implies that it may be easier to use
physical approaches to generating information, because the locations of
generated artifacts are likely to be smaller in scope. More research is
needed to determine available avenues through which Android datasets can
be constructed. This is especially true for iOS, for which there are far
fewer resources.
