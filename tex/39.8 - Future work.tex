AKF was built with the expectation that it would be easy to maintain,
develop, and extend. Indeed, there are several use cases that AKF does
not fulfill as of writing, primarily due to time constraints. Future
work related to AKF can be characterized as either tasks that integrate
cutting-edge advancements to implement new features or tasks that extend
or improve existing functionality. \autoref{open-ended-automation-with-ai} describes the application of recent AI
developments towards addressing tasks relevant to forensic dataset
development; the remaining sections focus on extending existing concepts
in AKF.

\section{Open-ended automation with
AI}\label{open-ended-automation-with-ai}

During the development of AKF, OpenAI announced the release of Operator,
an ``agent'' capable of automating tasks on webpages using natural
language prompts \cite{openaiIntroducingOperator2025}. Rather than
using a browser automation framework like Playwright or Selenium, it
leverages its own browser to interact with webpages. Users can provide
Operator with a high-level goal, which it then converts to concrete
actions on its browser to achieve that goal.

OpenAI provides an example in which Operator searches Allrecipes for a
clam linguine recipe and then orders the ingredients for the linguine
through Instacart. Various sensitive actions during this process are
delegated to the user to complete, such as inputting payment information
or logging into a service. Operator is also capable of identifying
ambiguity in a task and prompting the user for clarification, such as
asking which store to use for ordering items through Instacart. This
example demonstrates multiple notable features that are relevant to
forensic synthesizers; in particular, it demonstrates the ability to
both \emph{interpret} and \emph{interact} with arbitrary GUIs, as well
as the ability to convert human prompts into a sequence of automated
actions that may change as the agent discovers new information or
encounters unexpected issues.

It is powered by what OpenAI calls its ``Computer-Using Agent,'' or CUA,
which is trained to interact with a virtual machine by accepting natural
language and a screenshot of a virtual monitor
\cite{openaiComputerUsingAgent2025}. It leverages OpenAI's GPT-4o
model, following a three-step process in which it analyzes screenshots
of the virtual desktop, conducts reasoning to determine the necessary
steps to achieve a task (using prior context), and executes actions on
the virtual machine. OpenAI notes that CUA can reliably perform simple
tasks that a human would usually perform, such as navigating to specific
categories of websites or repeating UI interactions. However, it
struggles with some interfaces it has not encountered before and tends
to be inefficient or hallucinate on more complex tasks.

The challenge of automating open-ended tasks through AI is not new.
Multiple benchmarks (mentioned in OpenAI's articles), including
WebVoyager, WebArena, and OSWorld, were developed in early 2024 to
provide examples of typical webpage and OS interaction tasks performed
by humans
\cite{zhouWebArenaRealisticWeb2024,heWebVoyagerBuildingEndtoEnd2024,xieOSWorldBenchmarkingMultimodal2024}.
CUA is stated to achieve state-of-the-art results in these benchmarks,
representing the current ability of AI to address open-ended tasking.
Although CUA has clear limitations and falls well behind human
performance on these benchmarks, Operator demonstrates significant
progress in the ability of AI models to replicate actions that are often
performed by real users. Even in its current state, CUA can likely
automate the ``simple'' tasking involved in generating background noise
for forensic datasets, such as browsing news sites, interacting with
social media platforms, and more.

In the near future, works similar to CUA or Operator will likely be
capable of fully automating human actions as part of a larger scenario
with high reliability and accuracy. Improvements to interacting with
previously unseen GUI-based applications will increase the variety and
complexity of actions that can be automated. These capabilities are
extremely valuable in synthetic dataset generation, as they dramatically
reduce the time needed to implement artifact generation functionality
for novel technologies or developments.

What do these developments mean for ``conventional'' synthesizer
architectures, which depend on scripts instead of natural language?
First, the verbosity of writing a script and passing it to a synthesizer
is still valuable, particularly in contexts where the non-determinism
and opacity of an AI model may not be acceptable. Several works have
described the non-deterministic nature of LLMs in multiple distinct
tasks
\cite{astekinExploratoryStudyHow2024,songGoodBadGreedy2024,ouyangEmpiricalStudyNonDeterminism2025},
which negatively impacts the reproducibility of results -- an important
quality of forensic datasets as described by Grajeda et al.
\cite{grajedaAvailabilityDatasetsDigital2017}.~Non-determinism is
often acceptable in educational contexts so long as the actions taken by
the model are logged and can be verified after the fact. However, the
development of datasets for research and tool validation may require
that a set of instructions always generates the same dataset every time,
ensuring reproducibility from the instructions alone.

Second, these developments are not \emph{incompatible} with synthesizers
and should instead be seen as an option to complement them. The
capabilities provided by Operator could likely be built into AKF as part
of its internal library or an OS-specific agent, providing users with
access to both verbose imperative/declarative scripts as well as simpler
natural language prompts when automating actions and building scenarios.
Additionally, these agents cannot (currently) act as a substitute for
physical artifact generation, in which the underlying filesystem or disk
image must be edited to fulfill a particular task. Since these agents
are trained to work with ``live'' operating systems and applications,
they are likely unsuitable for automating physical operations on
filesystems and disk images.

\section{Alternative platform
support}\label{alternative-platform-support}

One future extension is implementing support for other desktop
environments, such as Mac and Linux. At a high level, this requires
implementing artifact generation methods described in \autoref{chapter-four} for each platform.

Consistent with AKF's focus on using existing automation frameworks
through agents to implement application-specific functionality, much of
the effort for supporting other platforms requires writing and deploying
a new OS-specific agent. Much of the code and overall design can be
inherited from the Windows agent, enabled through the portability of
Python and the lack of OS-specific assumptions made by RPyC. Certain
automation frameworks may have significant cross-platform support (such
as Playwright and \passthrough{\lstinline!pywinauto!}, which support
Windows, Mac, and Linux), though implementing functionality for other
applications may require more effort.

Implementing logical agentless generation through VirtualBox is likely
straightforward, mainly because Oracle supports Guest Additions on MacOS
and many Linux distributions. Similarly, the physical artifact
generation implemented in AKF for FAT32 and NTFS (the most common
filesystems for Windows) is likely extensible to other disks, such as
ext4 (the default for modern Linux distributions). This is because
\passthrough{\lstinline!dfvfs!}, the library used to support disk image
and filesystem editing, supports many filesystems, partitioning
standards, and disk images commonly used by other operating systems.

The primary challenge in cross-platform support lies in accounting for
artifacts or mechanisms unique to other operating systems. Some
features, such as PowerShell/WinRM and Bash/SSH, are analogous between
Windows and Unix-based systems and can be adapted accordingly. However,
some concepts are truly unique to an operating system, such as
performing modifications to the operating system through Windows
registry keys, and may require more effort to adapt the same outcomes to
other platforms.

\section{Additional interface
implementations}\label{additional-interface-implementations}

Another future extension involves developing new concrete
implementations of AKF interfaces using other technologies that may have
better support for specific host and guest platforms. AKF currently
provides a VirtualBox implementation of the hypervisor-agnostic
interface provided as part of \passthrough{\lstinline!akflib!}, largely
depending on the VirtualBox Guest Additions software to function.
Indeed, this is the approach taken by virtually all prior synthesizers
that require virtualization; existing work leveraging VirtualBox to
implement automation functionality has contributed significantly to its
adoption across many synthesizers, including AKF.

However, prior synthesizers have also considered KVM/Qemu and VMWare as
hypervisors; in particular, Forensig2 leveraged Qemu as its
virtualization platform \cite{mochForensicImageGenerator2009}.
Several considerations exist for using other synthesizers, including
support on different host platforms, performance, and available
functionality. For example, Hwang et al.~compared the performance and
low-level features of multiple hypervisors, including Hyper-V, KVM/Qemu,
and VSphere, finding significant variability between these hypervisors
when running workloads and applications
\cite{hwangComponentbasedPerformanceComparison2013}. It may be the
case that specific scenarios or host machines are better suited for
other hypervisors or even a multi-hypervisor environment. It is worth
noting that VMWare has guest-specific software similar to the VirtualBox
Guest Additions; multiple Python libraries exist for interacting with
guests on VSphere, such as \passthrough{\lstinline!pyvmomi!}
\cite{VmwarePyvmomi2025}.

Similarly, rather than create new implementations of existing
interfaces, it may also be worth building new interfaces entirely. One
example is the implementation of agents in languages other than Python.
Using non-Python agents may cover needs for specific tasks (such as
actions that can be automated using a library for which no Python
equivalent exists) or specific deployment restrictions (such as avoiding
specific forms of synthesis pollution). In general, the implementation
for an agent can be written in any language, so long as a corresponding
Python API exists; the burden would be on the author to handle any
communication or discrepancies across disparate languages, such as
writing a TCP-based protocol for issuing commands.

\section{Distribution}\label{distribution}

One notable development of prior synthesizers not implemented as part of
AKF is the various improvements to the distribution of generated
datasets, particularly in environments with limited bandwidth or storage
space. AKF primarily addresses challenges in the \emph{creation} of
forensic datasets rather than their \emph{distribution}. This section
addresses prior efforts to reduce the size of forensic datasets and how
they could be integrated into AKF in the future.

The size of forensic datasets can be trivially reduced in various ways.
For example, a full forensic dataset can be compressed using a standard
algorithm such as LZMA. Specific core outputs, such as disk images, can
be stored in a dynamically expandable image format such as VDI or VHD.
Such formats only use as much space as is currently used on the disk
image itself, even if the size reported to the operating system is much
larger. Such reductions are ``lossless'' in that they can be reversed to
reflect what a forensic analyst would encounter using real hardware.

However, more aggressive ``lossy'' reductions can be performed to
further reduce dataset size. These can be broadly described as one of
two approaches -- either irrelevant data is outright removed, or the end
user must take additional effort to finish preparing the forensic
scenario for use after downloading relevant files (which are
significantly smaller than the finished forensic dataset itself). The
first approach reduces realism in exchange for a smaller dataset size;
the second approach requires additional processing time in exchange for
reduced network transfer sizes.

The first method describes the process of simply removing data that is
not deemed to be relevant for educational or research purposes. For
example, suppose an instructor is interested in demonstrating only the
recovery and analysis of SQLite databases in Chrome's AppData directory.
For this scenario, there is no need to include files from most other
directories on the filesystem. One option for achieving this is to use a
logical disk image format, such as the AD1 format, which is a collection
of files (much like a ZIP archive) instead of a collection of physical
disk units (like a standard disk image).

A more notable approach to removing irrelevant information is described
by Russell et al.~as part of their work developing SFX
\cite{russellForensicImageDescription2012}. Their approach,
described as ``partition squeezing,'' is based on the idea that the
contents of files irrelevant to an educational scenario can be truncated
to the size of a single filesystem cluster or block. (In some
filesystems, a file can be further truncated to fit entirely within
filesystem metadata structures, such as resident files in NTFS.) This
preserves the file and directory structure of the overall filesystem
(thus preserving realism) while also removing data that is unlikely to
be useful to students performing analysis. Thus, the educational value
of the image remains the same, even if some data is missing. Russell et
al.~note that additional space can be saved by replacing irrelevant
files with hard links to other files, particularly those deep within the
filesystem that are highly unlikely to be analyzed in detail by
students. This entirely eliminates the original contents of the file
while preserving most of its attributes.

The use of logical images and partition squeezing, of course, is not
suitable in all cases. In particular, logical images generally do not
include slack or unallocated space. The partition squeezing approach
destroys data that may be required for certain forensic analysis tools
to function, as well as data that might be of interest to users of the
dataset in other research or educational contexts than initially
intended.

AKF does not provide any functionality for directly generating logical
images or performing partition squeezing. However, generating logical
images is relatively straightforward using a publicly available tool
such as FTK Imager, so long as the scenario developer is aware of
relevant directories that should be included -- a process that might
occur with AKF's reporting features in mind. Partition squeezing could
be achieved by leveraging the filesystem-aware functionality of
libraries such as \passthrough{\lstinline!libtsk!} and
\passthrough{\lstinline!dfvfs!}, which is used as part of AKF's physical
planting techniques described in \autoref{akf-implementation}. However, a more straightforward approach is to
mount the filesystem and truncate all irrelevant files to the size of a
single block or cluster. Such functionality could be implemented either
in AKF or as part of a wholly independent tool.

The second method describes various techniques in which end users must
take additional actions after downloading relevant materials before a
forensic dataset is ready for use. One example described by Scanlon et
al.~is the use of ``evidence packages,'' in which users are provided
with a base image of a single operating system that can be reused to
build multiple forensic datasets through EviPlant
\cite{scanlonEviPlantEfficientDigital2017}. After a scenario
developer has finished creating artifacts (whether manually or through a
synthesizer), they can use EviPlant's ``diffing'' tool to determine and
extract differences between the base image and the developer's current
disk state. These differences can be distributed to users, who can then
use EviPlant's ``injection'' tool on the previously distributed base
image to recreate the final disk image.

This approach can described more generally as ``differential imaging'',
in which only differences from some initial files are saved and
transmitted to users. These differences are often significantly smaller
than the disk images they are designed to build and are much easier to
distribute than complete disk images. (This is very similar to the
approach used by Vagrant to distribute and prepare reproducible virtual
machines, as described in \autoref{setup-and-basic-usage}.) In exchange, each user must spend additional
processing time before the image can be used for analysis. EviPlant's
injection tool operates through a combination of logical and physical
planting, which requires time, software, and resources that are not
required when using and distributing complete disk images. This is an
acceptable cost in some cases; for example, not all students in a remote
classroom may have high-speed internet, limiting the size of files that
can be distributed. Again, although AKF does not implement differential
imaging, this functionality can likely be implemented as part of an
independent tool or module.

\section{Mobile synthesis}\label{mobile-synthesis}

Finally, although well outside the scope of this thesis, there is also
the challenge of building a synthesizer for non-desktop platforms,
particularly mobile devices. For many people, mobile devices are the
primary means of interacting with the digital world, causing them to
play a unique role in investigations
\cite{chernyshevMobileForensicsAdvances2017}. Mobile devices can
contain data from many facets of one's life, including text messages,
location history, images, application logs, and other information that
can provide insight into an individual's actions during a period of
interest \cite{sutiknoCapabilitiesCellebriteUniversal2024}. This
information can be combined with other investigative methods, such as
desktop and conventional forensics, to form a better understanding of a
larger scenario.

Naturally, this means that there is a need for mobile datasets in
digital forensics, as well. Grajeda et al.~identify a small number of
existing mobile dataset collections, including those containing Android
malware, Android application files, and smartphone disk images
\cite{grajedaAvailabilityDatasetsDigital2017}. However, these
datasets are far less common than their desktop counterparts. Compared
to the hundreds of desktop disk images hosted on Digital Corpora and
CFReDS, there are only around 25 mobile disk images on the same
platforms. A mobile-specific survey conducted by Gonçalves et al.~in
2022 found not only a low availability of mobile images but also a lack
of background noise and recent applications that would be present in a
modern investigation \cite{goncalvesRevisitingDatasetGap2022}.

There has been limited work in developing forensic synthesizers for
mobile platforms. Two notable examples are FADE
\cite{ceballosdelgadoFADEForensicImage2022}, developed by Delgado et
al.~in 2022, and an unnamed framework developed by Demmel et al.~in 2024
\cite{demmelDataSynthesisGoing2024}. FADE operates primarily through
physical artifact generation, which it achieves by extracting and
mounting partitions from an emulated, rooted Android device using the
Android Debug Bridge. It then modifies application-specific database
files to create artifacts such as phone call entries and text messages.

In contrast, Demmel et al.~generate data using a logical agentless
approach, using a tool called AndroidViewClient (AVC) to send keypresses
and touch gestures \cite{demmelDataSynthesisGoing2024}. Unlike
VMPOP, which uses hardcoded mouse movements to click on GUI-based
elements, this framework leverages AVC to expose all interactable UI
elements and programmatically ``touch'' these elements once the desired
element has been found. This allows it to implement more complex
application-specific functionality, such as using WhatsApp and opening
Google Chrome. There has also been broader work in automating actions as
part of testing pipelines for Android applications
\cite{janickiObstaclesOpportunitiesDeploying2012,nagowahNovelApproachAutomation2012,linares-vasquezHowDevelopersTest2017},
though these primarily address application-specific actions, rather than
automating actions across the entire device. Of note is the lack of
iOS-relevant synthesizer functionality, perhaps due to the greater
difficulty of working on a closed-source operating system with fewer
options for exposing internal functionality.

It is possible that \emph{some} of AKF's architecture could be adapted
to support mobile dataset generation, primarily by interacting with
Android emulators and existing developer tools. However, the isolation
around individual Android applications suggests that it may be difficult
to use a logical agent-based approach to automating application
activity; a Python agent running as an application is unlikely to have
full filesystem access on a non-rooted device. For simple
application-specific artifacts, using physical techniques and editing
artifacts in an application's allocated directory may be more effective.
Much like Windows, however, Android applications may generate both
application-specific artifacts and system artifacts (such as system
logs) located elsewhere in the filesystem.

More research is needed to determine viable options for constructing
Android datasets. This is especially true for iOS, for which there are
significantly fewer resources.
