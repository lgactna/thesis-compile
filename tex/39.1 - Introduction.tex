\section{History of digital
forensics}\label{history-of-digital-forensics}

Digital forensics is a relatively new field that rose out of the growing
need to address computer crimes. Through the explosion of computing that
has occurred over the last 30 years, computers have become critical in
virtually every modern industry. In turn, they have become the target of
many attacks. Much like conventional crimes, these cyberattacks leave
behind traces of digital evidence that can be analyzed to determine the
methodologies of the attackers, the scope and extent of the damage, and
more. Today, digital forensics is also applicable to a variety of
non-criminal contexts, including research and corporate investigations,
further highlighting the need for quality educational material and
hands-on training for new forensic analysts.

The origins of digital forensics can be traced to the 1980s, during
which computers began to be adopted by the general public. Prior to
this, computers were largely restricted to industry, academia, and
governments with dedicated infrastructure and staff
\cite{pollittHistoryDigitalForensics2010}. However, with the
introduction of PCs more accessible to the typical consumer, such as the
Commodore 64 and the IBM PC, computer usage within the public grew. And
in turn, so did crimes committed with computers; people discovered that
they could hack telephone networks to illegally obtain software and
``free'' telephone services \cite{jonesInsightDigitalForensics2022}.
In 1983, Canada would be the first government to amend its criminal code
to cover cybercrime; the United States, the United Kingdom, and
Australia would follow suit in the following decade.

During this time, investigations were relatively ad-hoc and very simple.
Teams often built their own software to analyze devices, though a few
hobbyists and software vendors began to develop dedicated forensic
tools. However, computers had yet to truly enter the mainstream; the
vast majority of individual computer owners were (relatively wealthy)
computer hobbyists rather than the general public. In 1984, the U.S.
Census Bureau determined that only about 8.2\% of U.S. households owned
a computer \cite{robertkominskiComputerUseUnited1988}, nearly
doubling to 15\% in 1989 \cite{robertkominskiComputerUseUnited1991}.

As technology continued to advance throughout the 1990s, digital
forensics -- and computing as a whole -- began to quickly grow in scope
and importance. The rise of mobile devices and the internet drastically
changed the role of computing in the eyes of the public. With it came
the rise of cybercrime and the importance of computer investigations.
Early investigations were often performed by investigators who happened
to be experienced with computers, rather than those with formal digital
forensics training \cite{hargreavesDigitalForensicsEducation2017}.

By the turn of the century, digital forensics had grown beyond law
enforcement and niche cases to a major focus of research and education.
The first use of the phrase ``computer forensics'' in ACM literature
appeared in 1999 \cite{cooperStandardsDigitalForensics2010}, with
the first Digital Forensic Research Workshop (DFRWS) held in 2001 to
identify priorities in the growing field of digital forensics
\cite{palmerRoadMapDigital2001}. A particularly notable case was
that of the September 11, 2001 attacks, in which computers were found to
contain meaningful evidence related to the organization and planning of
the attack. Intelligence communities and law enforcement agencies
throughout the world began establishing digital forensics teams,
shifting the agenda of digital forensics from individuals and small
teams to governments and professional organizations.

It was during this time that tools such as EnCase and Forensic Toolkit
(better known as FTK) evolved to become dedicated products that remain a
mainstay of the digital forensics field today
\cite{pollittHistoryDigitalForensics2010}. At the same time,
anti-forensics began to grow in popularity; numerous tools and resources
were developed with the express goal to exploit and hinder the digital
forensics process, generally with the stated motivation of guarding
users' privacy and protecting users from punishment for undesirable
computer activity
\cite{geigerEvaluatingCommercialCounterForensic2005,harrisArrivingAntiforensicsConsensus2006}.

By this time, it had become abundantly clear that there needed to be
dedicated training to develop specialists in digital forensics.
Undergraduate and graduate programs dedicated to the study of digital
forensics were gradually developed
\cite{andersonComparativeStudyTeaching2006}, along with numerous
efforts to standardize and improve digital forensics curricula
\cite{cooperStandardsDigitalForensics2010,nanceDigitalForensicsDefining2009,nanceDigitalForensicsDefining2010}.
This became particularly important due to the growing importance of
digital forensics from a legal perspective; analysts must follow a
strict procedure that ensures the admissibility of digital evidence into
court \cite{conklinComputerForensics2022}. Simultaneously, analysts
must have the experience needed to provide an unbiased, accurate opinion
of digital evidence in court as an expert witness.

Digital forensics continues to be an important focus in industry -- and
in turn, education and research. Broadly, the Bureau of Labor Statistics
projects that information security employment will grow 32\% over the
next decade from 2022 to 2032, adding over 50,000 new jobs
\cite{bureauoflaborstatisticsu.s.departmentoflaborInformationSecurityAnalysts2023}.
The diversity and depth of digital forensics continues to grow with the
broad variety of devices and software involved in modern computing.

With the growing importance and complexity of digital forensics, there
is a clear need for high-quality, realistic data for researchers and
instructors alike. Yet, there continues to be significant gaps in both
the quantity and variety of material suitable for digital forensic
training. The primary barrier to availability has been the privacy and
legal concerns associated with releasing real-world data
\cite{garfinkelForensicCorporaChallenge2007}. From a research
perspective, the result is that many researchers develop their own
datasets, often with a very narrow scope and a very low degree of
reproducibility
\cite{garfinkelBringingScienceDigital2009,grajedaAvailabilityDatasetsDigital2017}.
From an education perspective, the result is that most training material
is either manually created by instructors or reused from existing
sources. That is, researchers and instructors alike often create their
own datasets because publicly available corpora is insufficient;
however, this is a time-consuming process that responds slowly to
changes in technology and software.

Various efforts have been made to automate or streamline the process of
creating new forensic datasets from high-level descriptions and
predefined forensic artifacts. These forensic synthesis frameworks (also
referred to as ``synthesizers'' throughout this paper) include a variety
of features that are geared towards research and education. This
includes the rapid creation of datasets for a large classroom, the
generation of metadata useful in tool validation, and the ability to
export a variety of forensic artifacts from the synthesizer. In
particular, these frameworks enable instructors to create images that
allow students to learn about a specific forensic technique while
emulating some of the real-world challenges that forensic analysts face
in industry.

However, there is still much work to be done towards increasing the
accessibility and flexibility of these frameworks. Before exploring
synthesizers in greater detail, it is necessary to first understand the
purpose and characteristics of forensic datasets in a variety of
contexts. Doing so will not only outline \emph{why} the development of
synthesizers is necessary, but also \emph{what} features these
synthesizers must provide. Once we have established this foundation, we
can begin exploring \emph{how} a synthesizer should be architected --
the focus of the reminder of this thesis.

\section{Purpose of forensic
datasets}\label{purpose-of-forensic-datasets}

\subsection{In industry}\label{in-industry}

Before considering the use of forensic datasets in research and
education, we begin by exploring how these datasets are acquired and
used in the ``real world'' -- that is, investigations made by
professionals in industry.

In practice, forensic datasets -- and digital forensics as a whole --
are used for a variety of purposes. In particular, Conklin et
al.~identify three primary cases in which digital forensics may be
performed \cite{conklinComputerForensics2022}:

\begin{itemize}
\tightlist
\item
  \emph{To investigate computer systems related to a violation of law}:
  This includes cases such as the distribution and storage of illegal
  content, the use of a computer to launch denial of service attacks
  against an individual or organization, and the proliferation of
  harmful malware within an organization.
\item
  \emph{To investigate computer systems for compliance (or a violation
  of) an organization's policies}: This primarily covers internal
  investigations in which a user may act well within the laws of their
  jurisdiction, but may have violated a company policy. For example,
  many companies restrict access to computing systems based on the time
  of day as a security measure. Although a user may normally have
  authorization to access the organization's network, an unexpected
  weekend access to the network may require investigation to determine
  if the activity was done with malicious intent.
\item
  \emph{Responding to a legal (or internal) request for digital
  evidence}: This process is known as e-discovery, in which an
  organization preserves and produces digital information typically as
  part of the discovery process in civil lawsuits. With court approval,
  organizations can be compelled to turn over relevant information to a
  particular lawsuit, including digital documents and digital artifacts
  such as file metadata. (The \emph{Federal Rules of Civil Procedure}
  were amended in December 2006 to include ``electronically stored
  information'' as part of civil discovery
  \cite{withersj.ElectronicallyStoredInformation2006}.)
\end{itemize}

Additionally, it is important to note that the acquisition of forensic
images are only one specific part of the overall digital forensics
process, in which analysts must consider the priority of obtaining
evidence and local requirements for ensuring the admissibility of
digital evidence into a court of law. Again, Conklin et al.~identify
several steps throughout the lifespan of a digital forensic
investigation that are summarized here
\cite{conklinComputerForensics2022}:

\begin{itemize}
\tightlist
\item
  \emph{Identification}: While outside of the scope of forensics itself,
  it is important to determine the scope of the devices that need to be
  analyzed as a result of some incident. Necessarily, it is impossible
  to investigate an incident until an organization can ascertain that
  one has occurred (which, in turn, requires implementing detection and
  protection as described in the NIST Cybersecurity Framework
  \cite{nistNISTCybersecurityFramework2023}). Similarly, if the
  scope of an incident is poorly-defined, the subsequent forensic
  investigation may fail to find significant related evidence, which has
  a direct outcome on the success of the investigation (and often, the
  response of the organization as a whole).
\item
  \emph{Preservation}: After identifying the relevant machines, analysts
  must secure and preserve the physical device itself. With guidance
  from the organization and an analyst's judgment, this often involves
  prioritizing the devices that must be imaged first; for example, a
  critical server may be more likely to cycle out important logs first,
  or an employee's device may only hold important information in
  volatile memory.
\item
  \emph{Collection:} At this point, an analyst must now duplicate the
  digital evidence, in addition to any relevant physical evidence. This
  must be done in a way that passes legal scrutiny; that is to say, it
  must meet requirements for accuracy, reliability, and relevance
  \cite{conklinComputerForensics2022,garfinkelBringingScienceDigital2009}
  (in the United States). In the case of disk imaging, this is typically
  done with a write blocker and cryptographic hashing algorithms to
  ensure a faithful copy has been created of nonvolatile memory.
\item
  \emph{Analysis}: Here, an analyst uses tools and their own knowledge
  to identify significant pieces of information within collected images,
  reconstructing data fragments and drawing conclusions to form a
  coherent timeline and scenario. In many cases, this is done using
  tools such as Sleuth Kit, Autopsy, EnCase, and other domain-specific
  software \cite{jonesInsightDigitalForensics2022}, which often
  parse and automatically identify data of interest on a reconstructed
  file system.
\item
  \emph{Reporting}: After an analyst (or a team of analysts) has
  completed their analysis of the collected data, they must summarize
  and provide a non-technical overview of the conclusions drawn from the
  investigation.
\end{itemize}

Throughout this process, analysts must adhere to well-established norms
for ensuring the admissibility of any digital evidence in court. One
notable example is the use of a chain of custody, which details who has
had access to the evidence. This applies to ``conventional'' and digital
forensics alike, as this documentation asserts that the evidence has not
been tampered with as it is transported between analysts and locations.
Without proper documentation such as a chain of custody, critical
evidence may not meet the legal requirements for admissibility, changing
the outcome of a trial.

Generally speaking, a forensic investigation is only part of a larger
incident response effort; for example, if a data breach occurs, a
digital forensic investigation may be used to determine the scope and
methodology of the breach itself. However, other teams within the
organization may be responsible for patching the devices responsible for
the breach, determining legal consequences, and engaging in other
recovery-related activities. Additionally, a forensic analyst may have
other responsibilities outside of this ``chain''; for example, in
addition to developing a written report, they may be called as an expert
witness to testify and defend conclusions drawn from the investigation
\cite{andersonComparativeStudyTeaching2006,conklinComputerForensics2022,cooperStandardsDigitalForensics2010}.

In the context of research and education, most hands-on or practical
training typically focuses on the analysis and reporting steps, rather
than the rest of the investigation process. That is to say, although the
theory provided in training covers this entire process in detail, it is
relatively rare for students to secure data from an organization, use a
write blocker and other hardware needed to image devices, and then
present the conclusions as an expert witness
\cite{cooperStandardsDigitalForensics2010}. While experience here is
important, it is often impractical; not every university will have a
large forensics lab or a courtroom regularly available. Additionally,
the techniques of analysis and reporting are arguably the most
important; while the other steps can be learned ``on the job''
relatively quickly, all students must be familiar with modern software
and tooling, as well as recent advancements in forensic and
anti-forensic techniques. (As a more concrete example: deep technical
knowledge of operating systems is not required to acquire disk images,
but it is certainly required for effective analysis and reporting.)

It is for this reason that the vast majority of education and research
focuses primarily on the analysis step. Improvements to the distribution
and creation of training material -- such as the use of online labs and
synthesizers -- are largely driven by the fact that no physical lab is
strictly necessary, focusing instead on the software and skillset needed
to effectively analyze images and draw conclusions
\cite{bruecknerAutomatedComputerForensics2008,lawrenceFrameworkDesignWebbased2009}.

\subsection{In research}\label{in-research}

In many cases, the focus of research in digital forensics is on
improving specific processes in the analytic step of a forensic
investigation. This includes the development of analysis techniques for
niche platforms, direct improvements to existing techniques, or novel
methodologies for performing forensic investigations for a particular
platform. For example, recent publications from the Digital Forensic
Research Conference detail a new hashing technique for detecting
similarities in arbitrary files \cite{changFbHashNewSimilarity2019},
an analysis of the NAND memory of the Nintendo 3DS
\cite{pessolanoForensicAnalysisNintendo2019}, and the development of
a new tool for the automated analysis of Android mobile devices
\cite{linAutomatedForensicAnalysis2018}. Necessarily, each of these
publications require forensic datasets; in most cases, these are
obtained or developed manually by the authors, as opposed to using an
existing public or private dataset.

Besides novel contributions to the field, other research focuses on
upholding the quality of the investigation process as new technologies
and tools to analyze datasets are developed. For example, NIST maintains
the \emph{Computer Forensics Tool Testing} (CFTT) program, which
provides a common methodology and test corpora for evaluating specific
tool capabilities
\cite{nationalinstituteofstandardsandtechnologyComputerForensicsTool2017}.
Specific capabilities tested include the ability of a tool to perform
string searching, disk imaging, and the recovery of deleted files. The
project also includes catalogs for forensic algorithms, software, and
tools, as well as the \emph{Computer Forensic Reference Data Sets}
(CFReDS) project, which is a repository of forensic datasets contributed
by NIST and other organizations that is often used by both instructors
and researchers
\cite{nationalinstituteofstandardsandtechnologyCFReDSPortal}.

\subsection{In education}\label{in-education}

Finally, we address the generation of forensic datasets as it applies to
education. Datasets in an educational setting typically focus on
covering a range of techniques and tools, allowing students to practice
applying theoretical concepts learned through lectures and recitations
\cite{adelsteinAutomaticallyCreatingRealistic2005}. Often, this is
done in the context of a specific scenario, such as a user stealing
files from a protected company server, using steganography to hide
information, or recovering a file from volatile memory. Besides the the
specific technical skills covered by these scenarios, the overall goal
of these images is to develop the analytical skills needed to adapt to
new technologies. as tools and the technologies they analyze continue to
evolve \cite{cooperStandardsDigitalForensics2010}. In other words,
students should be familiar with common tools and patterns in digital
forensics, providing a foundation on which more niche techniques can be
learned \cite{lawrenceFrameworkDesignWebbased2009}.

Although the focus of individual forensic images is often to improve the
skills of students in the analysis phase, these images have a direct
impact on the reporting phase as well. Indeed, students must be able to
accurately summarize their conclusions, using their judgement to
describe a scenario and identify topics of interest in a manner that is
consistent with law. Although important, digital forensics is not just
the effective use of tools and techniques; a student should also aim to
be well-rounded in the legal, social, and professional aspects of
digital forensics \cite{andersonComparativeStudyTeaching2006}.

The challenge, however, is providing hands-on labs that comprehensively
support the ideas learned in theoretical courses. Indeed, instructors
face numerous challenges when providing realistic lab material,
including limited access to the necessary software and hardware; the
extensive time needed to develop, distribute and grade labs; and the
high variability between different forensics programs
\cite{adelsteinAutomaticallyCreatingRealistic2005,guptaDigitalForensicsLab2022,lawrenceFrameworkDesignWebbased2009}.
Much of this difficulty in providing high-quality images can be traced
to the issues associated with acquiring images for educational purposes,
as described in the following section.

\section{Real and synthetic
datasets}\label{real-and-synthetic-datasets}

Now that we have identified the purpose of forensic datasets and why
they are needed, we now move to a discussion of how these datasets are
acquired, as well as various issues encountered when using these
datasets. This section focuses on the qualities of real and synthetic
datasets, including some examples. A more detailed survey of existing
datasets is presented in \autoref{existing-forensic-corpora}.

Broadly speaking, there are two types of forensic datasets summarized by
Park \cite{parkTREDEVMPOPCultivating2018}, based on the original
taxonomy described by Garfinkel et al.
\cite{garfinkelBringingScienceDigital2009}. The first is ``real''
data, which is data that was organically created by human beings without
the explicit intent of being used in a forensic investigation. The other
is ``synthetic'' data, which is data generated for specific forensic
purposes, including research and education. Synthetic datasets are
considerably more common than real datasets in education for the reasons
discussed in this section, though their manual creation still poses a
significant problem.

It should also be noted that this section provides a general overview of
forensic datasets as it relates to education and research, providing the
necessary context for forensic synthesizers. However, a more exhaustive
description of dataset construction and usage has been performed by
Horsman and Lyle in \cite{horsmanDatasetConstructionChallenges2021}.

\subsection{Real datasets}\label{real-datasets}

Real data is inherently the most ``realistic'' form of forensic data,
containing extensive background noise as a result of typical computer
usage in addition to a broad variety of software, operating system
artifacts, and other files that might be of interest in a real forensic
investigation. These are most reflective of the scenarios that industry
professionals face, and allow students to train themselves in separating
relevant content from irrelevant content while identifying and
synthesizing details from both a system and a human perspective.

In general, real data can be sourced from far more than just the hard
drives of computers. Other potential and previously used data sources
include:

\begin{itemize}
\tightlist
\item
  social media (which often contains a variety of artifacts with
  revealing metadata and open source information)
  \cite{baggiliDataSourcesAdvancing2015};
\item
  packet sniffers and dedicated forensic tools for IoT devices
  \cite{meffertForensicStateAcquisition2017};
\item
  video game consoles
  \cite{grajedaAvailabilityDatasetsDigital2017,pessolanoForensicAnalysisNintendo2019};
\item
  cloud web server logs \cite{rahmanNewWebForensic2020};
\item
  honeypots \cite{mochForensicImageGenerator2009};
\item
  and the Apache and Python mailing archives
  \cite{grajedaAvailabilityDatasetsDigital2017}.
\end{itemize}

Many more public forensic repositories are described by Grajeda et al.,
though just as many datasets used in research remain private
\cite{grajedaAvailabilityDatasetsDigital2017}. Within reason, with
the billions of internet-connected devices today, there should be no
shortage of sources for real datasets. However, real datasets are the
most challenging to work with in education for a variety of reasons,
particularly the legal and privacy concerns surrounding the use and
distribution of these datasets, as well as their broad scope and lack of
prior analysis.

Necessarily, real datasets were not created to educate students about a
particular technique, and may not adequately supplement an instructor's
material without significant effort. For example, Garfinkel identifies
several requirements for forensic datasets to be suitable for a broad
variety of uses in research and industry
\cite{garfinkelForensicCorporaChallenge2007}. In particular,
Garfinkel notes that in addition to the lack of variety in publicly
available forensic corpora, many real datasets suffer from a lack of
complexity, supplemental annotations, or ongoing maintenance - all of
which are often needed in an educational context.

With respect to availability, some of these datasets are inherently
publicly available; in other cases, their access is restricted to within
an organization. However, just because the underlying data is public
does not necessarily mean that it exists in an aggregate form
immediately suitable for research. Even when aggregated, questions arise
from the use of public datasets for unintended purposes, such as the use
of social media as a source of forensic data; these questions are
already the focus of debate in generative AI, which use publicly
available data to train models for both commercial and research use
\cite{avrahamiOwnershipCreativityGenerative2021,eshraghianHumanOwnershipArtificial2020,rooseAIgeneratedPictureWon2022}.

That said, the primary barriers to the use of real forensic datasets are
privacy and legal issues. For example, between 1998 and 2006, Garfinkel
acquired over 1,000 hard drives through secondary markets, allowing
researchers to perform a range of studies on a large set of real-world
data \cite{garfinkelForensicCorporaChallenge2007}. Although legal
for use by private institutions, the same dataset was barred from use at
the Naval Postgraduate School due to concerns with federal privacy
legislation. Similarly, Grajeda et al.~noted that nearly half of all
digital forensic literature reviewed using a novel dataset did not
publish the dataset due to legal restrictions or NDAs. This was
typically because the datasets were obtained from government agencies,
corporations, and law enforcement agencies, and therefore could not be
publicly released \cite{grajedaAvailabilityDatasetsDigital2017}.
Another reason is that the dataset may contain objectionable or illicit
material, such as licensed software or pornography, preventing its
public distribution.

Certain organizations and researchers have made efforts to make real
datasets accessible to the public through various means. For example,
the emails seized during the Federal Energy Regulatory Commission's
investigation of Enron were purchased by the Massachusetts Institute of
Technology, which anonymized emails and attachments before distributing
the dataset to the public
\cite{yannikosDataCorporaDigital2014,garfinkelForensicCorporaChallenge2007}.
In other cases, institutions have aimed to ensure the data can be made
publicly available ahead of time, such as by requesting that individuals
sign an agreement before any data is collected.

Indeed, some educational institutions do use real-world datasets for
forensic labs. Besides the sources mentioned above, students may also
opt to image their own devices or the device of a friend with
permission; these approaches are mentioned in older works
\cite{andersonComparativeStudyTeaching2006,mochForensicImageGenerator2009},
prior to the advent of online platforms such as CFReDS that provided
easier access to education-focused datasets. Naturally, this method of
acquiring real-world images suffers from other issues as well; students
know exactly what they will find on their own computer, and although
individuals may consent to the use of their images for educational
purposes, there remains the risk of highly personal data being leaked
\cite{garfinkelBringingScienceDigital2009}. Simultaneously, they
often lack the ground truth or annotations needed to assert that a
student has found everything of interest.

\subsection{Overview of synthetic
datasets}\label{overview-of-synthetic-datasets}

Because of the various issues associated with real-world data, many
instructors opt to use synthetic data (or ``manually created data'')
instead, in which the scenario and data are artificially generated based
on some predetermined procedure. That is to say, the data exists with
the explicit intent of being used in research or education.

From an educational perspective, an ideal dataset should accurately
reflect the problems and challenges faced by industry professionals
while avoiding the problems of real-world datasets. In turn, synthetic
datasets are often created with the express intent of allowing students
to explore forensic techniques in realistic scenarios while avoiding the
privacy and legal restrictions of genuine real-world scenarios. (The
same largely holds true in research, as well.)

Again, as described by Park and Garfinkel et al., synthetic data can be
categorized into two distinct groups
\cite{garfinkelBringingScienceDigital2009,parkTREDEVMPOPCultivating2018}:

\begin{itemize}
\tightlist
\item
  \emph{Synthetic test data}, which refers to forensic corpora that have
  been developed to test specific features in a group of tools. These
  are well-annotated datasets with extensive reference information and
  ground truth data, and are typically used to assert that a tool is
  able to analyze and identify data of interest. One example discussed
  so far is the Computer Forensics Tool Testing program administered by
  NIST
  \cite{nationalinstituteofstandardsandtechnologyComputerForensicsTool2017}.
\item
  \emph{Synthetic realistic data}, which is designed to mimic a
  situation that a forensic examiner might encounter in a real-world
  investigation. These are typically much more applicable to an
  educational context than test data, though test data can be used as
  educational material for learning new tools or very specific
  techniques.
\end{itemize}

Synthetic realistic data varies greatly in scope. Simple realistic data
might form a realistic emulation of how a user might use a particular
piece of software or anti-forensic technique, which creates forensic
artifacts directly associated with these actions. In contrast, realistic
data could form a full simulation of a scenario, whether based on real
events (such as the one developed by Moch and Freiling based on the Arno
Funke blackmail case in Germany
\cite{mochForensicImageGenerator2009}) or on common industry themes,
such as the NIST CFReDS data leakage case
\cite{nationalinstituteofstandardsandtechnologyCFReDSDataLeakage}.

\subsection{Motivation for synthetic
datasets}\label{motivation-for-synthetic-datasets}

Besides the issues mentioned in \autoref{overview-of-synthetic-datasets}, one major motivation for the development of synthetic
datasets is the need to fill various gaps in publicly available corpora.
The survey done by Grajeda et al.~found that some researchers created
new datasets not because of legal or privacy hurdles, but simply because
there was no available dataset for their needs
\cite{grajedaAvailabilityDatasetsDigital2017}. The researchers
interviewed in the survey also recognized the value of publishing their
datasets, but were hindered by a lack of available resources to maintain
or distribute the datasets. Other researchers were prevented from doing
so due to a non-disclosure agreement. The survey concluded that there
was no preference for building new datasets from scratch in research,
though there was clear agreement that private datasets made it more
difficult for researchers to reproduce results; in general, publishing
datasets contributed to the advancement of the field.

Similar sentiment also exists in an educational context, though the
motivation for creating new images from scratch differs from that of
research. For example, the direct reuse of existing datasets is often
undesirable, as it is often the case that answer keys and walkthroughs
have already been published online
\cite{woodsCreatingRealisticCorpora2011}; additionally, students in
the same class can simply replicate the exact methodology of other
classmates to come to the same conclusions without needing to perform
meaningful analysis. Even in labs where students' actions can be
monitored to provide insight into their methodology and possible
mistakes, such as in the \emph{CYDEST} platform developed by ATC-NY
\cite{bruecknerAutomatedComputerForensics2008}, it is still possible
to exactly copy the methodology of another student if the labs
themselves do not differ.

Additionally, many datasets that may be reflective of real-world
scenarios are often unsuitable for academic purposes. For example, the
forensic challenges of the Honeynet Project, DFRWS, and DC3 are suitable
training material for experienced forensic analysts, but are often too
difficult for students to solve
\cite{woodsCreatingRealisticCorpora2011}. As mentioned before, many
real-world datasets lack ``ground truth'' or annotated information that
is suitable for determining what the contents of an image are,
preventing the development of an answer key and qualitative grading of
students' work based on what they find and report. The same is true of
certain synthetic datasets, which may have been developed without a
focus on detailed documentation.

Clearly, there is a need for images that allow students to explore the
techniques and tools used by forensic analysts in industry in a
realistic setting. Preferably, these images should be built in such a
way that they challenge students to explore the same technique, but have
distinguishable differences to dissuade cheating (for example, different
files of interest may be placed, the specific disk sector written to may
vary, or relevant metadata may be changed). Simultaneously, instructors
need to know what the contents of the image are (including these
variations), so that they can judge the accuracy of students' findings
and determine which images are suitable hands-on material to complement
lecture material.

As a result, many instructors ultimately turn to manually developing
realistic forensic scenarios that students can analyze, much like
researchers. Some of these efforts have lead to publicly available
datasets that encompass realistic scenarios, such as the datasets and
scenario published by Woods et al.~as a direct response to the issues
noted above \cite{woodsCreatingRealisticCorpora2011}.

\subsection{Challenges in developing synthetic
datasets}\label{challenges-in-developing-synthetic-datasets}

However, fulfilling all of the ideal characteristics of a forensic
dataset is difficult, especially if an instructor wants to create
variations of the same forensic scenario, multiplied by the number of
scenarios used throughout the course. There are three distinct issues
associated with the manual development of forensic datasets: the
significant amount of time involved in their creation, the inherent
non-determinism of human creation, and the lack of realistic background
usage.

The primary issue associated with manual development is the extensive
amount of work and time that must be put into not only planning and
developing the scenario, but also executing it. Virtually all prior
works recognize the time-consuming process involved in the manual
development of disk images suitable for education
\cite{adelsteinAutomaticallyCreatingRealistic2005,gobelForTraceHolisticForensic2022,guptaDigitalForensicsLab2022,mochForensicImageGenerator2009,russellForensicImageDescription2012,scanlonEviPlantEfficientDigital2017,woodsCreatingRealisticCorpora2011},
motivating research into possible solutions to streamlining their
development.

Depending on the artifacts involved in the scenario, it can also be
difficult to ``falsify'' metadata as a means of speeding development up.
More precisely, it can be challenging to produce the artifacts
associated with a scenario over a shorter period of time than is
suggested by the metadata attached to the artifacts. For example, it is
possible to directly change the time in virtualization software (the
approach often used by synthesizers
\cite{gobelForTraceHolisticForensic2022,mochForensicImageGenerator2009})
to allow time to pass without leaving any artifacts that suggest that
the system time has been tampered with. However, making this consistent
with online services can be challenging, since artifacts such as emails
or cached external websites would be based on ``real'' time; additional
work would then be needed to synchronize online content outside of the
control of the instructor. While there are workarounds that could be
considered -- such as saving the contents of various external websites
over a period of time and then ``replaying'' them at the time of
scenario development -- these are not trivial.

Finally, the long development period comes with the challenge of
addressing mistakes. In the event a mistake is made relative to the
desired scenario, instructors have several options. They may be able to
simply log the deviation and move on, or correct it after the fact by
directly editing the virtual hard drive after image creation. This was
the approach taken by Woods et al.~in which at least one researcher
logged into their personal email account while developing a fictional
scenario; the images were later scanned for personal identifiers and
stripped as needed \cite{woodsCreatingRealisticCorpora2011}.
However, consider a scenario in which it is absolutely essential that
certain files are deleted and written in a particular order, perhaps to
ensure the operating system behaves in a specific manner. In this case -
and possibly many others - the only option may be to start the process
of developing the image from the beginning.

The nondeterminism associated with human creation is a double-edged
sword. Given the same general scenario, this allows multiple images to
differ very slightly in nature (a desirable feature as described earlier
in this section); as described by Woods et al., the researchers all
acted out events in a pre-defined timeline, but did not do so at the
exact same time or in the exact same way
\cite{woodsCreatingRealisticCorpora2011}. This provides the
variability needed to prevent students from directly copying answers,
but allows students to explore the same techniques at a high level.

Simultaneously, however, this also adds a degree of uncertainty within
the images. Various artifacts or results on the images may be the result
of some confounding factor in the procedure; for example, suppose that
two researchers follow the same procedure to delete and overwrite a file
with the goal of making an image that allows students to explore file
carving in slack space. Due to variations in operating systems, drivers,
and other related software, it may be the case that the recoverable
contents in the slack space differ significantly between the two
researchers. Similarly, consider a case in which two researchers follow
the same procedure to develop volatile memory captures on a Linux
machine. It might be the case that the paging daemon evicts relevant
data for one image but not the other, causing significant differences.
Without careful preparation in creating a consistent environment for
development and testing, these differences might arise without a clear
reason. Where possible, sources of uncertainty should be minimized or at
least known to the scenario developer.

Finally, an inherent limitation of the long development time needed to
produce these images is the difficulty in populating realistic
background noise, in which a user performs normal activities on the
computer that are completely unrelated to the meaningful components of
the scenario. This must be done in a way that is realistic to the
scenario while avoiding any activity that could personally identify the
researcher or instructor. Although some of this background noise can be
automated - such as by scripting the process of browsing to websites or
sending emails - the fact remains that many realistic ``background''
activities, such as working in software such as Microsoft Office,
MATLAB, and SolidWorks, are not easy to perform without a human.
Furthermore, some of this background noise may comprise a significant
portion of the day, such as if these activities comprise a user's day
job in the scenario. Naturally, it is not the case that most instructors
or researchers have the time available to spend several hours each day
generating this information, much less over multiple variations of
multiple scenarios.

\section{Research objectives}\label{research-objectives}

At this point, we have clearly established a need for a more streamlined
method of developing scenarios for research and education, while still
providing the variability and content needed for images to be
interesting and adequately train students in forensic methodologies.
Additionally, these scenarios need to provide ground truth data to
determine the artifacts contained in each image, allowing instructors to
identify what should be contained in students' reports of the scenario.
Some research has been done into developing image synthesizers, which
aim to automate part or all of this work, as described in \autoref{chapter-two}.

This thesis aims to directly improve upon the foundations provided by
prior synthesizers, with the explicit goal of achieving feature parity
with all existing synthesizers. In particular, this thesis will describe
the components and architecture necessary to build an effective forensic
synthesizer for research and education with the following elements:

\begin{itemize}
\tightlist
\item
  The ability to create and export disk images, network captures, and
  volatile memory captures through arbitrary means;
\item
  the ability to generate forensic artifacts in a modular manner, both
  with and without operating system virtualization;
\item
  the ability to log every action in a standardized format, the
  Cyber-investigation Analysis Standard Expression (CASE)
  \cite{caseyAdvancingCoordinatedCyberinvestigations2017}, to
  address existing standardization concerns
  \cite{horsmanDatasetConstructionChallenges2021};
\item
  and the use of modern Python libraries, standards, and practices,
  promoting future development by reducing the overall complexity of the
  framework and abstracting specific concepts where possible.
\end{itemize}

Furthermore, this thesis aims to leverage recent advancements in
generative AI to streamline the development of full forensic scenarios,
as well as individual forensic artifacts to be added to a larger
forensic dataset. Finally, this thesis will evaluate the viability of
this framework in an actual classroom setting.

\section{Contribution}\label{contribution}

This thesis contributes the \emph{automated kinetic framework}, or AKF,
a modernized synthesizer framework that aims to provide the foundation
of a larger forensic dataset ecosystem. This framework can be used to
not only vastly reduce the time spent developing new datasets for
research and education, but also improve the discoverability of both
AKF-generated and non-AKF-generated datasets. Additionally, by focusing
on the long-term viability of AKF through its modular architecture, the
hope is that educators and researchers will have greater variety in the
datasets available to them, even as new developments and advancements in
technology occur.
