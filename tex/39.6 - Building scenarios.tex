This chapter addresses the modules responsible for allowing users to
invoke the framework, both through a standard Python script and through
a high-level YAML file. It also addresses the generative AI modules that
can assist a user in building a scenario, as depicted in the partial
architectural diagram below.

!\textbf{39.6 - Building scenarios 2025-02-08 17.23.40.excalidraw}

At this point, we have provided the implementations for automating
artifact generation in a near-deterministic manner with comprehensive
logging and reporting. However, there is still the challenge of exposing
this functionality in a user-friendly manner. Furthermore, even if the
\emph{process} of placing artifacts or performing actions can be
simplified, the challenge of deciding what actions to perform still
remains.

It is important to note that it is still largely the responsibility of
instructors to provide the actual background noise to populate the
images with. Although the high-level languages provided by many of these
frameworks make it easy to place files at desired locations or visit
websites that are part of a scenario, these must all be defined and
created ahead of time.

This chapter addresses the challenges of creating background noise and
providing simple APIs for complex GUI-driven applications. More
precisely, we address two questions -- how do we invoke AKF's automation
systems, and how does AKF assist a user in building a scenario? Here, we
explore AKF's imperative and declarative APIs, as well as the viability
of using large language models (LLMs) to assist in building individual
files and complete scenario descriptions.

\subsection{Background}\label{background}

We begin by analyzing how synthesizers accept user input to decide how
to operate, as well as what specific forms of data are accepted by these
synthesizers. In particular, we address the following questions:

\begin{itemize}
\item
  How do users define the high-level ``scenario'', or sequence of
  operations that the synthesizer should take to form the image?
\item
  How do users pass data, such as images to be placed or emails to be
  sent, into the synthesizer?
\item
  How do users pass data, such as images to be placed or emails to be
  sent, into the synthesizer? or many of the frameworks created in the
  last decade, users typically define scenarios by using a Python
  library to interact with the framework, setting up the virtualized
  environment and perform high-level actions on the environment. This
  abstracts the underlying calls to the virtualized environment away
  from the user. This code-based approach represents an
  \emph{imperative} strategy to scenario creation, where the user
  describes how the image should be created by describing the exact
  order and methodology by which actions should be taken.
\end{itemize}

It is worth noting that, like many automation frameworks such as
Playwright \cite{MicrosoftPlaywrightpython2025}, the language used
to interact with the synthesizer's API does not need to match the
language used to implement the synthesizer itself (although this is
often the case). For example, Playwright itself is implemented in
TypeScript, and therefore began with a Node.JS API. Today, Playwright
provides APIs in Python, Java, and C\#.

In contrast, custom scenario formats provided by \emph{D-FET}
\cite{williamCloudbasedDigitalForensics2011}, \emph{SFX}
\cite{russellForensicImageDescription2012}, and Yannikos
\cite{yannikosDataCorporaDigital2014} follow a \emph{declarative}
strategy. Here, a custom high-level language describes what the final
state of the image should be. For example, consider the following
example \emph{SFX} code taken from
\cite{russellForensicImageDescription2012}, in which a partition is
created on which a 64-bit version of Windows 7 is installed and a user
called ``Gordon'' uses Firefox to browse to the internet:

REPLACE CODEBLOCK HERE

The same might be partially expressed in \emph{ForTrace} as the
following, excluding additional overhead for ground truth generation:

REPLACE CODEBLOCK HERE

Note that \emph{hystck} - and by extension, \emph{ForTrace} - also
allows users to express scenarios as YAML files, such that both a
declarative and an imperative approach to defining scenarios is
available. This reduces the technical difficulty of using the framework,
while still allowing users experienced with both Python and the
framework to perform lower-level customization as needed. This
highlights the fact that both declarative and imperative approaches can
be used simultaneously; in particular, it demonstrates that
declarative-to-imperative translators can be written to support
arbitrary declarative languages, such as those of both SFX and ForTrace.
While not explored in this thesis, it is also worth noting the GUI-based
interfaces provided by \textbf{Yannikos et al.} and \textbf{ForGe} for
building scenarios.

\subsection{Setup and imperative
usage}\label{setup-and-imperative-usage}

Like many of its predecessors, AKF implements its functionality and
exposes its API in the same language -- Python 3. There are numerous
advantages to a Python-based API; besides the low difficulty of setting
up and using Python, its rich ecosystem allows scenarios to be extended
through the use of other libraries from the Python ecosystem. For
example, if a user wanted to conditionally execute certain parts of a
scenario by testing if a particular service is currently online, a user
could use the \emph{Requests} library \cite{Requests31Documentation}
to issue an HTTP request out-of-band before doing the same using
synthesizer-provided routines in the virtualized environment.

\subsection{Declarative usage}\label{declarative-usage}

\subsubsection{Overview}\label{overview}

\subsubsection{The AKF declarative
syntax}\label{the-akf-declarative-syntax}

\subsection{Using generative AI for individual
artifacts}\label{using-generative-ai-for-individual-artifacts}

As it currently stands, users of synthesizers must still perform a
significant amount of work towards generating the artifacts to be
planted. While the process of generating an image based on some
predefined scenario has been streamlined through existing synthesizers,
users still need to define all of the data that they want to plant. For
example:

\begin{itemize}
\item
  If a user wants to place 100 photos on the drive to simulate real
  usage, the user needs to pass in 100 realistic images;
\item
  If a user wants to simulate an email or other online conversation, the
  user needs to pass in the entirety of the conversation to simulate;
\item
  If a user wants to generate ``proprietary'' documents to emulate some
  form of corporate sabotage, the user would need to generate a variety
  of Microsoft Office, PDF, or other files in these formats ahead of
  time.
\item
  If a user wants to generate ``proprietary'' documents to emulate some
  form of corporate sabotage, the user would need to generate a variety
  of Microsoft Office, PDF, or other files in these formats ahead of
  time. he net result is that although creating images for the purposes
  of tool validation and research can be accomplished with existing
  frameworks, creating realistic images that are more reflective of
  real-world scenarios that a forensic analyst might encounter still
  requires extensive work. While true that images should often be small
  enough in a classroom setting to allow the student to explore a single
  specific technique, real-world scenarios encountered by analysts are
  typically not limited by time or size. An analyst might have to deal
  with a drive used over the course of a decade to store many
  photographs and send many messages. Such scenarios are valuable
  training material for courses that encapsulate a long period of
  forensic study, allowing a student to apply many different techniques
  in reconstructing a large-scale scenario.
\end{itemize}

With recent advancements in generative AI, popularized by services such
as Midjourney and ChatGPT, it is now significantly easier to generate
realistic images and text content from short, high-level descriptions.
Additionally, various services exist for creating realistic audio and
video files that emulate a particular person's voice or facial
movements; these can be used to generate additional scenario content of
interest, especially if the scenario is based on a real-world event.

It holds that generative AI can be used to quickly populate forensic
datasets with realistic conversations and images consistent with an
arbitrary scenario. For example, a corporate espionage case could be
built by providing a large language model such as ChatGPT with prompts
to describe complex machinery in both a technical writing and a
conversational style. Simultaneously, similar prompts can be passed into
an image synthesizer such as Midjourney to produce related images. The
images and text produced can then be used to create documents describing
an unreleased product of high value, providing a pipeline through which
significant artifacts can be planted onto a forensic image.

This idea can be extended further by training models on specific
datasets; for example, if an instructor wished to create a fictional
scenario in which a user frequently interacts with users of a particular
online community, a large language model could be trained on available
conversations to provide a degree of realism to the scenario. However,
as mentioned before, this faces the challenges of ownership, privacy,
and legality behind works derived from publicly available information
that was (likely) not published with the expectation of its usage in an
AI model.

It is important to note that the inclusion of generative AI into
synthesizers does not necessarily require deep integration with the
framework itself. Many existing frameworks could be extended to use
documents, images, or other data sourced from generative AI instead of
user-defined files without the need to change the architecture of the
framework. However, as advancements in AI continue, it may make sense to
directly integrate AI-driven actions into synthesizers. For example,
there may come a time in which synthesizers can be provided natural
language prompts (such as ``Open Firefox and browse to news-related
websites'') that directly lead to the generation of relevant artifacts,
without the need to explicitly program the process of browsing to a
website in advance.

\subsection*{Using LLMs for high-level
scenarios}\label{using-llms-for-high-level-scenarios}
\addcontentsline{toc}{subsection}{Using LLMs for high-level
scenarios}
