We begin with a literature review of two topics: the publicly available
forensic datasets that currently exist and the contributions of prior
synthesizers. In particular, we describe the datasets that currently
exist for digital forensics in greater detail, as well as gaps
identified in these datasets by various authors. This will provide
context for both the need of synthesizers, as well as the specific gaps
that these synthesizers have gradually filled.

\section{Existing forensic
corpora}\label{existing-forensic-corpora}

Forensic datasets have been available for public use (or by request)
since the early days of the digital forensics field, though their
sourcing and qualities have changed greatly over time. This is explored
in far greater detail by Grajeda et al., who performed a survey of over
700 research articles to identify the various datasets used by the
digital forensics field over time. However, it is still important to
recognize certain datasets identified by prior literature as it is
relevant to synthesizers.

Early datasets were largely derived from real sources, whether made
available to the public or otherwise. The earliest collections of real
datasets include the used hard drives collected by Garfinkel from 1998
to 2006 and the Enron email corpus obtained during the federal
investigation of Enron \cite{garfinkelForensicCorporaChallenge2007}.
Other early real datasets identified by Grajeda et al.~include the
public Apache mailing archive, the Reuters news corpora, and various
facial recognition collections such as the MORPH corpus, all of which
were made in the early to mid-2000s
\cite{yannikosDataCorporaDigital2014,grajedaAvailabilityDatasetsDigital2017}.

There were also a variety of synthetic datasets constructed during this
early period. This includes the network captures obtained from simulated
attacks conducted by the MIT Lincoln Laboratory from 1998 to 2000
\cite{garfinkelForensicCorporaChallenge2007}, as well as standalone
datasets used for tool validation developed as part of the early CFTT
program developed by NIST. Notably, some synthetic datasets during this
period were generated as part of challenges, such as those produced for
DFRWS conferences, which are noted as being difficult for students to
solve \cite{woodsCreatingRealisticCorpora2011}.

The variety of forensic datasets increased considerably towards the late
2000s, which can be credited to both the overall growth of the field
(including the broader field of incident response) and computing as a
whole. Various notable sources described by Grajeda et al.~include
malware samples discovered ``in the wild'', large natural language
collections from multiple languages, and file-specific datasets such as
collections of Microsoft Office files. It was also during this time that
non-disk datasets began to be more prevalent, such as volatile memory
dumps and network captures. Although not explored by this thesis in
great detail, mobile datasets -- such as smartphone disk images, mobile
malware and applications, and SIM card images -- grew in prevalence as
well.

Many of these datasets were not maintained as part of a larger
collection of datasets with the explicit intent of providing them for
digital forensics research. This began to change towards the late 2000s;
Garfinkel's collection would eventually evolve into the Real Data
Corpus, growing to 30 terabytes by 2013
\cite{garfinkelBringingScienceDigital2009a,yannikosDataCorporaDigital2014}.
The collection then included hard disk images, flash drive images, and a
variety of optical discs sourced from real-world usage, requiring
institutional review board approval to use. This collection would
eventually be part of the Digital Corpora platform, which includes a set
of purely synthetic datasets. Separately, NIST began developing the CFTT
and CFReDS projects, both of which provide forensic datasets for a
variety of purposes. Digital Corpora, CFTT, and CFReDS are actively
maintained as of writing.

There are other datasets that are rather unique in nature, and are
maintained as part of a larger niche collection. Collections of malware
samples have grown significantly in size, in part because of the modern
threat intelligence ecosystem with platforms such as VirusTotal. There
exists datasets focusing on the dark web, such as those operating on the
Tor network, including ``black market'' sites on which illegal goods are
bought and sold. Finally, there exist network captures from various
novel sources, including various iterations of the Collegiate Cyber
Defense Competition and various network captures from university IT
departments \cite{grajedaAvailabilityDatasetsDigital2017}.

These datasets span a wide range of technologies -- in particular,
versions of the same technology -- that require different methodologies
to effectively analyze. In particular, there is the challenge of making
datasets available that reflect current advancements in technology, such
as new operating system versions or new applications. For example,
instant messaging applications have changed considerably over the
history of the field, ranging from MSN Messenger in the early 2000s to
Skype, Discord, Telegram, Signal, Slack, and more. Artifacts from each
of these applications must be handled differently, even if the
underlying service provided is largely the same. Similarly, the
strategies for analyzing Windows artifacts have changed significantly
from version to version, as new registry keys become relevant in
analyzing applications while others become unused.

It is this gradual ``aging'' of datasets, in which their relevance
degrades over time, that contributes to the continuous need for new
datasets. Furthermore, there have been a broad variety of datasets that
have proved to be relevant in digital forensics research, even if not
typical or immediately evident. Indeed, this is one of the reasons
identified by Grajeda et al.~for the manual development of new datasets
by research authors; it would often be the case that a modern dataset
simply did not exist for their needs. The other motivation is that the
dataset used was never made public, either due to the author not having
the resources to distribute the dataset themselves, or because there
were legal or privacy concerns.

How are these datasets relevant to the development of synthesizers?
Clearly, there has been a broad variety of datasets that have proved to
be relevant in digital forensics research, even if not immediately
evident. It should be the goal, then, that any effort to streamline the
develop of forensic datasets be able to cover as many use cases as
possible. To do so, a synthesizer should aim to achieve two separate
goals:

\begin{itemize}
\tightlist
\item
  A synthesizer should be able to replicate the features of existing
  datasets, provided that the underlying technologies are still
  available; and
\item
  A synthesizer should be able to account for developments in operating
  systems, applications, or other technologies, without requiring
  significant changes to the underlying architecture.
\end{itemize}

Indeed, the synthesizers described in the following section have
explicitly addressed these two concerns. For example, many of these
synthesizers have focused on implementing features that reflect the
qualities of real-world datasets. This includes the ability to execute
malware samples in a virtualized environment, send emails to arbitrary
email servers, and insert data on removable drives -- all of which
generate forensic artifacts that have been used as key datasets in
forensics-related research as shown here. Similarly, modern synthesizers
are capable of larger generating disk images, network captures, and
volatile memory dumps (in addition to extracting specific artifacts,
such as application-specific files), which are also reflective of the
focus of existing datasets. The gradual growth in the ability of
synthesizers to generate various artifacts can be seen in the specific
contributions made by each synthesizer in \autoref{analysis-of-existing-synthesizers}.

Of note is the generation of similar datasets that typically involve
significant human interaction, such as public email distribution lists,
photographs of human faces, and other transcripts of conversations.
While this has not been explored in significant detail by prior
synthesizers, it is explored as part of the generative AI work done as
part of AKF in \autoref{chapter-six}.

The second issue, in which synthesizers must be extensible in such a way
that they can support new applications, has been approached in several
ways. This is described in greater detail in \autoref{chapter-three}, but has been a major design consideration in the
development of most synthesizers.

\section{Analysis of existing
synthesizers}\label{analysis-of-existing-synthesizers}

Numerous frameworks have been built over the last two decades that aim
to significantly reduce the effort involved in creating synthetic images
from scratch by automating various application- and OS-specific actions
at specified locations and times. The functionality and availability of
the frameworks described in literature have varied considerably over
time, but the goal of these frameworks has remained largely consistent:
they all aim to provide a rapid method for instructors to develop
forensic labs for students. Note that low-level studies are described in
later, more relevant sections throughout this thesis for clarity, rather
than providing detailed analyses as part of this chapter. Before
analyzing and adapting the low-level implementations of prior
synthesizers, it is best to first contextualize their broader
contributions.

The first identified effort to streamline the creation of forensic labs
through a high-level language was published by Adelstein et al.~in 2005
through the development of \emph{FALCON, the Framework of Laboratory
Exercises Conducted Over Networks}
\cite{adelsteinAutomaticallyCreatingRealistic2005}. It proposed an
architecture for not only the automatic creation of lab exercises, but
also their deployment to a virtual lab and the evaluation of students'
actions compared to the ground truth. Very few details of the
implementation are provided, though some examples of its usage are
provided. Although not primarily a synthesizer, a related work was in
the development of \emph{CYDEST, the CYber DEfenSe Trainer}
\cite{bruecknerAutomatedComputerForensics2008}, which similarly
provided complex virtualized lab environments to students via the
internet. Again, limited implementation details are provided, and it is
unclear if either \emph{FALCON} or \emph{CYDEST} are publicly available
or actively maintained.

\emph{Forensig2}, described by Moch and Freiling in 2009 and revisited
in 2012, appears to be the first detailed description of an image
synthesizer
\cite{mochForensicImageGenerator2009,mochEvaluatingForensicImage2012}.
Operators define scenarios through a Python 2 library provided by the
authors that abstracts various virtual machine operations (such as the
formatting of disks, the creation of partitions, and the copying of
files from the host to the virtual machine). Actions are performed live
through a Qemu-based VM, with the result being a ground truth report and
(effectively) an image to be analyzed by students. The source code for
\emph{Forensig2} is not currently maintained and appears to be
unavailable.

The \emph{Digital Forensic Evaluation Test (D-FET)} platform described
by William et al.~in 2011 provides a custom scripting language for users
to define system- and user-level actions
\cite{williamCloudbasedDigitalForensics2011}. Similar to
\emph{FALCON} and \emph{CYDEST}, it provided virtualized labs through
VMware ESXi. Unlike most other frameworks, it was primarily built to
provide a platform to efficiently evaluate digital forensic tools on a
scalable infrastructure, rather than streamline the process of building
and distributing labs for instructors (though it is mentioned that the
infrastructure was also used to host student labs). It is unclear if a
public implementation is available.

Russell et al.~define an XML specification, the \emph{Summarized
Forensic XML (SFX)} language, to describe scenarios as a sequence of
various actions on various partitions of a disk
\cite{russellForensicImageDescription2012}. This can be passed into
an interpreter to produce an image that can be analyzed by students.
Besides high-level operations such as disk partitioning and file
copying, it also provides Windows- and Linux-specific routines for
minimizing partition sizes with the goal of reducing the ``effective''
size of irrelevant files associated with the OS. It also briefly
describes approaches for updating web browsers and the Windows registry
to reflect certain actions. Limited implementation details are provided,
and it is unclear if a public implementation is available.

Yannikos et al.~define a framework for describing scenarios as a series
of Markov chains, allowing users to graphically define scenarios through
a graph view in which nodes and relationships can be easily edited
\cite{yannikosDataCorporaDigital2014}. Its design makes it
well-suited for quickly developing variations of the same scenario from
a large dataset of available artifacts to be placed on the image, though
it is not clear what application- or OS-specific routines are provided.

The remaining frameworks are all largely similar to each other in that
they are all built in Python, providing users with a Python library to
define and generate scenarios. Various functions in the library provide
abstractions to complex actions, such as Facebook browser activity or
sending emails on the Thunderbird application. Additionally, each of
these produce some form of detailed output that can be used as answer
keys or ground truth in an instructional setting. A brief summary of the
notable aspects of each of these frameworks relative to each other is
provided below:

\begin{itemize}
\tightlist
\item
  \emph{ForGeOSI} \cite{maxfraggMaxfraggForGeOSI2023} introduced the
  use of the VirtualBox SDK to automate various operations, forming the
  basis for much of the work done as part of VMPOP.
\item
  \emph{ForGe} \cite{vistiAutomaticCreationComputer2015} is
  specifically designed to generate NTFS and FAT32 images with a
  particular focus on placing data by maintaining and serializing custom
  data structures for supported filesystems.
\item
  \emph{EviPlant} \cite{scanlonEviPlantEfficientDigital2017}
  encompasses a novel method for the distribution of generated images in
  an educational context, whereby ``evidence packages'' are distributed
  to students to apply to a base image file. Then, an OS-native
  injection tool plants relevant artifacts according to the evidence
  package. Since the base image must only be downloaded once, each
  scenario is significantly smaller than if distributed as standalone
  images.
\item
  \emph{VMPOP} \cite{parkTREDEVMPOPCultivating2018} provides an
  architecture and routines for relatively elaborate VirtualBox control
  (such as attaching USB devices and starting video captures) in
  addition to a variety of OS-specific commands. Provided Windows
  routines include the ability to create restore points, install
  programs, map network drives, set registry values, and more. Although
  the architecture as a whole is platform-independent, the provided
  implementations operate with VirtualBox and Windows.
\item
  \emph{hystck} \cite{gobelNovelApproachGenerating2020} provides
  routines for automating OS- and application-specific commands through
  YAML configuration files (passed through a generator) and/or Python
  scripts (for low-level control). Similar to \emph{EviPlant}, it uses
  ``differential'' images to allow lightweight scenarios to be
  distributed relative to ``template'' images, and produces both network
  and disk captures.
\item
  \emph{ForTrace} \cite{gobelForTraceHolisticForensic2022} is the
  most recently developed synthesizer, which directly builds upon
  \emph{hystck} by providing volatile memory captures (alongside disk
  and network captures) in addition to various other new features (such
  as the ability to execute PowerShell scripts to create Windows
  artifacts), with a focus on a modular architecture. A variant focusing
  on Android artifact generation was developed in 2024
  \cite{demmelDataSynthesisGoing2024}.
\end{itemize}

Clearly, a considerable amount of work has been done in exploring ways
to streamline the process of developing images, although the
availability and functionality of each framework varies greatly.
Notably, with the exception of \emph{ForTrace}, none of these works are
direct extensions of past works, implying that at a fundamental level,
it was necessary for frameworks to be developed from scratch to support
the framework authors' needs. This is not directly stated in any of
these works with the exception of \emph{hystck}
\cite{gobelNovelApproachGenerating2020}, though it can be reasonably
concluded that the lack of maturity, availability, and maintenance of
prior works was a contributing factor to the independent development of
the other frameworks. This will be addressed in more detail in
\autoref{chapter-three}.
