See **\_Chapters\_v2**

The UNR filing guidelines (includes formatting and other fun stuff):
https://www.unr.edu/grad/current-students/filing-guidelines

Global notes

\begin{itemize}
\tightlist
\item
  Does the inclusion of the architectural diagram at the start of every
  chapter work, or should it just be included? Is it good/does it make
  sense/does it fit on the page? Do I need to describe it in greater
  detail, if I'm going to bother introducing it at all?
\item
  Does the introduction of each chapter work? Or is it too sudden?
\end{itemize}

Things that can be done Right Now:

\begin{itemize}
\tightlist
\item
  5.2 (core outputs)
\item
  8.2 (future work), things that are out of scope that we
  \emph{definitely} know won't happen and therefore don't have to wait
  until we're done with implementation to figure out what didn't get
  done
\item
  9 (conclusion), it's just like one paragraph lmao
\end{itemize}

\begin{center}\rule{0.5\linewidth}{0.5pt}\end{center}

\textbf{39.0 - Abstract} Done (may require some light editing if we
don't fulfill all the promises made in the abstract)

\autoref{chapter-one} Done

\autoref{chapter-two} Done

\begin{itemize}
\tightlist
\item[$\boxtimes$]
  \#task 2.1 - Write this entire section, probably mostly from
  \cite{grajedaAvailabilityDatasetsDigital2017} 📅 2025-02-08 ✅
  2025-02-08
\end{itemize}

\autoref{chapter-three} Done (unless the architecture
diagram needs to be reworked, either to fit in better with the text or
to actually move stuff around in the diagram itself)

\begin{itemize}
\tightlist
\item[$\boxtimes$]
  \#task 3 - Consider incorporating concepts explicitly from
  \cite{horsmanDatasetConstructionChallenges2021} 📅 2025-02-09 ✅
  2025-02-08
\item[$\boxtimes$]
  \#task 3.2 - Update the architecture diagram to look nicer, and remove
  all ``notes'' that are supposed to just be for me 📅 2025-02-09 ✅
  2025-02-08
\item[$\boxtimes$]
  \#task 3.2 - Consider making the connection between the ``three
  distinct concepts'' and the diagram to be more clear, most likely by
  editing or adding a simpler diagram that provides a closer 1-1
  connection between those components 📅 2025-02-09 ✅ 2025-02-08

  \begin{itemize}
  \tightlist
  \item
    that is, you'd have one big scary diagram, and then a simplified
    version of the same diagram
  \end{itemize}
\end{itemize}

\autoref{chapter-four}

\begin{itemize}
\tightlist
\item[$\boxtimes$]
  \#task 4 - Consider adding an inset diagram from the complete
  architectural diagram, and then briefly explaining what parts of the
  diagram correspond to which sections below; also consider moving the
  content into 4.1 📅 2025-02-08 ✅ 2025-02-08
\item
  4.2 - after more agentless generation is implemented, it needs to be
  discussed in greater detail in the context of AKF
\item
  4.3 - add any new application-specific information to the table

  \begin{itemize}
  \tightlist
  \item[$\square$]
    \#task by extension: achieve feature party with what's discussed in
    ForTrace page 11, at least as close as possible 📅 2025-02-09
  \end{itemize}
\item[$\square$]
  \#task 4.4 - after physical generation is implemented, it needs to be
  discussed in greater detail in the context of AKF 📅 2025-02-10
\end{itemize}

\autoref{chapter-five}

\begin{itemize}
\tightlist
\item
  5.1 - Consider removing this section and just having it be at the top
  level
\item[$\square$]
  \#task 5.2 - Write the entire section (how do we generate relevant
  outputs from the framework? how do we optimize these outputs for
  distribution and storage?) 📅 2025-02-08
\item
  5.3.3 - Show how the CASE bindings are actually used throughout AKF
\item
  5.4 - Write the entire section (human readable reporting) - also
  requires implementation
\end{itemize}

\autoref{chapter-six}

\begin{itemize}
\tightlist
\item
  6.2 - Demonstrate what simple usage of the AKF libraries look like
\item
  6.3 - Describe the design decisions and analysis of prior declarative
  languages, and also implement the declarative syntax + translator
  itself
\item
  6.4 - Describe using generative AI for artifacts
\item
  6.5 - Describing using generative AI for scenarios
\end{itemize}

\autoref{chapter-seven}

\begin{itemize}
\tightlist
\item
  The whole thing -- can't be worked on unless we make enough progress
\end{itemize}

\autoref{chapter-eight}

\begin{itemize}
\tightlist
\item
  The whole thing -- can't be worked on until we're close to running out
  of time and know what isn't happening
\end{itemize}

\autoref{chapter-nine}

\begin{itemize}
\tightlist
\item
  Should be like one paragraph or whatever, difficult to talk about
  until we're actually there
\end{itemize}

\begin{center}\rule{0.5\linewidth}{0.5pt}\end{center}

Chapters, v1

!\textbf{Chapters}
