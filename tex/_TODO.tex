See **\_Chapters\_v2**

\section{Things to address}\label{things-to-address}

Things to address

\begin{itemize}
\tightlist
\item
  Does the inclusion of the architectural diagram at the start of every
  chapter work, or should it just be included? Is it good/does it make
  sense/does it fit on the page? Do I need to describe it in greater
  detail, if I'm going to bother introducing it at all?
\item
  Does the introduction of each chapter work? Or is it too sudden if
  it's just like ``here's a diagram'', blah blah blah?
\end{itemize}

Mechanical things to address:

\begin{itemize}
\tightlist
\item
  Still need to figure out a good way to do tables, maybe we just have
  to do them manually
\item
  Also images, that's probably going to be kind of annoying
\item
  All the sections called ``overview'' conflict with each other and have
  the same label defined twice, which is Bad
\item
  \st{Code blocks also need to be done manually, which is a little
  annoying}

  \begin{itemize}
  \tightlist
  \item
    no longer is this the case, but the spacing and sizing is kinda
    annoying
  \item
    ==Do code blocks also have to adhere to the double spacing
    requirement? Do code blocks need a figure/listing number and a
    caption? Do they even belong in the thesis body?==
  \end{itemize}
\item
  Gotta standardize use of existing synthesizers - sometimes I wikilink
  them, sometimes I just italicize them, and that's inconsistent. Do I
  make a citation to the associated paper or repo every single time I
  mention it?

  \begin{itemize}
  \tightlist
  \item
    ==``Script and italic typefaces are not acceptable except where
    absolutely necessary i.e.~in Latin designations of species, etc.''==
    - so no italicization for emphasis or to designate names of tools?
    what's the convention for that?
  \end{itemize}
\end{itemize}

\section{Things to do}\label{things-to-do}

Things that can be done Right Now:

\begin{itemize}
\tightlist
\item
  5.2 (core outputs)
\item
  \st{8.2 (future work), things that are out of scope that we
  \emph{definitely} know won't happen and therefore don't have to wait
  until we're done with implementation to figure out what didn't get
  done}
\item
  \st{9 (conclusion), it's just like one paragraph lmao}
\item
  \st{put a link to the AKF repos\ldots{} somewhere lol}

  \begin{itemize}
  \tightlist
  \item
    currently it's in \autoref{contribution},
    surely there is a better place to do it\ldots{}
  \end{itemize}
\item
  apparently i just Forgot that \textbf{TraceGen} exists, so that has to
  be included where relevant
\item
  have all references used that are currently in the Forensic Image
  Synthesis library/bibliography
\item
  apparently i need like a glossary. and acronyms. and stuff. gotta make
  those Soon:tm:
\end{itemize}

Actual things that still have to be done:

\begin{itemize}
\tightlist
\item
  Analysis and review of ForTrace's declarative system

  \begin{itemize}
  \tightlist
  \item
    \ldots{} and implementation of a declarative-to-imperative system
  \end{itemize}
\item
  Integration of the CASE libraries EVERYWHERE (ideally)
\item
  Physical artifact generation (can \passthrough{\lstinline!pytsk!}
  really solve the issue of parsing and reading disk images?)
\item
  Finish up the VirtualBox concrete hypervisor API
\item
  Freshen up the Windows VM, develop a dedicated process for setting up
  a Windows VM and installing the agent on it for use
\item
  All the funny AI stuff
\end{itemize}

\section{How to do the things (for the
paper)}\label{how-to-do-the-things-for-the-paper}

Compilation notes:

\begin{itemize}
\tightlist
\item
  \passthrough{\lstinline!\\autoref!} is used to stick the word
  ``chapter'' and ``section'' into the actual reference. Don't do
  hyperlinks like ``refer to chapter \textbf{39.0 - Abstract}'', just
  use ``refer to \textbf{39.0 - Abstract}''
\item
  I'm pretty sure figures will have to be done manually, look for
  anything in the form \passthrough{\lstinline"!\\textbf\{...\}"}

  \begin{itemize}
  \tightlist
  \item
    This also includes the rendering of figures (e.g.~excalidraw to png)
    and uploading them into Overleaf, as an example
  \end{itemize}
\end{itemize}

Order of operations:

\begin{itemize}
\tightlist
\item
  Write the whole thing (code and paper), get the overall structure
  approved (so answer the questions above)
\item
  Make sure that the bibliography file contains only used references,
  rewrite so that this is the case
\item
  Do grammar checking by copying \emph{from Obsidian} to Grammarly, and
  apply edits in Obsidian
\item
  Use the thesis compiler to generate
  \passthrough{\lstinline!thesis.tex!}, copy it directly into Overleaf
\item
  Add/fix figures, code blocks, and other things as needed (this is the
  most time-consuming step, and hopefully only needs to be done once -
  use \passthrough{\lstinline!git diff!} on the base output if needed)
\item
  tada\ldots{}
\end{itemize}

\section{Actual chapter to-dos}\label{actual-chapter-to-dos}

\textbf{39.0 - Abstract} (100\%) Done (may require some light editing if
we don't fulfill all the promises made in the abstract)

\autoref{chapter-one} (100\%) Done

\autoref{chapter-two} (100\%) Done

\begin{itemize}
\tightlist
\item[$\boxtimes$]
  \#task 2.1 - Write this entire section, probably mostly from
  \cite{grajedaAvailabilityDatasetsDigital2017} 📅 2025-02-08 ✅
  2025-02-08
\end{itemize}

\autoref{chapter-three} (100\%) Done (unless the
architecture diagram needs to be reworked, either to fit in better with
the text or to actually move stuff around in the diagram itself)

\begin{itemize}
\tightlist
\item[$\boxtimes$]
  \#task 3 - Consider incorporating concepts explicitly from
  \cite{horsmanDatasetConstructionChallenges2021} 📅 2025-02-09 ✅
  2025-02-08
\item[$\boxtimes$]
  \#task 3.2 - Update the architecture diagram to look nicer, and remove
  all ``notes'' that are supposed to just be for me 📅 2025-02-09 ✅
  2025-02-08
\item[$\boxtimes$]
  \#task 3.2 - Consider making the connection between the ``three
  distinct concepts'' and the diagram to be more clear, most likely by
  editing or adding a simpler diagram that provides a closer 1-1
  connection between those components 📅 2025-02-09 ✅ 2025-02-08

  \begin{itemize}
  \tightlist
  \item
    that is, you'd have one big scary diagram, and then a simplified
    version of the same diagram
  \end{itemize}
\end{itemize}

\autoref{chapter-four} (\textasciitilde75\%)

\begin{itemize}
\tightlist
\item[$\boxtimes$]
  \#task 4 - Consider adding an inset diagram from the complete
  architectural diagram, and then briefly explaining what parts of the
  diagram correspond to which sections below; also consider moving the
  content into 4.1 📅 2025-02-08 ✅ 2025-02-08
\item
  4.2 - after more agentless generation is implemented, it needs to be
  discussed in greater detail in the context of AKF
\item
  4.3 - add any new application-specific information to the table

  \begin{itemize}
  \tightlist
  \item[$\square$]
    \#task by extension: achieve feature party with what's discussed in
    ForTrace page 11, at least as close as possible 📅 2025-02-09
  \end{itemize}
\item[$\square$]
  \#task 4.4 - after physical generation is implemented, it needs to be
  discussed in greater detail in the context of AKF 📅 2025-02-10
\end{itemize}

\autoref{chapter-five} (\textasciitilde30\%)

\begin{itemize}
\tightlist
\item
  5.1 - Consider removing this section and just having it be at the top
  level
\item[$\square$]
  \#task 5.2 - Write the entire section (how do we generate relevant
  outputs from the framework? how do we optimize these outputs for
  distribution and storage?) 📅 2025-02-08
\item
  5.3.3 - Show how the CASE bindings are actually used throughout AKF
\item
  5.4 - Write the entire section (human readable reporting) - also
  requires implementation
\end{itemize}

\autoref{chapter-six} (\textasciitilde20\%)

\begin{itemize}
\tightlist
\item
  6.2 - Demonstrate what simple usage of the AKF libraries look like
\item
  6.3 - Describe the design decisions and analysis of prior declarative
  languages, and also implement the declarative syntax + translator
  itself
\item
  6.4 - Describe using generative AI for artifacts
\item
  6.5 - Describing using generative AI for scenarios
\end{itemize}

\autoref{chapter-seven} (0\%)

\begin{itemize}
\tightlist
\item
  The whole thing -- can't be worked on \emph{at all} unless we make
  enough progress
\end{itemize}

\autoref{chapter-eight} (100\%) Done, except for some sentences that
may be dependent on things getting done or not

\autoref{chapter-nine} (100\%) Done, except for some sentences that
may be dependent on things getting done or not

\begin{center}\rule{0.5\linewidth}{0.5pt}\end{center}

Chapters, v1

!\textbf{Chapters}
