See **\_Chapters\_v2**

\section{The shortest term}\label{the-shortest-term}

Paper-related:

\begin{itemize}
\tightlist
\item
  Rework all architectural diagram-related stuff. Make appendix A, which
  has individual architectural diagrams.

  \begin{itemize}
  \tightlist
  \item
    At the start of each (sub)section, there should be a focus on just a
    specific part of the complex diagram, with non-related elements
    simplified.
  \item
    Each chapter will have the complete simple diagram at the top, with
    irrelevant components reduced in opacity.
  \item
    Make a label-less version of the complex diagram (which will allow
    it to fit, rotated, on a single page). This can go into the
    appendix, and will help orient the user to where things are. It
    should be followed by a series of complex diagram insets, describing
    each part.
  \end{itemize}
\item
  Rework text containing code blocks to only describe why things matter.
  Make appendix B, which has longer code blocks describing why specific
  structural changes are better. The appendix should (ideally?) have
  sections that can be referenced by the main text body.
\item
  As the below are finished, write their relevant sections.
\item
  Review text so that ``feature parity'' actually means ``feature parity
  with most synthesizers'' or something that doesn't mean ``full feature
  parity'', just ``good enough and it is possible to reimplement all
  existing features if there was more time''
\end{itemize}

\begin{center}\rule{0.5\linewidth}{0.5pt}\end{center}

Implementation:

\begin{itemize}
\tightlist
\item
  Implement the remaining hypervisor API stuff. More likely than not,
  we're going to fold on implementing direct disk writes.

  \begin{itemize}
  \tightlist
  \item
    There is still \emph{one} option, and that's seeing if pytsk can
    tell us the exact offset (and size) of a block with slack space so
    that we can just write directly to that space.
  \end{itemize}
\item
  Implement declarative wrappers for the hypervisor API stuff, as well
  as simple Chrome-related actions (like ``visit three websites, at some
  random interval, from this larger list of URLs'')
\item
  Add CASE integrations
\item
  Single AI artifact generation
\item
  AI full scenario generation
\item
  Bonus for the agent: individual applications, pyautogui, mouse,
  keyboard
\item
  GitHub action to automatically build AKF agent for multiple platforms
  on tagged commits; update the Vagrant file to point to the latest
  available Windows release (or a specific release is fine too)
\end{itemize}

\section{Things to address}\label{things-to-address}

Things to address

\begin{itemize}
\tightlist
\item
  Does the inclusion of the architectural diagram at the start of every
  chapter work, or should it just be included? Is it good/does it make
  sense/does it fit on the page? Do I need to describe it in greater
  detail, if I'm going to bother introducing it at all?
\item
  Does the introduction of each chapter work? Or is it too sudden if
  it's just like ``here's a diagram'', blah blah blah?
\end{itemize}

Mechanical things to address:

\begin{itemize}
\tightlist
\item
  Still need to figure out a good way to do tables, maybe we just have
  to do them manually
\item
  Also images, that's probably going to be kind of annoying
\item
  All the sections called ``overview'' conflict with each other and have
  the same label defined twice, which is Bad
\item
  \st{Code blocks also need to be done manually, which is a little
  annoying}

  \begin{itemize}
  \tightlist
  \item
    no longer is this the case, but the spacing and sizing is kinda
    annoying
  \item
    ==Do code blocks also have to adhere to the double spacing
    requirement? Do code blocks need a figure/listing number and a
    caption? Do they even belong in the thesis body?==

    \begin{itemize}
    \tightlist
    \item
      currently they're condensed in size to make them easier to read
    \item
      go to grad school and ask q
    \end{itemize}
  \item
    Guidance: \textbf{Limit inline code blocks, and make them as short
    as possible. Make your point with a code block \emph{once} - it's
    not supposed to be technical documentation. For longer code blocks,
    or those that require additional explanation, these belong in the
    appendix.}
  \end{itemize}
\item
  Gotta standardize use of existing synthesizers - sometimes I wikilink
  them, sometimes I just italicize them, sometimes I do nothing, and
  that's inconsistent. ==Do I make a citation to the associated paper or
  repo every single time I mention it?== \textbf{Answer: yes} (it won't
  hurt)

  \begin{itemize}
  \tightlist
  \item
    ==``Script and italic typefaces are not acceptable except where
    absolutely necessary i.e.~in Latin designations of species, etc.''==
    - so no italicization for emphasis or to designate names of tools?
    what's the convention for that? - ask the grad school
  \item
    Should my tools also go in the Glossary?
  \end{itemize}
\item
  ==Citing code repositories - do I need to go find the name of the
  person who made it? Or is what Zotero spits out fine?== (but what if
  it's \emph{a lot} of people?)

  \begin{itemize}
  \tightlist
  \item
    ==Also, do I even cite names of tools to begin with? If I mention
    that \passthrough{\lstinline!pycdlib!} is a thing, do I now need to
    cite it? Or do I just need to make a generic hyperlink to it?==
    \textbf{Answer: yes}; references aren't just to cover your ass, they
    may be actual references to interesting things as well. zotero's
    autocite is fine, and you can refer to the University of Arizona's
    guidelines for more detail.
  \end{itemize}
\item
  Glossary and acronym entries need to be ref-linked. Those use
  \passthrough{\lstinline!\\acrfull\{\}!} tags based on the names in the
  glossary.

  \begin{itemize}
  \tightlist
  \item
    ==Do I need a glossary and acronym listing?== optional, but if it
    would be helpful you can
  \item
    ==Should the names of synthesizers go into the glossary? What
    belongs in a glossary?== - sure, it wouldn't hurt
  \end{itemize}
\end{itemize}

\section{Things to do}\label{things-to-do}

Things that can be done Right Now:

\begin{itemize}
\tightlist
\item
  \st{5.2 (core outputs)}
\item
  \st{8.2 (future work), things that are out of scope that we
  \emph{definitely} know won't happen and therefore don't have to wait
  until we're done with implementation to figure out what didn't get
  done}
\item
  \st{9 (conclusion), it's just like one paragraph lmao}
\item
  \st{put a link to the AKF repos\ldots{} somewhere lol}

  \begin{itemize}
  \tightlist
  \item
    currently it's in \autoref{contribution},
    surely there is a better place to do it\ldots{}
  \end{itemize}
\item
  apparently i just Forgot that \textbf{TraceGen} exists, so that has to
  be included where relevant
\item
  have all references used that are currently in the Forensic Image
  Synthesis library/bibliography, or delete them
\item
  apparently i need like a glossary. and acronyms. and stuff. gotta make
  those soon, since i'll probably end up wikilinking them or
  smth\ldots{}
\item
  Fix the architectural diagrams as per Nancy's suggestions
\item
  Double-check and clean up anything that has a code block near it; does
  it solve the purpose that it's supposed to be solving?

  \begin{itemize}
  \tightlist
  \item
    I think we should move most/all of the code blocks, along with any
    significant explanations, to a dedicated appendix section for code
    snippets. This would also allow us to introduce that section with
    the links to the relevant repositories.
  \end{itemize}
\item
  Roll back the ``feature parity'' language to something like ``almost
  full feature parity'', since physical generation seems Extra annoying

  \begin{itemize}
  \tightlist
  \item
    but we can say somewhere that this is reflective of over 15 years of
    development in this field!
  \end{itemize}
\end{itemize}

Actual things that still have to be done:

\begin{itemize}
\tightlist
\item[$\boxtimes$]
  Analysis and review of ForTrace's declarative system
\item[$\boxtimes$]
  \ldots{} and implementation of a declarative-to-imperative system,
  with the bare minimum working
\item[$\square$]
  Integration of the CASE libraries EVERYWHERE (ideally)

  \begin{itemize}
  \tightlist
  \item
    Start with adding CASE bundle options to the agent and hypervisor
    APIs, and let's go from there
  \item
    Also includes managing CASE bundles as part of the declarative
    translator; presumably, this means that the bundle would be part of
    global state, and also that the metadata for CASE bundles would also
    have to be included in scenario definitions
  \end{itemize}
\item[$\square$]
  Physical artifact generation (can \passthrough{\lstinline!pytsk!}
  really solve the issue of parsing and reading disk images? what else?)

  \begin{itemize}
  \tightlist
  \item
    implication: pytsk is not suitable for this task because it's
    largely read-only; you can \emph{analyze} arbitrary filesystems
    through a very well-abstracted API, but it seems you can't write to
    anything. it'd probably have to be a manual implementation from one
    of the prior synthesizers but that seems like a lot of WORK (and may
    require some linux-specific tooling, which in turn requires docker)
  \end{itemize}
\item[$\square$]
  Logical agentless generation (finish up the VirtualBox concrete
  hypervisor API, includes all the dumps, human inputs, and flash drive
  ``from a folder'' logic)
\item[$\square$]
  AI stuff: single artifacts
\item[$\square$]
  AI stuff: a whole ass declarative scenario
\item[$\boxtimes$]
  Freshen up the Windows VM, develop a dedicated process for setting up
  a Windows VM and installing the agent on it for use

  \begin{itemize}
  \tightlist
  \item
    see \textbf{Setting up the VM}. seems to yield a stable VM
  \end{itemize}
\item[$\square$]
  Implement more application-specific stuff (this is the whole ``feature
  parity'' promise) - whatever is necessary to implement what's
  described below
\item[$\square$]
  Re-implement individual modules in the declarative-to-imperative
  system to implement whatever's necessary below

  \begin{itemize}
  \tightlist
  \item
    this also includes implementing new modules that serve as a wrapper
    around multiple features, like ``go visit a bunch of these websites
    at random''
  \end{itemize}
\end{itemize}

The ultimate goal: Develop a scenario in which all of the following
happens:

\begin{itemize}
\tightlist
\item
  A person copies a bunch of ``important'' documents to their device
  from a flash drive (which, ideally, is just a mountable folder
  converted to an ISO using something like
  \url{https://clalancette.github.io/pycdlib/,} and then
  \url{https://serverfault.com/questions/171665/how-to-attach-a-virtual-hard-disk-using-vboxmanage}
  to attach the iso)

  \begin{itemize}
  \tightlist
  \item
    These documents are generated using the single-artifact AI thing
  \end{itemize}
\item
  They interact with the internet, preferably doing a mix of web
  browsing and chatting with someone using the native email application
  (or something similar that leaves actual chat artifacts)
\item
  That other person sends them some ransomware
\item
  A bunch of those important documents disappear with the ransomware
\item
  The person pays the ransom by going to some website\ldots{} maybe? Or
  maybe by just going to some stores implying they purchased some gift
  cards, which they then visited a fictitious website to enter
\end{itemize}

The goal would be to have the student solve two things:

\begin{itemize}
\tightlist
\item
  Who did they pay the ransom to, and to what wallet/website/etc did
  they pay the ransom to (whether giftcard or crypto)?
\item
  Can you reverse engineer the ransomware to recover the documents? In
  particular, there's one document with some important details\ldots{}
\end{itemize}

this might be hard to convince AI to do, but we can just give examples,
I'm guessing; also, some routine parts, like ``browse on the internet
with a subset of the websites, then send some emails from the provided
dataset'' can be done using the scenario-wide AI

\section{How to do the things (for the
paper)}\label{how-to-do-the-things-for-the-paper}

Compilation notes:

\begin{itemize}
\tightlist
\item
  \passthrough{\lstinline!\\autoref!} is used to stick the word
  ``chapter'' and ``section'' into the actual reference. Don't do
  hyperlinks like ``refer to chapter \textbf{39.0 - Abstract}'', just
  use ``refer to \textbf{39.0 - Abstract}''
\item
  I'm pretty sure figures will have to be done manually, look for
  anything in the form \passthrough{\lstinline"!\\textbf\{...\}"}

  \begin{itemize}
  \tightlist
  \item
    This also includes the rendering of figures (e.g.~excalidraw to png)
    and uploading them into Overleaf, as an example
  \end{itemize}
\end{itemize}

Order of operations:

\begin{itemize}
\tightlist
\item
  Write the whole thing (code and paper), get the overall structure
  approved (so answer the questions above)
\item
  Make sure that the bibliography file contains only used references,
  rewrite so that this is the case
\item
  Do grammar checking by copying \emph{from Obsidian} to Grammarly, and
  apply edits in Obsidian
\item
  Use the thesis compiler to generate
  \passthrough{\lstinline!thesis.tex!}, copy it directly into Overleaf
\item
  Add/fix figures, code blocks, and other things as needed (this is the
  most time-consuming step, and hopefully only needs to be done once -
  use \passthrough{\lstinline!git diff!} on the base output if needed)
\item
  tada\ldots{}
\end{itemize}

\section{Actual chapter to-dos}\label{actual-chapter-to-dos}

deadline by april 1st, after spring break, for committee to review

\textbf{39.0 - Abstract} (\textasciitilde100\%) Done (may require some
light editing if we don't fulfill all the promises made in the abstract)

\autoref{chapter-one} (100\%) Done

\autoref{chapter-two} (100\%) Done

\begin{itemize}
\tightlist
\item[$\boxtimes$]
  \#task 2.1 - Write this entire section, probably mostly from
  \cite{grajedaAvailabilityDatasetsDigital2017} 📅 2025-02-08 ✅
  2025-02-08
\end{itemize}

\autoref{chapter-three} (\textasciitilde95\%) Done,
except for reworking chapters 3-6 based on the whole architectural
diagram stuff

\begin{itemize}
\tightlist
\item[$\boxtimes$]
  \#task 3 - Consider incorporating concepts explicitly from
  \cite{horsmanDatasetConstructionChallenges2021} 📅 2025-02-09 ✅
  2025-02-08
\item[$\boxtimes$]
  \#task 3.2 - Update the architecture diagram to look nicer, and remove
  all ``notes'' that are supposed to just be for me 📅 2025-02-09 ✅
  2025-02-08
\item[$\boxtimes$]
  \#task 3.2 - Consider making the connection between the ``three
  distinct concepts'' and the diagram to be more clear, most likely by
  editing or adding a simpler diagram that provides a closer 1-1
  connection between those components 📅 2025-02-09 ✅ 2025-02-08

  \begin{itemize}
  \tightlist
  \item
    that is, you'd have one big scary diagram, and then a simplified
    version of the same diagram
  \end{itemize}
\end{itemize}

\autoref{chapter-four} (\textasciitilde95\%) Done, except for
the architectural diagram stuff and any additional table rows for new
RPyC submodules

\begin{itemize}
\tightlist
\item[$\boxtimes$]
  \#task 4 - Consider adding an inset diagram from the complete
  architectural diagram, and then briefly explaining what parts of the
  diagram correspond to which sections below; also consider moving the
  content into 4.1 📅 2025-02-08 ✅ 2025-02-08
\item[$\boxtimes$]
  4.2 - after more agentless generation is implemented, it needs to be
  discussed in greater detail in the context of AKF

  \begin{itemize}
  \tightlist
  \item
    basically done, it's just the same as VMPOP
  \end{itemize}
\item
  4.3 - add any new application-specific information to the table

  \begin{itemize}
  \tightlist
  \item
    this can kinda just be written as we go, it's just individual table
    lines
  \end{itemize}
\item[$\boxtimes$]
  \#task 4.4 - after physical generation is implemented, it needs to be
  discussed in greater detail in the context of AKF 📅 2025-02-10 ✅
  2025-02-27
\end{itemize}

\autoref{chapter-five} (\textasciitilde75\%) Also needs
architectural diagram stuff

\begin{itemize}
\tightlist
\item
  5.1 - Consider removing this section and just having it be at the top
  level
\item[$\boxtimes$]
  5.2 - Write the entire section (how do we generate relevant outputs
  from the framework? how do we optimize these outputs for distribution
  and storage?)
\item[$\boxtimes$]
  5.3.3 - Show how the CASE bindings are actually used throughout AKF

  \begin{itemize}
  \tightlist
  \item
    i don't really show how they're used but it's at a high-enough level
    detail that you'll get the point -- anyways, that can be deferred to
    sections 6.2 and 6.3
  \end{itemize}
\item
  5.4 - Write the entire section (human readable reporting) - also
  requires some implementation
\item[$\boxtimes$]
  5.5 - need to talk about making scenarios reproducibility
\end{itemize}

\autoref{chapter-six} (\textasciitilde60\%) Also needs
architectural diagram stuff

\begin{itemize}
\tightlist
\item[$\square$]
  6.2 - Demonstrate what simple usage of the AKF libraries look like

  \begin{itemize}
  \tightlist
  \item
    Still need example imperative script of AKF doing some basic browser
    stuff and dumping out core outputs, preferably also using CASE
  \end{itemize}
\item[$\square$]
  6.3 - Describe the design decisions and analysis of prior declarative
  languages, and also implement the declarative syntax + translator
  itself

  \begin{itemize}
  \tightlist
  \item
    Still need example declarative script doing the same browser stuff,
    with brief explanation
  \end{itemize}
\item[$\square$]
  6.4 - Describe using generative AI for artifacts
\item[$\square$]
  6.5 - Describing using generative AI for scenarios
\end{itemize}

\autoref{chapter-seven} (0\%)

\begin{itemize}
\tightlist
\item
  The whole thing -- can't be worked on \emph{at all} unless we make
  enough progress
\end{itemize}

\autoref{chapter-eight} (\textasciitilde100\%) Basically done, might
need to change some language depending on what does/doesn't get done

\autoref{chapter-nine} (\textasciitilde100\%) Done, except for some
sentences that may be dependent on things getting done or not

\textbf{39.A - Architectural diagrams} (0\%) Need to rework everything,
break up the architectural diagrams and put them here (maybe)

\textbf{39.B - Code samples} (\textasciitilde100\%) Likely done unless
more code blocks need to be moved in here

\textbf{39.C - Acronyms} (?) Will take some work to integrate with rest
of thesis

\textbf{39.D - Glossary} (?) Will take some work to integrate with
thesis; also need to identify actual glossary terms to include

\begin{center}\rule{0.5\linewidth}{0.5pt}\end{center}

Chapters, v1

!\textbf{Chapters}
