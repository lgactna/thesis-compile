Public forensic datasets are invaluable to advancing research and
education throughout digital forensics. However, high-quality datasets
are presently few in number and may not fit specific needs, motivating
the development of new datasets. Constructing these datasets by hand is
time-consuming and prone to errors, yet it continues to be the primary
method through which new datasets are made. In turn, there is a need for
synthesizers, which allow users to create datasets using high-level
scripting languages that can automate many common actions. Although
prior synthesizers have addressed the creation of specific forensic
artifacts, there are still opportunities to improve their flexibility
and usability while promoting their usage throughout the forensic
community.

AKF, the \emph{automated kinetic framework}, introduces a modern
approach to forensic synthesis through a modular architecture focusing
on sustained development and community adoption. It provides significant
advancements in artifact generation, logging and validation, and the
overall construction of new scenarios. It improves artifact generation
by integrating numerous technologies not leveraged by prior
synthesizers, greatly simplifying the implementation of the overall
architecture without compromising the breadth of features available
through the framework. By using a centralized logging architecture and
the CASE ontology, AKF is able to generate detailed, queryable metadata
of the datasets it generates, allowing users to quickly identify
artifacts of interest in a dataset. AKF also exposes a simple
declarative syntax for generating artifacts, allowing users to develop
scenarios without writing code using the underlying Python
libraries.~Finally, we provide a demonstration of using generative AI
and AKF to further streamline scenario development, both when
constructing individual artifacts and when building a complete scenario.

Although there are still limitations to what AKF can accomplish,
numerous opportunities exist to leverage existing and emerging
technologies to extend the framework. These contributions have been made
in the hope that they will advance research and education throughout
digital forensics, allowing the community to fill known gaps in public
datasets.
