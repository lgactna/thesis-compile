Public forensic datasets are invaluable to advancing research and
education throughout the field of digital forensics. However,
high-quality datasets are few in number and may not fit specific needs,
motivating the development of new datasets. Constructing these datasets
by hand is time-consuming and prone to errors, yet continues to be the
primary method through which new datasets are made. In turn, there is a
need for a synthesizer that is capable of fulfilling the needs of
scenario developers while simultaneously promoting its usage throughout
the forensic community.

AKF introduces a modern approach to forensic synthesis through a modular
architecture with a focus on sustained development and community
adoption. It provides significant improvements in artifact generation,
logging and validation, and the overall construction of new scenarios.
It improves on artifact generation by integrating numerous technologies
not leveraged by prior synthesizers, greatly simplifying implementation
of the overall architecture without compromising the breadth of features
available through the framework. By using a centralized logging
architecture and the CASE standard, AKF is able to generate detailed,
queryable metadata of the datasets it generates, allowing users to
quickly identify artifacts of interest in a dataset. AKF also exposes a
simple declarative syntax for generating artifacts, allowing users to
develop scenarios without needing to write code. Finally, it provides a
demonstration of using LLMs to further streamline scenario development,
both when constructing individual artifacts and when building a complete
scenario.

Although there are still limitations in what AKF is able to accomplish,
there are numerous opportunities to leverage existing and emerging
technologies to extend the framework. These contributions have been made
in the hope that it will advance research and education throughout
digital forensics, allowing organizations to fill known gaps in public
datasets.
