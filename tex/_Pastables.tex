Contains a variety of things that need to be pasted back in every time
we have to rebuild the document.

\begin{center}\rule{0.5\linewidth}{0.5pt}\end{center}

\section{Full architectural diagram}\label{full-architectural-diagram}

The complete architectural diagram, as a figure, with autorefs -- it's
readable if you zoom into the PDF, but is pretty impossible to read on
paper within the size limits (i'm assuming). any larger, and it'll go
well past the margin limitations.

\begin{lstlisting}[language=TeX]
These four requirements are fulfilled by various mechanisms throughout
AKF. A complete architecture diagram is shown in \autoref{fig:architecture-full}. 

\begin{sidewaysfigure}[htbp]
    \centering
    \includegraphics[width=20cm]{Figures/architecture-full.png}
    \caption{Complete AKF architectural diagram}
    \label{fig:architecture-full}
\end{sidewaysfigure} 
\end{lstlisting}

!\textbf{Pasted image 20250209180929.png}

\section{The stupid tables}\label{the-stupid-tables}

Naturally, this means that the text in the table shall not be double
spaced because the thing is HUGE. Also try 0.2-0.2-0.6 for the table
widths. The row spacing is honestly fine with
\passthrough{\lstinline!\\small!}, but you can take it lower if
necessary. Note that this will bring it closer to the text of the
previous paragraph if you don't compensate.

\passthrough{\lstinline!\\small!}, on a paper that is 12pt default,
produces 10pt text. This is still within the limits of the filing
guidelines.

\begin{lstlisting}[language=TeX]
{
\small % Begin group to apply small font size
% \renewcommand{\arraystretch}{0.8} % Adjust row spacing
\setstretch{1} % Adjust line spacing (single spacing)
\begin{longtable}[]{@{}
  >{\raggedright\arraybackslash}p{(\linewidth - 4\tabcolsep) * \real{0.20}}
  >{\raggedright\arraybackslash}p{(\linewidth - 4\tabcolsep) * \real{0.20}}
  >{\raggedright\arraybackslash}p{(\linewidth - 4\tabcolsep) * \real{0.60}}@{}}
\caption{Supported applications for the AKF Windows agent}\label{table:akf-applications} \\
\toprule\noalign{}
\begin{minipage}[b]{\linewidth}\raggedright
Module
\end{minipage} & \begin{minipage}[b]{\linewidth}\raggedright
Dependencies
\end{minipage} & \begin{minipage}[b]{\linewidth}\raggedright
Details
\end{minipage} \\
\midrule\noalign{}
\endhead
\bottomrule\noalign{}
\endlastfoot
Chromium & Playwright \cite{MicrosoftPlaywrightpython2025} & Allows
arbitrary webpages to be visited on Chrome and Edge, as well as perform
complex actions such as completing forms and clicking links based on
HTML selectors \\
Chromium & Playwright \cite{MicrosoftPlaywrightpython2025} & Allows
arbitrary webpages to be visited on Chrome and Edge, as well as perform
complex actions such as completing forms and clicking links based on
HTML selectors \\
\end{longtable}
} % End group
\end{lstlisting}

With the exception of the caption and label, it should be possible to
just replace everything inside the
\passthrough{\lstinline!\{longtable\}!} environment itself.

The first table should be static so:

\begin{lstlisting}[language=TeX]
{
\small % Begin group to apply small font size
% \renewcommand{\arraystretch}{0.8} % Adjust row spacing
\setstretch{1} % Adjust line spacing (single spacing)
\begin{longtable}[]{@{}
  >{\raggedright\arraybackslash}p{(\linewidth - 6\tabcolsep) * \real{0.20}}
  >{\raggedright\arraybackslash}p{(\linewidth - 6\tabcolsep) * \real{0.2666}}
  >{\raggedright\arraybackslash}p{(\linewidth - 6\tabcolsep) * \real{0.2666}}
  >{\raggedright\arraybackslash}p{(\linewidth - 6\tabcolsep) * \real{0.2666}}@{}}
\caption{Summary of artifact generation techniques in prior synthesizers}\label{table:prior-techniques} \\
\toprule\noalign{}
\begin{minipage}[b]{\linewidth}\raggedright
Synthesizer
\end{minipage} & \begin{minipage}[b]{\linewidth}\raggedright
Physical
\end{minipage} & \begin{minipage}[b]{\linewidth}\raggedright
Agentless
\end{minipage} & \begin{minipage}[b]{\linewidth}\raggedright
Agent-based
\end{minipage} \\
\midrule\noalign{}
\endhead
\bottomrule\noalign{}
\endlastfoot
\textbf{FALCON} \cite{adelsteinAutomaticallyCreatingRealistic2005},
2005 & ? & ? & ? \\
\textbf{CYDEST} \cite{bruecknerAutomatedComputerForensics2008}, 2008
& ? & ? & ? \\
\textbf{Forensig2}
\cite{mochForensicImageGenerator2009,mochEvaluatingForensicImage2012},
2009 & Yes (filesystem mounting; filesystem-independent editing)
& Yes (over SSH only) & No \\
\textbf{D-FET} \cite{williamCloudbasedDigitalForensics2011}, 2011 &
Yes (filesystem mounting; filesystem-independent editing) & No &
No \\
\textbf{SFX} \cite{russellForensicImageDescription2012}, 2012 & Yes
(filesystem mounting) & No & No \\
\textbf{Yannikos et al.} \cite{yannikosDataCorporaDigital2014}, 2014
& ? & ? & ? \\
\textbf{ForGeOSI} \cite{maxfraggMaxfraggForGeOSI2023}, 2014 & No &
Yes (hypervisor interfaces) & No \\
\textbf{ForGe} \cite{vistiAutomaticCreationComputer2015}, 2015 & Yes
(filesystem-aware editing) & No & No \\
\textbf{ForGen} \cite{jjk422Jjk422ForGen2019}, 2016 & No & No &
No \\
\textbf{EviPlant} \cite{scanlonEviPlantEfficientDigital2017}, 2017 &
Yes (filesystem-independent editing) & No & Yes (unknown
mechanism) \\
\textbf{VMPOP} \cite{parkTREDEVMPOPCultivating2018}, 2018 & No & Yes
(hypervisor interfaces) & No \\
\textbf{hystck} \cite{gobelNovelApproachGenerating2020}, 2020 & No &
No & Yes (Python agent) \\
\textbf{TraceGen} \cite{duTraceGenUserActivity2021}, 2021 & No & No
& Yes (unknown mechanism) \\
\textbf{ForTrace} \cite{gobelForTraceHolisticForensic2022}, 2022 &
No & No & Yes (Python agent) \\
\end{longtable}
} % End group
\end{lstlisting}
