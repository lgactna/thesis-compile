\section{Code repositories}\label{code-repositories}

The complete implementation of AKF is available on GitHub at the
following locations:

\begin{itemize}
\tightlist
\item
  Core libraries (\passthrough{\lstinline!akflib!}):
  \url{https://github.com/lgactna/akflib}
\item
  The AKF agent for Windows:
  \url{https://github.com/lgactna/akf-windows}
\end{itemize}

Additionally, a standalone library for CASE/UCO 2.0 is provided at
\url{https://github.com/lgactna/CASE-pydantic,} which is used throughout
AKF.

\section{Comparison of ForTrace and AKF
agents}\label{comparison-of-fortrace-and-akf-agents}

Invoking commands in ForTrace entails issuing a simple string to the
agent running on the VM. For example, invoking a command to add a new
user to the VM may look like the following:

\begin{lstlisting}[language=Python]

# From Demo.py

guest = virtual_machine_monitor1.create_guest(guest_name=imagename, platform="windows")

# Wait for the VM to connect to the VMM

guest.waitTillAgentIsConnected()

# create userManagement object

userManagement_obj = guest.application("userManagement", {})

# Add different users

logger.info("Adding user1")
try:
    userManagement_obj.addUser("user1", "password")
    while userManagement_obj.is_busy is True:
        time.sleep(1)
except Exception as e:
    print("An error occured: ")
    print(e)
time.sleep(5)
\end{lstlisting}

The agent's main loop, which receives and interprets these commands, is
extremely simple:

\begin{lstlisting}[language=Python]
def main():
    # create logger
    logger = create_logger('guestAgent', logging.INFO)

    logger.info("create Agent")
    a = Agent(operating_system=platform.system().lower(), logger=logger)
    logger.info("connect to fortrace controller: %s:%i" % (fortrace_CONTROLLER_IP, fortrace_CONTROLLER_PORT))
    a.connect(fortrace_CONTROLLER_IP, fortrace_CONTROLLER_PORT)

    # let all network interfaces come up
    time.sleep(15)

    # inform fortrace controller about network configuration
    a.register()

    # wait for commands
    while 1:
        time.sleep(1)
        a.receiveCommands()
\end{lstlisting}

When the \passthrough{\lstinline!Agent!} is first instantiated, it forms
a permanent TCP socket out to the configured port and IP address where
it expects the server (the VMM) to be issuing commands.
\passthrough{\lstinline!Agent.receiveCommands()!} receives every message
available by reading the socket, parsing it, and then invoking
\passthrough{\lstinline!Agent.do\_command()!}, which converts the
command message into a specific Python function call with arguments.

The server, or ``virtual machine monitor'' (VMM), issues commands over
TCP to a running instance of the agent on the VM. Each module under
\passthrough{\lstinline!fortrace.application!} can be thought of as a
coherent group of commands associated with a particular \emph{real}
application (like Firefox or Thunderbird). Recall earlier how we found
our \passthrough{\lstinline!userManagement!} application and invoked
\passthrough{\lstinline!addUser!}:

\begin{lstlisting}[language=Python]
userManagement_obj = guest.application("userManagement", {})
userManagement_obj.addUser("user1", "password")
\end{lstlisting}

At a high level, the \passthrough{\lstinline!application()!} call
attempts to import
\passthrough{\lstinline!fortrace.application.\{application\_name\}!}.
Although not enforced by a higher-level interface, each of these modules
contains subclasses of the following four classes (defined in
\passthrough{\lstinline!fortrace.application.application!}) at minimum,
with possibly more helper classes for OS-specific functionality or other
modularity as needed:

\begin{itemize}
\tightlist
\item
  \passthrough{\lstinline!ApplicationVmmSide!}: Contains one function
  for each command implemented in
  \passthrough{\lstinline!ApplicationGuestSide!}, building a message
  that will be interpreted and acted upon by the corresponding
  \passthrough{\lstinline!ApplicationGuestSideCommands!} class.
\item
  \passthrough{\lstinline!ApplicationVmmSideCommands!}: Accepts and
  interprets module-specific messages returned by the agent, which may
  be used to update the remote state as tracked by the host.
\item
  \passthrough{\lstinline!ApplicationGuestSide!}: Implements the actual
  application-specific functionality for the agent, providing one
  function for each separated command by this module.
\item
  \passthrough{\lstinline!ApplicationGuestSideCommands!}: Interprets
  commands and arguments, calling the respective function in the
  corresponding \passthrough{\lstinline!ApplicationGuestSide!} subclass.
  This allows the actual dispatch of commands to be delegated to this
  class, which is free to choose how actions are performed (threading,
  multiprocessing, etc.) as well as any module-wide state it may need to
  maintain.
\end{itemize}

This naming convention is intentional, as individual modules dictate the
name of the module in camelcase. For example,
\passthrough{\lstinline!fortrace.application.userManagement!} contains
\passthrough{\lstinline!UserManagementVmmSide!},
\passthrough{\lstinline!UserManagementVmmSideCommands!}, and so on.
Discovering and getting handles to these classes is performed through
string manipulation of the relevant application's module name, as shown
in the example below:

\begin{lstlisting}[language=Python]

# Agent.do_command()

# load class moduleGuestSide and moduleGuestSideCommands
name = "fortrace." + package + "." + module
self.logger.debug("module to load: " + name)
mod = __import__(name, fromlist=[''])
self.logger.debug("module '" + module + "' will be loaded via __import__")
class_commands = getattr(mod, module[0].upper() + module[1:] + 'GuestSideCommands')
\end{lstlisting}

In the example above,
\passthrough{\lstinline!UserManagementGuestSide.addUser()!} contains the
literal code to execute on the guest when
\passthrough{\lstinline!addUser()!} is called, such as adding the
registry keys needed for a user to be created. On the other hand,
\passthrough{\lstinline!UserManagementVmmSide.addUser()!} contains the
code to send a message over TCP that the agent will understand,
eventually leading to the execution of the agent's version of
\passthrough{\lstinline!addUser()!}.

More precisely, the \passthrough{\lstinline!ApplicationVmmSide!}
subclass effectively serves as the API for calling the associated
functions in the \passthrough{\lstinline!ApplicationGuestSide!} class
running on the virtual machine. This subclass, along with the complete
agent-side code, is stored in a single file. For example, the API and
code for opening a Firefox browser window is as follows:

\begin{lstlisting}[language=Python]
class WebBrowserFirefoxVmmSide(ApplicationVmmSide):
    def open(self, url):
        """Sends a command to open a webBrowserFirefox on the associated guest.

        @param url: Website to open.
        """
        try:
            self.logger.info("function: WebBrowserFirefoxVmmSide::open")
            self.url = url
            self.window_id = self.guest_obj.current_window_id
            self.guest_obj.send(
                "application " + "webBrowserFirefox " + str(self.window_id) + " open " + self.webBrowserFirefox + " " + self.url)

            self.guest_obj.current_window_id += 1
        except Exception as e:
            raise Exception("error WebBrowserFirefoxVmmSide::open: " + str(e))

# Example usage:

browser = WebBrowserFirefoxVmmSide()
browser.open("google.com")
\end{lstlisting}

Calling the \passthrough{\lstinline!open()!} command from the host
causes a message of the following form to be sent to the agent:

\begin{lstlisting}
application webBrowserFirefox <window id
\end{lstlisting}

Upon receiving this message, the agent's main loop will search for the
\passthrough{\lstinline!webBrowserFirefox!} module and import its
corresponding \passthrough{\lstinline!ApplicationGuestSideCommands!}
subclass. After any state management, the subclass will then search for
a function called \passthrough{\lstinline!open!} in its corresponding
\passthrough{\lstinline!ApplicationGuestSide!} class, as implemented
below:

\begin{lstlisting}[language=Python]
class WebBrowserFirefoxGuestSide(ApplicationGuestSide):
    def open(self, args):
        # docstring omitted
        try:
            arguments = args.split(" ")
            web_browser = arguments[0]
            url = arguments[1]

            if len(arguments) 
                self.timeout = arguments[2]
            else:
                self.timeout = 30

            self.logger.info(self.module_name + "GuestSide::open")
            self.last_driven_url = url
            self.logger.debug("URL to call: " + url)

            self.logger.info("open url: " + url)

            self.helper.run_firefox()  # start ff session

            retval = self.helper.navigate_to_url(url)  # browse to the specified url
            if not retval:
                self.logger.warning("could not open url")

            self.agent_object.send("application " + self.module_name + " " + str(self.window_id) + " opened")

            self.agent_object.send("application " + self.module_name + " " + str(self.window_id) + " ready")
            self.window_is_crushed = False
        except Exception as e:
            # Some logging/teardown...
            self.window_is_crushed = True
            self.agent_object.send("application " + self.module_name + " " + str(self.window_id) + " error")
\end{lstlisting}

As described in \autoref{the-akf-agent}, this is achieved in AKF using RPyC services, which are analogous
to the \passthrough{\lstinline!ApplicationGuestSideCommands!} and
\passthrough{\lstinline!ApplicationGuestSide!} classes of individual
ForTrace modules. In addition to providing the RPyC services themselves,
agents also implement an API to call these services using typed
functions.

For example, suppose that we wanted to implement functionality similar
to the ForTrace code above, which allows users to navigate to a webpage.
An equivalent RPyC service and its corresponding API may be implemented
as follows:

\begin{lstlisting}[language=Python]

# Server-side code

class ChromiumService(AKFService):
    def exposed_set_browser(
        self, browser_type: Literal["msedge", "chrome"], profile: str = "Default"
    ) -
        # Various setup code...
        self.browser = chromium.launch_persistent_context(
            headless=False,
            user_data_dir=profile_path,
            channel=browser_type,
            args=[f"--profile-directory={profile}"],
        )

        return self.browser

# Client-side API

class ChromiumServiceAPI(WindowsServiceAPI):
    def __init__(self, host: str, port: int) -
        """
        Initialize the API with an RPyC connection to the service.
        """
        self.rpyc_conn = rpyc.connect(
            host,
            port,
            config={"sync_request_timeout": None},
        )

    def set_browser(
        self, browser_type: Literal["msedge", "chrome"], profile: str = "Default"
    ) -
        self.browser = self.rpyc_conn.root.set_browser(browser_type, profile)
        return self.browser
\end{lstlisting}

Observe that implementing support for a particular service on the
client-side API is as simple as calling an untyped function
\passthrough{\lstinline!rpyc\_conn.root.set\_browser()!}. By wrapping it
around a typed function,
\passthrough{\lstinline!ChromiumServiceAPI.set\_browser()!}, users
regain the ability to use code completion and static linting tools.

\section{CASE Python bindings}\label{case-python-bindings}

As described in \autoref{case-and-python-bindings}, CASE is defined using Turtle, allowing objects to be
written in a human-readable text format. For example, the following set
of triples describes an object called
\passthrough{\lstinline!ApplicationFacet!} with two properties,
\passthrough{\lstinline!numberOfLaunches!} and
\passthrough{\lstinline!applicationIdentifier!}:

\begin{lstlisting}
observable:ApplicationFacet
    a
        owl:Class ,
        sh:NodeShape
        ;
    rdfs:subClassOf core:Facet ;
    rdfs:label "ApplicationFacet"@en ;
    rdfs:comment "An application facet is a grouping of characteristics unique to a particular software program designed for end users."@en ;
    sh:property
        [
            sh:datatype xsd:integer ;
            sh:maxCount "1"^^xsd:integer ;
            sh:nodeKind sh:Literal ;
            sh:path observable:numberOfLaunches ;
        ] ,
        [
            sh:datatype xsd:string ;
            sh:maxCount "1"^^xsd:integer ;
            sh:nodeKind sh:Literal ;
            sh:path observable:applicationIdentifier ;
        ] ,
        ...
        ;
    sh:targetClass observable:ApplicationFacet ;
    .
\end{lstlisting}

An instance of a \passthrough{\lstinline!Application!} object may thus
be represented in the JSON-LD format using the
\passthrough{\lstinline!ApplicationFacet!} as follows (including
attributes omitted above):

\begin{lstlisting}
{
    "@id": "kb:dcec8d09-a8bc-4b7c-93ab-16c7b363d48b",
    "@type": "uco-observable:Application",
    "uco-core:hasFacet": [
        {
            "@id": "kb:68004de9-1139-405f-aea7-2c05f3a84709",
            "@type": "uco-observable:ApplicationFacet",
            "uco-observable:numberOfLaunches": 12,
            "uco-observable:applicationIdentifier": "test"
        },
    ]
}
\end{lstlisting}

The CASE project provides Python bindings, though it stores all object
attributes as dictionaries after instantiation. This can be seen in the
example of an \passthrough{\lstinline!ApplicationFacet!} below.
Intuitive usage suggests that the
\passthrough{\lstinline!application\_identifier!} attribute is
accessible through the \passthrough{\lstinline!facet!} object, but it
must instead be accessed as a dictionary key with a non-intuitive name.

\begin{lstlisting}[language=Python]
facet = ApplicationFacet(application_identifier = "test", number_of_launches=3)

# These attributes do not exist

facet.application_identifier
facet.number_of_launches

# The number of launches must be accessed as a dictionary key, which does

# not return a simple integer, but instead returns a dictionary
app_facet['uco-observable:numberOfLaunches']

# -
\end{lstlisting}

AKF's Pydantic-based bindings greatly simplify the declaration of
individual CASE objects while allowing for normal attribute-based
access. For example, AKF's declaration of
\passthrough{\lstinline!ApplicationFacet!} can be seen below:

\begin{lstlisting}[language=Python]
from typing import Optional

from uco import core

class ApplicationFacet(core.Facet):
    numberOfLaunches: Optional[int] = None
    applicationIdentifier: Optional[str] = None
\end{lstlisting}

This definition is only three lines, which is 40 lines shorter than the
declaration of \passthrough{\lstinline!ApplicationFacet!} provided by
the CASE project's existing Python bindings.

A simple CASE bundle, representing the complete contents of a forensic
scenario, can be constructed and exported to JSON-LD as shown below:

\begin{lstlisting}[language=Python]
from caselib import case, uco

bundle = uco.core.Bundle(
    description="An Example Case File",
    specVersion="UCO/CASE 2.0",
    tag="Sample artifacts",
)

url_object = uco.observable.ObservableObject()
url_facet = uco.observable.URLFacet(fullValue="www.docker.com/howto")
url_object.hasFacet.append(url_facet)
bundle.object.append(url_object)

with open("example.jsonld", "wt+") as f:
    data = bundle.model_dump(serialize_as_any=True)
    f.write(json.dumps(data, indent=2))
\end{lstlisting}

\section{Historical declarative
syntaxes}\label{historical-declarative-syntaxes}

While designing the declarative syntax for AKF, the syntaxes of D-FET,
SFX, and the work of Yannikos et al.~were reviewed. For completeness,
brief examples of scenarios in each of these synthesizers are included
here. Notably, none of these works provide details into how the parser
of their declarative languages are implemented, nor the architecture of
the code used to carry out the actions described in the examples below.
This may be attributed to the fact that the scenario creation enabled by
these declarative syntaxes, rather than the syntaxes themselves, was the
primary contribution of these works.

D-FET uses a custom language that does not depend on any existing
text-based languages \cite{williamCloudbasedDigitalForensics2011}:

\begin{lstlisting}
INSTANCE LOAD [Image=WINDOWS2003] 
MOUNT INSTANCE [Disk=STANDARDDISK] AS [Partition="c"] 
ACTIVITY LOAD [Number=12] [Type=JPEG IMAGES; Class=DRUGS] 
    INTO [Folder=USER FOLDER] 
    AT [Period=1 MINUTE] [Interval=INTERVAL] 
    FOR [User=Fred]
\end{lstlisting}

The scenario above will create an instance based on WINDOWS2003 from the
Host Forensics Image library (which contains predefined instances and
are created from OS installation discs). It will then load 12 .jpg
images from the ``DRUGS'' class of images, using the ``STANDARDDISK''
disk image. This creates a host with \emph{predefined}, but not
\emph{timed} activity -- the machine simply begins in this state, rather
than simulating a user doing this over some period of time.

To make activity that can be placed on a timeline, \emph{events} can be
added, such as logging in as a user, deleting files, and then logging
out.

\begin{lstlisting}
INSTANCE LOAD [Image=WINDOWS2003] 
MOUNT INSTANCE [Disk=STANDARDDISK] AS [Partition="c"] 
ACTIVITY LOAD [Number=12] [Type=JPEG IMAGES; Class=DRUGS] 
    INTO [Folder=USER FOLDER] 
    AT [Period=1 MINUTE] [Interval=INTERVAL] 
    FOR [User=Fred]
ACTIVITY EVENT [Event=LOGIN; User=Fred] 
ACTIVITY EVENT [Event=DELETEFILE; User=Fred; File=JPEF IMAGES] 
ACTIVITY EVENT [Event=LOGOUT; User=Fred ]
\end{lstlisting}

SFX uses an XML-based language with tags and attributes that are easily
readable \cite{russellForensicImageDescription2012}:

\begin{lstlisting}[language=XML]
<disk size="512M" alignment="cylinder" diskid="0x12345678"
    <partition index="p1" hidden="0" size="48M" type="vfat"
        <expand archive="part1.zip" /
        <copy from="fake.dat" to="/fake01.dat" /
        <copy from="fake.dat" to="/fake02.dat" /
        <delete from="/Thomas.jpg" /
    </partition
    <partition index="p2" hidden="0" size="48M" type="ntfs"
        <base os="windows7x64" /
        <copy from="fake.dat" to="/fake03.dat" /
        <copy from="fake.dat" to="/fake04.dat" /
        <user username="Gordon"
            <browserhistory browser="firefox"
                <url link="[http://bbc.co.uk"](http://bbc.co.uk") time="13:14:00 1 Jan 2013" /
            </browserhistory
        </user
    </partition
    <partition index="p3" hidden="1" size="64M" type="ntfs"
        <expand archive="part3.zip" /
        <copy from="fake.dat" to="/fake11.dat" /
        <copy from="fake.dat" to="/fake12.dat" /
        <delete from="/docs/image.exe" /
        <slackspace offset="20" file="/tomas.gif" message="This is a secret message" /
    </partition
    <partition index="s1" hidden="0" size="144M" type="ext3"
        <base os="fedora15x64" /
        <expand archive="part2.tar" /
        <copy from="fake.dat" to="/tmp/fake01.dat" /
        <copy from="fake.dat" to="/tmp/fake02.dat" /
    </partition
</disk
\end{lstlisting}

Here, a user is able to define multiple partitions on a single disk,
each with a distinct filesystem that may or may not contain an
underlying filesystem. SFX allows users to generate artifacts in
multiple ways, with each unique application- or OS-specific feature
using a distinct XML element name. Artifact generation features include
copying files to the guest machine in bulk, inserting files in the slack
space of an existing file, and using browser artifacts.

Finally, the work of Yannikos et al.~is particularly notable because it
appears to be fully GUI-based, expecting users to visually construct
Markov chains to define scenarios
\cite{yannikosDataCorporaDigital2014}. An example scenario from
their publication can be seen below:

!\textbf{Pasted image 20250207182912.png}

Although details are relatively limited, each node appears to be a
distinct action that can be automated. Individual nodes accept
parameters that can be used to configure how their associated artifacts
are created. Each ``box'' encompasses a complete Markov chain, with
transitions from one box to another occurring after an unknown condition
is fulfilled. Finally, the diamonds outside the boxes, known as
extensions, are additional libraries that provide additional
functionality during the execution of the overall scenario.
