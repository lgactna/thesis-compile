\section{Motivation}\label{motivation}

As described previously in \autoref{analysis-of-existing-synthesizers}, with the notable exception of
ForTrace \cite{gobelForTraceHolisticForensic2022} and its related
synthesizers, no synthesizer has been an extension of another
synthesizer. This raises the question -- why reinvent the wheel by
developing yet another distinct architecture for this thesis? This
section briefly explores the deficiencies in existing synthesizers and
explains why these are issues that warrant building a new architecture
from scratch rather than extending an existing synthesizer.

The motivations for developing entirely new codebases instead of
extending existing synthesizers have varied considerably. One reason is
that several synthesizers are not open source and thus cannot easily be
extended, as is the case with Forensig2
\cite{mochForensicImageGenerator2009} and TraceGen
\cite{duTraceGenUserActivity2021}. Another reason is that the focus
of certain synthesizers results in an architecture that is simply
incompatible with the goals of newer works. For example, ForGe's
architecture \cite{vistiAutomaticCreationComputer2015} focuses
largely on direct filesystem manipulation to generate forensic artifacts
and is unsuitable for a synthesizer requiring virtualization. Similarly,
synthesizers such as VMPOP \cite{parkTREDEVMPOPCultivating2018},
which exclusively leverage agentless artifact generation as described in
\autoref{agentless-artifact-generation},
require significant architectural changes to support agent-based
artifact generation.

However, perhaps the largest motivation for constructing new
synthesizers is the lack of ongoing support for virtually all
synthesizers. It appears that no synthesizer has gained significant
traction within the broader forensic community, possibly with the
exception of ForTrace; for the synthesizers that \emph{are} open source,
none are under active development and maintenance. Additionally, the
forensic datasets generated by these synthesizers have not seen
significant adoption in either education or research; many instructors
continue to use the human-generated datasets available on public
platforms.

The inflexibility of prior synthesizers, combined with the overall lack
of support and success of synthesizer-based datasets, contributes to the
disparate codebases that are now observed today. However, this is not to
say that the individual contributions of each prior synthesizer cannot
be merged into a single project that resolves many of the architectural
barriers that have limited the adoption and extension of existing
synthesizers.

In turn, AKF is built on the following four pillars to help promote its
long-term usage. In particular, these design principles allow it to
generate datasets that address several of the considerations raised by
Horsman and Lyle in the construction of various datasets, such as the
need for comprehensive documentation, an awareness of the ``realism'' of
the resulting dataset, and transparency in the scenario development
process \cite{horsmanDatasetConstructionChallenges2021}.

First, AKF conforms to modern Python development practices. This
includes the use of modern project management practices (such as the use
of \passthrough{\lstinline!uv!} \cite{AstralshUv2025} and
\passthrough{\lstinline!pyproject.toml!}, rather than the use of
\passthrough{\lstinline!setup.py!} observed in older synthesizers),
static linters (\passthrough{\lstinline!flake8!}
\cite{PyCQAFlake82025}), and style enforcers
(\passthrough{\lstinline!black!}
\cite{langaBlackUncompromisingPython2025} and
\passthrough{\lstinline!isort!} \cite{PyCQAIsort2025}) to promote
adherence to the PEP8 standard. Additionally, AKF's libraries make heavy
use of Python 3.11+ features, including type hinting and special type
annotations, which allow for tools such as
\passthrough{\lstinline!mypy!} \cite{PythonMypy2025} to perform
static type checking. In addition to significantly improving the
development experience, these practices also increase the likelihood of
discovering errors earlier in the development process.

Second, AKF takes a modular, agent-based approach to implementing
application-specific functionality. This allows AKF to use existing
automation frameworks for web browsers and other applications, greatly
simplifying the codebase when compared to the implementation of the same
features in other synthesizers. This focus on flexibility makes it
significantly easier to install the agent, implement new
application-specific features, and more. This is a critical design
focus, given AKF's dependence on agents for most of its
application-specific functionality.

Third, AKF is architected to maintain feature parity with most prior
synthesizers. That is, although AKF deviates considerably from the
implementation details of prior synthesizers, this does not come at the
loss of prior advancements in the field. In particular, AKF supports all
three artifact generation techniques described in \autoref{chapter-four} using hypervisor-agnostic interfaces, reducing the tight
coupling that made certain features challenging to implement across
different synthesizers. This reduces the likelihood that a new feature
or technique will require a significant architectural change to support
it.

Finally, AKF is designed with the explicit intent of developing an
ecosystem of AKF-generated datasets, promoting long-term usage. This is
reflected in the design of AKF's logging and reporting mechanisms as
described in \autoref{chapter-five}; the use of an
RDF-based standard, CASE, allows for arbitrarily complex queries against
AKF datasets. These queries can be made in bulk, significantly improving
the ability of researchers to find and use datasets that may be relevant
to them.

These four pillars are reflected throughout the design of AKF's
architecture, which is described in the following section and the
following three chapters of this thesis.

\section{Overview}\label{overview}

At a minimum, every synthesizer must fulfill these high-level
requirements through some means, derived mainly from the criteria
developed by Horsman and Lyle
\cite{horsmanDatasetConstructionChallenges2021}:

\begin{itemize}
\tightlist
\item
  The synthesizer must accept commands on how it should operate. These
  commands should serve as a form of self-documentation, in which it is
  clear to another person \emph{what} the expected contents of the image
  are, as well as the \emph{intent} behind these commands.
\item
  The synthesizer must be able to accept external inputs, such as files
  and other binary data, to be used as part of artifact and dataset
  generation.
\item
  The synthesizer must implement the mechanisms and technologies
  necessary to carry out the commands it has been provided -- that is,
  it must be able to generate artifacts. Such mechanisms should be
  independently verifiable and leave as little extraneous information as
  possible.
\item
  The synthesizer must be able to generate a final output (such as a
  disk image), along with ground truth and reporting that describes the
  expected contents of that image. This should include a well-structured
  description of the dataset, a unique identifier, and explicit labels
  applied to data of ``evidential value'' where possible.
\end{itemize}

These four requirements are fulfilled by various mechanisms throughout
AKF. A complete architecture diagram is shown in the figure below.

!\textbf{Architecture 2025-02-07 17.07.24.excalidraw}

\begin{itemize}
\tightlist
\item[$\square$]
  \#task move this to the appendix and break it up into individual
  colored boxes . . . simply can't be placed here
\end{itemize}

This can be simplified to the following high-level diagram, which
broadly groups AKF into a set of seven modules:

!\textbf{Architecture 2025-02-08 16.35.46.excalidraw}

At a high level, this architecture can be broken up into three distinct
concepts, each covering a distinct chapter.

The first set of modules is responsible for artifact generation. This
encompasses three major systems - a hypervisor and its associated SDK,
an OS-specific agent, and \passthrough{\lstinline!akflib!}.
\passthrough{\lstinline!akflib!} is a Python library containing the
abstract interfaces and concrete implementations necessary to generate
artifacts and datasets. This includes routines for directly interacting
with hypervisors, issuing commands to virtual machines, and directly
modifying disk images. This library is also the foundation for
OS-specific agents, which carry out actions on the virtual machine on
behalf of the host. These modules are described in greater detail in
\autoref{chapter-four} and correspond to the \emph{action
automation library}, the \emph{OS-specific agent}, and the \emph{virtual
machine} (as well as the \emph{user applications} running on the
machine).

The second set of modules is responsible for logging and reporting. This
encompasses independent libraries for generating outputs and ground
truth, as well as the various logging-related mechanisms contained
throughout the artifact generation libraries. These modules are
responsible for exporting and documenting artifacts generated by AKF; in
particular, they make heavy use of CASE, a standardized ontology for
documenting the contents of forensic datasets
\cite{caseyAdvancingCoordinatedCyberinvestigations2017}. This is
covered in \autoref{chapter-five} and corresponds to the
\emph{output and validation library} and any associated components
within the \emph{action automation library}.

The final set of modules is responsible for invoking AKF itself and
supporting scenario development. AKF is an \emph{imperative}
synthesizer, which means that commands are written and executed using an
imperative language (Python 3) that dictates precisely \emph{how}
scenarios should be constructed. However, AKF also supports a
\emph{declarative} syntax, which allows users to specify \emph{what}
forensic artifacts and datasets are generated without the need to
understand the AKF libraries and write Python code. Additionally, AKF
contains generative AI tools for constructing scenarios in the
declarative syntax as well as individual artifacts. This is discussed in
\autoref{chapter-six} and covers the \emph{scenario
construction library}, the \emph{translation unit}, and any scripts that
leverage AKF libraries.

The following three chapters will focus on these module groups. In
simpler terms, this thesis addresses the following questions in order:

\begin{itemize}
\tightlist
\item
  How do we automate or streamline the generation of artifacts?
\item
  How do we document and report on the artifacts and datasets that are
  generated?
\item
  Given the solutions that address these two questions, how do we
  actually use them to build scenarios?
\end{itemize}
