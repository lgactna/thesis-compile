\subsection{3.1 - Motivation}\label{motivation}

As described previously in \ref{analysis-of-existing-synthesizers}, with the notable exception of
\ref{agentless-artifact-generation}, require significant
architectural changes to support agent-based artifact generation.

However, perhaps the largest motivation for constructing new
synthesizers is simply the lack of ongoing support for virtually all
synthesizers. It appears that no synthesizer has gained significant
traction within the broader forensic community, possibly with the
exception of \textbf{ForTrace}; for the synthesizers that \emph{are}
open source, none of them are under active development and maintanance.
Additionally, the forensic datasets generated by these synthesizers have
not seen significant adoption in either education or research; many
instructors continue to use the human-generated datasets available on
public platforms.

The inflexibility of prior synthesizers, combined with the overall lack
of support and success of synthesizer-based datasets, contributes to the
disparate codebases that are now observed today. However, this is not to
say that the individual contributions of each prior synthesizer cannot
be merged into a single project that resolves many of the architectural
barriers that have reduced the adoption and extension of existing
synthesizers.

In turn, AKF is built on the following four pillars to help promote its
long-term usage. In particular, it allows it to generate datasets that
address several of the considerations raised by Horsman and Lyle in the
construction of various datasets, such as the need for comprehensive
documentation, an awareness of the ``realism'' of the resulting dataset,
and transparency in the scenario development process
\cite{horsmanDatasetConstructionChallenges2021}.

First, AKF conforms to modern Python development practices. This
includes the use of modern project management practices (such as the use
of \emph{uv} and \texttt{pyproject.toml}, rather than the use of
\texttt{setup.py} observed in older synthesizers), as well as static
linters (\emph{flake8}) and style enforcers (\emph{black}, \emph{isort})
to promote adherence to the PEP8 standard. Additionally, AKF's libraries
make heavy use of Python 3.11+ features, such as type hinting and
special type annotations, which allows for tools such as \emph{mypy} to
perform static type checking. In addition to greatly improving the
development experience, these practices also increase the likelihood of
discovering errors earlier in the development process.

Second, AKF takes a modular, agent-based approach to implementing
application-specific functionality. This allows AKF to use existing
automation frameworks for web browsers and other applications, greatly
simplifying the codebase when compared to the same features implemented
in other synthesizers. This focus on flexibility makes it significantly
easier to install the agent, implement new application-specific
features, and more - a particularly important design focus given AKF's
dependence on agents for the majority of its application-specific
functionality.

Third, AKF is architected to maintain feature parity with all prior
synthesizers. That is, although AKF deviates considerably from the
implementation details of prior synthesizers, this does not come at the
loss of prior advancements in the field. In particular, AKF supports all
three artifact generation techniques described in \textbf{39.4 - Action
automation} using hypervisor-agnostic interfaces, reducing the tight
coupling that made certain features difficult to implement in prior
synthesizers. This reduces the likelihood that a new feature or
technique will require a significant architectural change to support it.

Finally, AKF is designed with the explicit intent of developing an
ecosystem of AKF-generated datasets, promoting long-term usage. This is
reflected in the design of AKF's logging and reporting mechanisms as
described in \textbf{39.5 - Output and validation}; the use of an
RDF-based standard, CASE, allows for arbitrarily complex queries to be
made against AKF datasets. These queries can be made in bulk,
significantly improving the ability of researchers to find and use
datasets that may be relevant to them.

These four pillars reflect throughout the design of AKF's architecture,
which is described in the following section and the next three chapters
of this thesis.

\subsection{3.2 - Overview}\label{overview}

At a minimum, every synthesizer must fulfill these high-level
requirements through some means, largely derived from the criteria
developed by Horsman and Lyle
\cite{horsmanDatasetConstructionChallenges2021}:

\begin{itemize}
\item
  The synthesizer must accept commands on how it should operate. These
  commands should serve as a form of self-documentation, in which it is
  clear to another person \emph{what} the expected contents of the image
  are, as well as the intent behind these commands.
\item
  The synthesizer must be able to accept external inputs, such as files
  and other binary data, to be included in any final outputs.
\item
  The synthesizer must implement the mechanisms and technologies
  necessary to carry out the commands it has been provided -- that is,
  it must be able to generate artifacts. Such mechanisms should be
  independently verifiable, and preferably leave as little extraneous
  information as possible.
\item
  The synthesizer must be able to generate a final output (such as a
  disk image), along with ground truth and reporting that describes the
  expected contents of that image. This should include a well-structured
  description of the dataset, a unique identifier for the dataset, and
  an explicit identification of any data of ``evidential value'' where
  possible.
\item
  The synthesizer must be able to generate a final output (such as a
  disk image), along with ground truth and reporting that describes the
  expected contents of that image. This should include a well-structured
  description of the dataset, a unique identifier for the dataset, and
  an explicit identification of any data of ``evidential value'' where
  possible. hese four requirements are fulfilled by various mechanisms
  throughout AKF. A complete architecture diagram is shown in the figure
  below.
\end{itemize}

!\textbf{Architecture 2025-02-07 17.07.24.excalidraw}

This can be simplified to the following high-level diagram, which
broadly groups AKF into a set of seven modules:

!\textbf{Architecture 2025-02-08 16.35.46.excalidraw}

At a high level, this architecture can be broken up into three distinct
concepts, each of which covers a distinct chapter.

The first set of modules are responsible for artifact generation. This
encompasses three major systems - a hypervisor and its associated SDK,
an OS-specific agent, and \texttt{akflib}. \texttt{akflib} is a Python
library containing the abstract interfaces and concrete implementations
necessary to generate individual artifacts and complete datasets. This
includes routines for directly interacting with hypervisors, issuing
commands to virtual machines, and directly modifying virtual hard
drives. This library is also the foundation for OS-specific agents,
which carry out actions on the virtual machine on behalf of the host.
This is described in greater detail in \textbf{39.4 - Action
automation}, and corresponds to the \emph{action automation library},
the \emph{OS-specific agent}, and the \emph{virtual machine} (as well as
the \emph{user applications} running on the machine).

The second set of modules are responsible for logging and reporting.
This encompasses both independent libraries for generating outputs and
ground truth, as well as the various logging-related mechanisms that are
contained throughout the artifact generation libraries. These modules
are responsible for exporting and documenting artifacts generated by
AKF; in particular, it makes heavy use of CASE, a standardized ontology
for documenting the contents of forensic datasets
\cite{caseyAdvancingCoordinatedCyberinvestigations2017}. This is
covered in \textbf{39.5 - Output and validation}, and corresponds to the
\emph{output and validation library} and any associated components
within the \emph{action automation library}.

The final set of modules are responsible for invoking AKF itself and
supporting scenario development. AKF is an \emph{imperative}
synthesizer, which means that commands are written and executed using an
imperative language (here, Python 3) that dictates exactly \emph{how}
scenarios should be constructed. However, AKF also supports a
\emph{declarative} syntax, which allows users specifies \emph{what}
forensic artifacts and datasets are generated without the need to learn
the AKF libraries and write Python code. Additionally, AKF contains
generative AI tools for constructing scenarios in the declarative syntax
as well as individual artifacts. This is described in \textbf{39.6 -
Building scenarios}, and covers the \emph{scenario construction library}
and the \emph{translation unit}, as well as any scripts that leverage
AKF libraries.

The following three chapters will focus on these module groups. In
simpler terms, this thesis addresses the following questions in order:

\begin{itemize}
\item
  How do we automate or streamline the generation of artifacts?
\item
  How do we document and report on the artifacts and datasets that are
  generated?
\item
  How do we document and report on the artifacts and datasets that are
  generated? Given the solutions that address these two challenges, how
  do we actually use them to build scenarios?
\end{itemize}
